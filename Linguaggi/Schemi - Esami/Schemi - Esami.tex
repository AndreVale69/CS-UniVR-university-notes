\documentclass[a4paper]{article}
\usepackage[T1]{fontenc}			% pacchetto per \chapter
\usepackage[italian]{babel}
\usepackage[italian]{isodate}  		% formato delle date in italiano
\usepackage{graphicx}				% gestione delle immagini
\usepackage{amsfonts}
\usepackage{booktabs}				% tabelle di qualità superiore
\usepackage{amsmath}				% pacchetto matematica
\usepackage{mathtools}				% per sottolineare sotto le equazioni
\usepackage{stmaryrd} 				% per '\llbracket' e '\rrbracket'
\usepackage{amsthm}					% teoremi migliorati
\usepackage{enumitem}				% gestione delle liste
\usepackage{pifont}					% pacchetto con elenchi carini
\usepackage{enumitem}				% pacchetto per elenchi con lettere dell'alfabeto
\usepackage{cancel}					% per cancellare delle espressioni matematiche
\usepackage{caption}				% caption personalizzati
\usepackage[]{mdframed}				% box per il testo
\usepackage{multirow}				% più linee in una tabella
\usepackage{gensymb}				% simbolo di degree


% draw a frame around given text
\newcommand{\framedtext}[1]{%
	\par%
	\noindent\fbox{%
		\parbox{\dimexpr\linewidth-2\fboxsep-2\fboxrule}{#1}%
	}%
}



\usepackage[x11names]{xcolor}		% pacchetto colori RGB
% Link ipertestuali per l'indice
\usepackage{xcolor}
\usepackage[linkcolor=black, citecolor=blue, urlcolor=cyan]{hyperref}
\hypersetup{
	colorlinks=true
}

\usepackage{tikz}
\newcommand{\MyTikzmark}[2]{%
	\tikz[overlay,remember picture,baseline] \node [anchor=base] (#1) {#2};%
}
\newcommand{\DrawVLine}[3][]{%
	\begin{tikzpicture}[overlay,remember picture]
		\draw[shorten <=0.3ex, #1] (#2.north) -- (#3.south);
	\end{tikzpicture}
}
\newcommand{\DrawHLine}[3][]{%
	\begin{tikzpicture}[overlay,remember picture]
		\draw[shorten <=0.2em, #1] (#2.west) -- (#3.east);
	\end{tikzpicture}
}


%\usepackage{showframe}				% visualizzazione bordi
%\usepackage{showkeys}				% visualizzazione etichetta

\newtheorem{theorem}{\textcolor{Red3}{\underline{Teorema}}}
\newtheorem{lemma}{Lemma}
\renewcommand{\qedsymbol}{QED}
\newcommand{\exec}[1]{\llbracket #1\:\rrbracket}
\newcommand{\dquotes}[1]{``#1''}
\newcommand{\longline}{\noindent\rule{\textwidth}{0.4pt}}
\newcommand{\circledtext}[1]{\raisebox{.5pt}{\textcircled{\raisebox{-.9pt}{#1}}}}

\newenvironment{rowequmat}[1]{\left(\array{@{}#1@{}}}{\endarray\right)}
\newenvironment{rowequmatbra}[1]{\left[\array{@{}#1@{}}}{\endarray\right]}

\begin{document}
	\author{VR443470}
	\title{Guida agli Esami di Linguaggi}
	\date{\printdayoff\today}
	\maketitle
	
	\newpage
	
	% indice
	\tableofcontents
	
	\newpage
	
	\section{Esercizio 1 - Domanda di teoria su Interprete e Compilatore}
	
	\subsection{Interprete}
	
	In molti esami si presenta la richiesta della definizione di interprete. Nonostante possa essere banale, viene richiesto un \dquotes{alto} livello di approfondimento dato che vale ben 4 punti all'interno dell'esame. In ogni caso, è possibile affermare che questa domanda sia una delle più gettonate.
	
	\subsection{Compilatore}
	
	Non è frequente la richiesta della definizione di compilatore, ma rimane una domanda di teoria che può essere richiesta.
	
	\section{Esercizio 2 - Induzione}
	
	\subsection{Dimostrare $\forall n \in \mathbb{N}.n + n^{2}$ è un numero pari}
	
	\subsection{Dimostrare $\displaystyle\sum_{i=1}^{n}\dfrac{1}{i\left(i+1\right)} = \dfrac{n}{n+1}$}
	
	\subsection{Dimostrare $\displaystyle\sum_{i=0}^{n} i^{2} = \dfrac{n\left(n+1\right)\left(2n+1\right)}{6}$}
	
	\subsection{Dimostrare $\forall n \in \mathbb{N}. \: n > 2$ si ha che $n^{2} > 2n + 1$}
	
	\section{Esercizio 3 - Scoping statico e dinamico}
	
	\subsection{Tipologia codice 1}
	
	% 2023-02-01
	% 2022-09-19
	% 2022-09-19
	% 2020-02-03
	
	\subsection{Tipologia codice 2}
	
	% 2022-02-24
	
	\subsection{Tipologia codice 3}
	
	% 2021-07-13
	
	\subsection{Tipologia codice 4}
	
	% 2021-02-25
	% 2020-02-20
	
	\subsection{Tipologia codice 5}
	
	% 2021-02-08
	
	\subsection{Tipologia codice 6}
	
	% 2020-09-24
	
	\subsection{Tipologia codice 7}
	
	% 2020-09-24
	
	\subsection{Tipologia codice 8}
	
	% 2020-02-04
	
	\subsection{Tipologia codice 9}
	
	% 2023-06-28
	
	\section{Esercizio 4 - Scoping (statico/dinamico) e Binding}
	
	\subsection{Regole di scoping e di binding}
	
	\subsection{Codice da inserire in caso di scoping statico/dinamico}
	
	\subsubsection{Tipologia di codice 1}
	% 2022-02-24
	
	\subsubsection{Tipologia di codice 2}
	% 2023-06-28
	% 2020-02-20
	
	\subsubsection{Tipologia di codice 3}
	% 2020-09-24
	% 2020-02-24
	
	\section{Esercizio 5 - Ricorsione e passaggio di parametri}
	
	\subsection{Ricorsione e ricorsione in coda}
	
	\subsection{Passaggio di parametri: per valore e per riferimento}
	
	\subsubsection{Tipologia di codice 1}
	% 2023-06-28
	
	\subsubsection{Tipologia di codice 2}
	% 2023-02-01
	
	\subsubsection{Tipologia di codice 3}
	% 2021-02-08
	
	\subsubsection{Tipologia di codice 4}
	% 2020-02-04
	
	\section{Esercizio 6 - Regole della semantica dinamica}
	
	\subsection{Derivazioni semantica dinamica}
	
	\subsubsection{Tipologia di memoria 1}
	% 2023-06-28
	
	\subsubsection{Tipologia di memoria 2}
	% 2023-02-01
	% 2022-09-19
	% 2022-02-03
	
	\subsubsection{Tipologia di memoria 3}
	% 2021-02-08
	
	\subsubsection{Vecchi esercizi}
	% 2020-02-20
	% 2020-09-24
	% 2020-02-04
	
	\subsection{Regole della semantica dinamica per il comando condizionale}
	
	% 2022-02-24
	% 2021-02-25
	
	\subsection{Regole della semantica dinamica per l'assegnamento}
	
	% 2021-07-13
\end{document}