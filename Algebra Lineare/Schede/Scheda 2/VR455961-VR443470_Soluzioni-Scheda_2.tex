\documentclass[a4paper]{article}
\usepackage[T1]{fontenc}			% pacchetto per \chapter
\usepackage[italian]{babel}
\usepackage[italian]{isodate}  		% formato delle date in italiano
\usepackage{graphicx}				% gestione delle immagini
\usepackage{amsfonts}
\usepackage{booktabs}				% tabelle di qualità superiore
\usepackage{amsmath}				% pacchetto matematica
\usepackage{mathtools}				% per sottolineare sotto le equazioni
\usepackage{stmaryrd} 				% per '\llbracket' e '\rrbracket'
\usepackage{amsthm}					% teoremi migliorati
\usepackage{enumitem}				% gestione delle liste
\usepackage{pifont}					% pacchetto con elenchi carini
\usepackage{enumitem}				% pacchetto per elenchi con lettere dell'alfabeto
\usepackage{cancel}					% per cancellare delle espressioni matematiche

\usepackage{mathrsfs}

\usepackage[x11names]{xcolor}		% pacchetto colori RGB
% Link ipertestuali per l'indice
\usepackage{xcolor}
\usepackage[linkcolor=black, citecolor=blue, urlcolor=cyan]{hyperref}
\hypersetup{
	colorlinks=true
}

\usepackage{tikz}
\newcommand{\MyTikzmark}[2]{%
	\tikz[overlay,remember picture,baseline] \node [anchor=base] (#1) {#2};%
}
\newcommand{\DrawVLine}[3][]{%
	\begin{tikzpicture}[overlay,remember picture]
		\draw[shorten <=0.3ex, #1] (#2.north) -- (#3.south);
	\end{tikzpicture}
}
\newcommand{\DrawHLine}[3][]{%
	\begin{tikzpicture}[overlay,remember picture]
		\draw[shorten <=0.2em, #1] (#2.west) -- (#3.east);
	\end{tikzpicture}
}


%\usepackage{showframe}				% visualizzazione bordi
%\usepackage{showkeys}				% visualizzazione etichetta

\newtheorem{theorem}{\textcolor{Red3}{\underline{Teorema}}}
\newtheorem{lemma}{Lemma}
\renewcommand{\qedsymbol}{QED}
\newcommand{\exec}[1]{\llbracket #1\:\rrbracket}
\newcommand{\dquotes}[1]{``#1''}
\newcommand{\longline}{\noindent\rule{\textwidth}{0.4pt}}
\newcommand{\circledtext}[1]{\raisebox{.5pt}{\textcircled{\raisebox{-.9pt}{#1}}}}
\DeclareMathOperator{\rk}{rk}

\newenvironment{rowequmat}[1]{\left(\array{@{}#1@{}}}{\endarray\right)}
\newenvironment{rowequmatbra}[1]{\left[\array{@{}#1@{}}}{\endarray\right]}

\begin{document}
	\author{\begin{tabular}{ll}
			VR455961 & Davide Bragantini \\
			VR443470 & Andrea Valentini
		\end{tabular}
		   }
	\title{Università degli studi di Verona \\
			\:\\
			Soluzioni scheda 2}
	\date{\printdayoff\today}
	\maketitle
	
	\newpage
	
	% indice
	\tableofcontents
	
	\newpage
	
	\section{Soluzione esercizio 1}
	
	Sia $A_{k}$ la seguente matrice reale:
	\begin{equation*}
		A_{k} = \begin{pmatrix}
			2 & 2 & 2k \\
			k-1 & k & k^{2} \\
			-k & -k & 0
		\end{pmatrix}
	\end{equation*}
	
	\subsection{Soluzione \emph{a}}
	
	Si determini per quali valori di $k \in \mathbb{R}$ la matrice $A_{k}$ ammette inversa.\newline
	
	\noindent
	Una matrice quadrata a coefficienti in un campo dell'insieme $\mathbb{K}$ è invertibile se e solo se il suo determinante è diverso da zero. Quindi, si procede con il calcolo del determinante della matrice $A_{k}$. Visto che si tratta di una matrice di ordine $3$, si utilizza la regola di Sarrus per risolvere il determinante. Si duplica la matrice:
	\begin{equation*}
		\begin{matrix}
			2 	& 2		& 2k 	& 2 	& 2 	& 2k 		\\
			k-1 & k 	& k^{2} & k-1 	& k 	& k^{2} 	\\
			-k 	& -k	& 0		& -k 	& -k 	& 0
		\end{matrix}
	\end{equation*}
	Si sommano i prodotti lungo le prime tre diagonali principali:
	\begin{equation*}
		\begin{array}{lll}
			\text{1° diag} & : & \left(2 \cdot k \cdot 0\right) = 0 \\ [0.5em]
			\text{2° diag} & : & \left(2 \cdot k^{2} \cdot -k\right) = 2k^{2} \cdot -k = -2k^{3} \\[0.5em]
			\text{3° diag} & : & \left[2k \cdot \left(k-1\right) \cdot -k\right] = 2k \cdot \left(-k^{2} + k\right) = -2k^{3} + 2k^{2} \\ [0.5em]
			\text{Somma}   & : & 0 + \left(-2k^{3}\right) + \left(-2k^{3} + 2k^{2}\right) = -4k^{3} + 2k^{2}
		\end{array}
	\end{equation*}
	E si esegue lo stesso calcolo considerando le tre diagonali opposte:
	\begin{equation*}
		\begin{array}{lll}
			\text{1° diag opp} & : & \left(2k \cdot k \cdot -k\right) = -2k^{3} \\ [0.5em]
			\text{2° diag opp} & : & \left[2 \cdot \left(k-1\right) \cdot 0\right] = 0  \\ [0.5em]
			\text{3° diag opp} & : & \left(2 \cdot k^{2} \cdot -k\right) = -2k^{3} \\ [0.5em]
			\text{Somma}   & : & -2k^{3} + 0 + \left(-2k^{3}\right) = -4k^{3}
		\end{array}
	\end{equation*}
	Si esegue la sottrazione dei due risultati ottenuti mantenendo a sinistra quello della diagonale principale:
	\begin{equation*}
		\left(-4k^{3} + 2k^{2}\right) - \left(-4k^{3}\right) = 2k^{2}
	\end{equation*}
	Quindi il determinante è:
	\begin{equation*}
		\det\left(A_{k}\right) = 2k^{2}
	\end{equation*}
	La matrice $A_{k}$ ammette inversa per qualsiasi valore reale di $k$, poiché non esiste nessun valore (in $\mathbb{R}$) in grado di annullare l'espressione $2k^{2}$.\newpage
	
	\subsection{Soluzione \emph{b}}
	
	Sia $k \in \mathbb{R}$ tale che $A_{k}$ ammette inversa. Si calcoli $A_{k}^{-1}$ usando la formula $A_{k}^{-1} = \frac{1}{\det\left(A_{k}\right)} A_{k}^{*}$.\newline
	
	\noindent
	Per calcolare la matrice inversa si calcolano prima i complementi algebrici $\mathrm{Com}$:
	\begin{equation*}
		\begin{array}{ll}
			\mathrm{Com}\left(A_{11}\right) = & \left(-1\right)^{1+1} \cdot C_{11} = \left(-1\right)^{2} \cdot \det\begin{pmatrix}
				k & k^{2} \\
				-k & 0
			\end{pmatrix} = 1 \cdot \left[0 - \left(-k^{3}\right)\right] = k^{3} \\ [1.2em]
			
			\mathrm{Com}\left(A_{21}\right) = & \left(-1\right)^{2+1} \cdot C_{21} = \left(-1\right)^{3} \cdot \det\begin{pmatrix}
				2  & 2k \\
				-k & 0
			\end{pmatrix} = -1 \cdot \left[0 - \left(-2k^{2}\right)\right] = -2k^{2} \\ [1.2em]
			
			\mathrm{Com}\left(A_{31}\right) = & \left(-1\right)^{3+1} \cdot C_{31} = \left(-1\right)^{4} \cdot \det\begin{pmatrix}
				2 & 2k \\
				k & k^{2}
			\end{pmatrix} = 1 \cdot \left(2k^{2} - 2k^{2}\right) = 0 \\ [1.2em]
			
			\mathrm{Com}\left(A_{12}\right) = & \left(-1\right)^{1+2} \cdot C_{12} = \left(-1\right)^{3} \cdot \det\begin{pmatrix}
				k-1 & k^{2} \\
				-k  & 0
			\end{pmatrix} = -1 \cdot \left[0 - \left(-k^{3}\right)\right] = -k^{3} \\ [1.2em]
			
			\mathrm{Com}\left(A_{22}\right) = & \left(-1\right)^{2+2} \cdot C_{22} = \left(-1\right)^{4} \cdot \det\begin{pmatrix}
				2  & 2k \\
				-k & 0
			\end{pmatrix} = 1 \cdot \left[0 - \left(-2k^{2}\right)\right] = 2k^{2} \\ [1.2em]
			
			\mathrm{Com}\left(A_{32}\right) = & \left(-1\right)^{3+2} \cdot C_{32} = \left(-1\right)^{5} \cdot \det\begin{pmatrix}
				2   & 2k	\\
				k-1 & k^{2}
			\end{pmatrix} = -1 \cdot \left[2k^{2} - \left(2k^{2} - 2k\right)\right] = -2k\\ [1.2em]
			
			\mathrm{Com}\left(A_{13}\right) = & \left(-1\right)^{1+3} \cdot C_{13} = \left(-1\right)^{4} \cdot \det\begin{pmatrix}
				k-1 & k \\
				-k  & -k
			\end{pmatrix} = 1 \cdot \left[-k^{2} + k - \left(-k^{2}\right)\right] = k \\ [1.2em]
			
			\mathrm{Com}\left(A_{23}\right) = & \left(-1\right)^{2+3} \cdot C_{23} = \left(-1\right)^{5} \cdot \det\begin{pmatrix}
				2  & 2 \\
				-k & -k
			\end{pmatrix} = -1 \cdot \left[-2k - \left(-2k\right)\right] = 0 \\ [1.2em]
			
			\mathrm{Com}\left(A_{33}\right) = & \left(-1\right)^{3+3} \cdot C_{33} = \left(-1\right)^{6} \cdot \det\begin{pmatrix}
				2   & 2 \\
				k-1 & k
			\end{pmatrix} = 1 \cdot \left[2k - \left(2k - 2\right)\right] = -2
		\end{array}
	\end{equation*}
	Con i complementi algebrici si costruisce la matrice e si esegue la trasposta per ottenere $A_{k}^{*}$:
	\begin{equation*}
		A_{k}^{*} = \begin{pmatrix}
			k^{3}	& -k^{3}	& k \\
			-2k^{2} & 2k^{2}	& 0 \\
			0		& -2k		& -2
		\end{pmatrix}^{T} = \begin{pmatrix}
			k^{3} 	& -2k^{2} 	& 0 \\
			-k^{3}	& 2k^{2}	& -2k \\
			k		& 0			& -2
		\end{pmatrix}
	\end{equation*}
	Infine, si applica la formula:
	\begin{equation*}
		A_{k}^{-1} = \dfrac{1}{\det\left(A_{k}\right)} A_{k}^{*} = \dfrac{1}{2k^{2}} \begin{pmatrix}
			k^{3} 	& -2k^{2} 	& 0 \\
			-k^{3}	& 2k^{2}	& -2k \\
			k		& 0			& -2
		\end{pmatrix} = \begin{rowequmat}{ccc}
			\frac{k}{2}		& -1	&	0 			\\ [0.3em]
			-\frac{k}{2}	& 1		& -\frac{1}{k} 	\\ [0.3em]
			\frac{1}{2k}	& 0		& -\frac{1}{k^{2}}
		\end{rowequmat}
	\end{equation*}\newpage
	
	\section{Soluzione esercizio 2}
	
	Nello spazio vettoriale $\mathbb{R^{R}}$ definito nell'Esempio 5.2(2), si consideri il seguente sottoinsieme per ogni $t \in \mathbb{R}$:
	\begin{equation*}
		\mathscr{S}_{t} = \left\{f \in \mathbb{R^{R}} \: | \: f\left(0\right) = t\right\}
	\end{equation*}
	
	\subsection{Soluzione \emph{a}}
	
	Si trovino i valori di $t$ per cui l'insieme $\mathscr{S}_{t}$ è un sottospazio di $\mathbb{R^{R}}$.
	
	\subsection{Soluzione \emph{b}}
	
	\section{Soluzione esercizio 3}
	
	Sia $f: \mathbb{C}^{3} \rightarrow \mathbb{C}^{2}$ l'applicazione data da:
	\begin{equation*}
		f \left(
		\begin{pmatrix}
			 x \\ y \\ z
		\end{pmatrix}
		\right) = \begin{pmatrix}
			x-y+z \\
			3x-3y+3z
		\end{pmatrix}
	\end{equation*}
	Per ogni $\begin{pmatrix}
		x \\ y \\ z
	\end{pmatrix} \in \mathbb{C}^{3}$.
	\subsection{Soluzione \emph{a}}
	
	\subsection{Soluzione \emph{b}}
	
	\subsection{Soluzione \emph{c}}
	
	\subsection{Soluzione \emph{d}}
	
	\subsection{Soluzione \emph{e}}
	
	\subsection{Soluzione \emph{f}}
	
	\section{Soluzione esercizio 4}
	
	Sia $\mathscr{C}$ la base di $\mathbb{C}^{3}$ dell'esercizio 3(d) e sia $\mathscr{D} = \left\{u_{1}, u_{2}, u_{3}\right\}$ dove $u_{1} = \begin{pmatrix}
		1 \\ 0 \\ 1
	\end{pmatrix}, v_{2} = \begin{pmatrix}
		6 \\ -1 \\ 8
	\end{pmatrix}, v_{3} = \begin{pmatrix}
		-8 \\ -8 \\ 1
	\end{pmatrix}$. Si verifichi che $\mathscr{D}$ è una base di $\mathbb{C}^{3}$ e si calcoli la matrice del cambio di base $\mathscr{C} \rightarrow \mathscr{D}$.
\end{document}