\documentclass[a4paper]{article}
\usepackage[T1]{fontenc}			% pacchetto per \chapter
\usepackage[italian]{babel}
\usepackage[italian]{isodate}  		% formato delle date in italiano
\usepackage{graphicx}				% gestione delle immagini
\usepackage{amsfonts}
\usepackage{booktabs}				% tabelle di qualità superiore
\usepackage{amsmath}				% pacchetto matematica
\usepackage{mathtools}				% per sottolineare sotto le equazioni
\usepackage{stmaryrd} 				% per '\llbracket' e '\rrbracket'
\usepackage{amsthm}					% teoremi migliorati
\usepackage{enumitem}				% gestione delle liste
\usepackage{pifont}					% pacchetto con elenchi carini
\usepackage{enumitem}				% pacchetto per elenchi con lettere dell'alfabeto
\usepackage{cancel}					% per cancellare delle espressioni matematiche



\usepackage[x11names]{xcolor}		% pacchetto colori RGB
% Link ipertestuali per l'indice
\usepackage{xcolor}
\usepackage[linkcolor=black, citecolor=blue, urlcolor=cyan]{hyperref}
\hypersetup{
	colorlinks=true
}

\usepackage{tikz}
\newcommand{\MyTikzmark}[2]{%
	\tikz[overlay,remember picture,baseline] \node [anchor=base] (#1) {#2};%
}
\newcommand{\DrawVLine}[3][]{%
	\begin{tikzpicture}[overlay,remember picture]
		\draw[shorten <=0.3ex, #1] (#2.north) -- (#3.south);
	\end{tikzpicture}
}
\newcommand{\DrawHLine}[3][]{%
	\begin{tikzpicture}[overlay,remember picture]
		\draw[shorten <=0.2em, #1] (#2.west) -- (#3.east);
	\end{tikzpicture}
}


%\usepackage{showframe}				% visualizzazione bordi
%\usepackage{showkeys}				% visualizzazione etichetta

\newtheorem{theorem}{\textcolor{Red3}{\underline{Teorema}}}
\newtheorem{lemma}{Lemma}
\renewcommand{\qedsymbol}{QED}
\newcommand{\exec}[1]{\llbracket #1\:\rrbracket}
\newcommand{\dquotes}[1]{``#1''}
\newcommand{\longline}{\noindent\rule{\textwidth}{0.4pt}}
\newcommand{\circledtext}[1]{\raisebox{.5pt}{\textcircled{\raisebox{-.9pt}{#1}}}}
\DeclareMathOperator{\rk}{rk}

\newenvironment{rowequmat}[1]{\left(\array{@{}#1@{}}}{\endarray\right)}
\newenvironment{rowequmatbra}[1]{\left[\array{@{}#1@{}}}{\endarray\right]}

\begin{document}
	\author{VR443470 - Andrea Valentini}
	\title{Soluzioni scheda 1}
	\date{\printdayoff\today}
	\maketitle
	
	\newpage
	
	% indice
	\tableofcontents
	
	\newpage
	
	\section{Soluzione esercizio 1}
	
	Si considerino le seguenti matrici su $\mathbb{C}$:
	\begin{equation*}
		A = \begin{rowequmat}{cc}
			1 & -1 \\ [0.3em]
			i & \frac{1}{2}
		\end{rowequmat}, \hspace{1em}
		B = \begin{rowequmat}{cc}
			0 & 3i \\
			-1 & 0
		\end{rowequmat}, \hspace{1em}
		C = \begin{rowequmat}{cc}
			3 & 4 \\
			-2 & -2
		\end{rowequmat}, \hspace{1em}
		D = \begin{rowequmat}{cc}
			0 & -1 \\
			1+i & 2
		\end{rowequmat}
	\end{equation*}
	
	\subsection{Soluzione \emph{a}}
	
	Si determini la matrice risultante ottenibile eseguendo l'operazione $\left(CD\right)A$:
	\begin{equation*}
		\begin{array}{lll}
			\left(CD\right)A & = &
			\left[ \begin{rowequmat}{cc}
				3 & 4 \\
				-2 & -2
			\end{rowequmat} \cdot \begin{rowequmat}{cc}
				0 & -1 \\
				1+i & 2
			\end{rowequmat} \right] \cdot \begin{rowequmat}{cc}
				1 & -1 \\ [0.3em]
				i & \frac{1}{2}
			\end{rowequmat} \\
			\\
			&=& \begin{rowequmat}{cc}
				4+4i & 5 \\
				-2-2i & -2
			\end{rowequmat} \cdot \begin{rowequmat}{cc}
				1 & -1 \\ [0.3em]
				i & \frac{1}{2}
			\end{rowequmat} \\
			\\
			&=& \begin{rowequmat}{cc}
				4+9i  & -\frac{3}{2}-4i \\ [0.3em]
				-2-4i & 1+2i
			\end{rowequmat}
		\end{array}
	\end{equation*}
	
	\subsection{Soluzione \emph{b}}
	
	Si determini la matrice risultante ottenibile eseguendo l'operazione $A^{T} B$:
	\begin{gather*}
		A = \begin{rowequmat}{cc}
			1 & -1 \\ [0.3em]
			i & \frac{1}{2}
		\end{rowequmat} \longrightarrow
		A^{T} = \begin{rowequmat}{cc}
			1 & i \\ [0.3em]
			-1 & \frac{1}{2}
		\end{rowequmat}\\
		\\
		A^{T} B = \begin{rowequmat}{cc}
			1 & i \\ [0.3em]
			-1 & \frac{1}{2}
		\end{rowequmat} \cdot \begin{rowequmat}{cc}
			0 & 3i \\
			-1 & 0
		\end{rowequmat} =
		\begin{rowequmat}{cc}
			-i & 3i \\[0.3em]
			-\frac{1}{2} & -3i
		\end{rowequmat}
	\end{gather*}
	
	\subsection{Soluzione \emph{c}}
	
	Si determini la matrice risultante ottenibile eseguendo l'operazione $3A\left(B-D^{T}\right)$:
	\begin{gather*}
		D = \begin{rowequmat}{cc}
			0 	& -1 \\
			1+i & 2
		\end{rowequmat} \longrightarrow
		D^{T} = \begin{rowequmat}{cc}
			0 	& 1+i \\
			-1 	& 2
		\end{rowequmat} \longrightarrow
		-D^{T} = \begin{rowequmat}{cc}
			0 	& -1-i \\
			1 	& -2
		\end{rowequmat} \\
		\\
		\begin{array}{lll}
			3A\left(B-D^{T}\right) & = & 
			3\cdot\begin{rowequmat}{cc}
				1 & -1 \\ [0.3em]
				i & \frac{1}{2}
			\end{rowequmat} \cdot \left[
			\begin{rowequmat}{cc}
				0 & 3i \\
				-1 & 0
			\end{rowequmat} + \begin{rowequmat}{cc}
				0 	& -1-i \\
				1 	& -2
			\end{rowequmat}
			\right] \\
			\\
			&=& 3\cdot\begin{rowequmat}{cc}
				1 & -1 \\ [0.3em]
				i & \frac{1}{2}
			\end{rowequmat} \cdot \begin{rowequmat}{cc}
				0 & -1+2i \\
				0 & -2
			\end{rowequmat} \\
			\\
			&=& \begin{rowequmat}{cc}
				3 & -3 \\ [0.3em]
				3i & \frac{3}{2}
			\end{rowequmat} \cdot \begin{rowequmat}{cc}
				0 & -1+2i \\
				0 & -2
			\end{rowequmat} = \begin{rowequmat}{cc}
				0 & 3+6i \\
				0 & -9-3i
			\end{rowequmat}
		\end{array}
	\end{gather*}\newpage
	
	\subsection{Soluzione \emph{d}}
	
	Si determini la matrice risultante ottenibile eseguendo l'operazione $\left(4B-C\right)^{T}~-~DC$:
	\begin{equation*}
		\begin{array}{lll}
			\left(4B-C\right)^{T}-DC &=&
			\left[
			\begin{rowequmat}{cc}
				0  & 12i \\
				-4 & 0
			\end{rowequmat} - \begin{rowequmat}{cc}
				3 & 4 \\
				-2 & -2
			\end{rowequmat}
			\right]^{T} -
			\begin{rowequmat}{cc}
				0 & -1 \\
				1+i & 2
			\end{rowequmat}	\cdot
			\begin{rowequmat}{cc}
				3 & 4 \\
				-2 & -2
			\end{rowequmat} \\
			\\ %%%
			&=& \left[
			\begin{rowequmat}{cc}
				-3 & -4+12i \\
				-2 & 2
			\end{rowequmat}
			\right]^{T} - 
			\begin{rowequmat}{cc}
				2 & 2 \\
				-1+3i & 4i
			\end{rowequmat} \\
			\\ %%%
			&=& \begin{rowequmat}{cc}
				-3 & -2 \\
				-4+12i & 2
			\end{rowequmat} - \begin{rowequmat}{cc}
				2 & 2 \\
				-1+3i & 4i
			\end{rowequmat} \\
			\\ %%%
			&=& \begin{rowequmat}{cc}
				-5 & -4 \\
				-3+9i & 2-4i
			\end{rowequmat}
		\end{array}
	\end{equation*}\newpage
	
	\section{Soluzione esercizio 2}
	
	Si considerino le seguenti matrici su $\mathbb{R}$:
	\begin{gather*}
		A = \begin{rowequmat}{ccccc}
			0 & 0 & -1 & 4 & 0 \\
			0 & 2 & 0 & -2 & 0 \\
			0 & 0 & 3 & 0 & 3 \\
			0 & 0 & 0 & 0 & 5
		\end{rowequmat} \hspace{1em}
		B = \begin{rowequmat}{ccccccc}
			0 & 0 & 0 & 1 & 0 & 0 & 0 \\
			0 & 0 & 1 & 0 & 1 & 0 & 0 \\
			0 & 1 & 0 & 1 & 0 & 1 & 0 \\
			1 & 0 & 1 & 0 & 1 & 0 & 1
		\end{rowequmat} \\
		C = \begin{rowequmat}{ccc}
			2 & 0 & 1 \\
			-4 & 0 & -2 \\
			0 & 3 & 2 \\
			2 & -2 & 3 \\
			4 & 3 & 4
		\end{rowequmat} \hspace{1em}
		D = \begin{rowequmat}{ccc}
			2 & 0 & 1 \\
			-1 & 2 & 2 \\
			2 & 0 & 2
		\end{rowequmat}
	\end{gather*}
	
	\subsection{Soluzione \emph{a}}
	
	Si utilizza l'algoritmo di Eliminazione di Gauss per determinare una forma ridotta di ognuna delle matrici elencate.\newline
	
	\noindent
	La forma ridotta della \textbf{matrice} $\boldsymbol{A}$ è la seguente:
	\begin{gather*}
		\begin{rowequmat}{ccccc}
			0 & 0 & -1 & 4 & 0 \\
			0 & 2 & 0 & -2 & 0 \\
			0 & 0 & 3 & 0 & 3 \\
			0 & 0 & 0 & 0 & 5
		\end{rowequmat} \xlongrightarrow{E_{1,2}}
		\begin{rowequmat}{ccccc}
			0 & 2 & 0 & -2 & 0 \\
			0 & 0 & -1 & 4 & 0 \\
			0 & 0 & 3 & 0 & 3 \\
			0 & 0 & 0 & 0 & 5
		\end{rowequmat} \xlongrightarrow{E_{2,3}\left(3\right)}
		\begin{rowequmat}{ccccc}
			0 & 2 & 0 & -2 & 0 \\
			0 & 0 & -1 & 4 & 0 \\
			0 & 0 & 0 & 12 & 3 \\
			0 & 0 & 0 & 0 & 5
		\end{rowequmat} \\
		\\
		\xlongrightarrow[E_{1}\left(\frac{1}{2}\right)]{E_{2}\left(-1\right)}
		\begin{rowequmat}{ccccc}
			0 & 1 & 0 & -1 & 0 \\
			0 & 0 & 1 & -4 & 0 \\
			0 & 0 & 0 & 12 & 3 \\
			0 & 0 & 0 & 0 & 5
		\end{rowequmat} \xlongrightarrow[E_{4}\left(\frac{1}{5}\right)]{E_{3}\left(\frac{1}{12}\right)}
		\begin{rowequmat}{ccccc}
			0 & 1 & 0 & -1 & 0 \\
			0 & 0 & 1 & -4 & 0 \\
			0 & 0 & 0 & 1 & 4 \\
			0 & 0 & 0 & 0 & 1
		\end{rowequmat} \\
		\\
		\xlongrightarrow[E_{3,2}\left(4\right)]{E_{4,3}\left(-4\right)}
		\begin{rowequmat}{ccccc}
			0 & 1 & 0 & -1 & 0 \\
			0 & 0 & 1 & 0 & 0 \\
			0 & 0 & 0 & 1 & 0 \\
			0 & 0 & 0 & 0 & 1
		\end{rowequmat} \xlongrightarrow{E_{3,1}\left(1\right)}
		\begin{rowequmat}{ccccc}
			0 & 1 & 0 & 0 & 0 \\
			0 & 0 & 1 & 0 & 0 \\
			0 & 0 & 0 & 1 & 0 \\
			0 & 0 & 0 & 0 & 1
		\end{rowequmat}
	\end{gather*}
	La forma ridotta della \textbf{matrice} $\boldsymbol{B}$ è la seguente:
	\begin{gather*}
		\begin{rowequmat}{ccccccc}
			0 & 0 & 0 & 1 & 0 & 0 & 0 \\
			0 & 0 & 1 & 0 & 1 & 0 & 0 \\
			0 & 1 & 0 & 1 & 0 & 1 & 0 \\
			1 & 0 & 1 & 0 & 1 & 0 & 1
		\end{rowequmat} \xlongrightarrow[E_{3,2}]{E_{4,1}}
		\begin{rowequmat}{ccccccc}
			1 & 0 & 1 & 0 & 1 & 0 & 1 \\
			0 & 1 & 0 & 1 & 0 & 1 & 0 \\
			0 & 0 & 1 & 0 & 1 & 0 & 0 \\
			0 & 0 & 0 & 1 & 0 & 0 & 0
		\end{rowequmat} \\
		\\
		\xlongrightarrow{E_{4,2}\left(-1\right)}
		\begin{rowequmat}{ccccccc}
			1 & 0 & 1 & 0 & 1 & 0 & 1 \\
			0 & 1 & 0 & 0 & 0 & 1 & 0 \\
			0 & 0 & 1 & 0 & 1 & 0 & 0 \\
			0 & 0 & 0 & 1 & 0 & 0 & 0
		\end{rowequmat} \xlongrightarrow{E_{3,1}\left(-1\right)}
		\begin{rowequmat}{ccccccc}
			1 & 0 & 0 & 0 & 0 & 0 & 1 \\
			0 & 1 & 0 & 0 & 0 & 1 & 0 \\
			0 & 0 & 1 & 0 & 1 & 0 & 0 \\
			0 & 0 & 0 & 1 & 0 & 0 & 0
		\end{rowequmat}
	\end{gather*}\newpage
	
	\noindent
	La forma ridotta della \textbf{matrice} $\boldsymbol{C}$ è la seguente:
	\begin{gather*}
		\begin{rowequmat}{ccc}
			2 & 0 & 1 \\
			-4 & 0 & -2 \\
			0 & 3 & 2 \\
			2 & -2 & 3 \\
			4 & 3 & 4
		\end{rowequmat} \xlongrightarrow[E_{3,4}]{E_{5,2}\left(1\right)}
		\begin{rowequmat}{ccc}
			2 & 0 & 1 \\
			0 & 3 & 2 \\
			2 & -2 & 3 \\
			0 & 3 & 2 \\
			4 & 3 & 4
		\end{rowequmat} \xlongrightarrow[E_{2,5}\left(-1\right)]{E_{2,4}\left(-1\right)}
		\begin{rowequmat}{ccc}
			2 & 0 & 1 \\
			0 & 3 & 2 \\
			2 & -2 & 3 \\
			0 & 0 & 0 \\
			4 & 0 & 2
		\end{rowequmat} \\
		\\
		\xlongrightarrow{E_{1,5}\left(-2\right)}
		\begin{rowequmat}{ccc}
			2 & 0 & 1 \\
			0 & 3 & 2 \\
			2 & -2 & 3 \\
			0 & 0 & 0 \\
			0 & 0 & 0
		\end{rowequmat} \xlongrightarrow[E_{2,3}\left(\frac{2}{3}\right)]{E_{1,3}\left(-1\right)}
		\begin{rowequmat}{ccc}
			2 & 0 & 1 \\ [0.3em]
			0 & 3 & 2 \\ [0.3em]
			0 & 0 & \frac{10}{3} \\ [0.3em]
			0 & 0 & 0 \\ [0.3em]
			0 & 0 & 0
		\end{rowequmat} \xlongrightarrow[E_{3,1}\left(-1\right)]{E_{3}\left(\frac{3}{10}\right)}
		\begin{rowequmat}{ccc}
			2 & 0 & 0 \\ 
			0 & 3 & 2 \\ 
			0 & 0 & 1 \\ 
			0 & 0 & 0 \\ 
			0 & 0 & 0
		\end{rowequmat} \\
		\\
		\xlongrightarrow{E_{3,2}\left(-2\right)}
		\begin{rowequmat}{ccc}
			2 & 0 & 0 \\ 
			0 & 3 & 0 \\ 
			0 & 0 & 1 \\ 
			0 & 0 & 0 \\ 
			0 & 0 & 0
		\end{rowequmat} \xlongrightarrow[E_{2}\left(\frac{1}{3}\right)]{E_{1}\left(\frac{1}{2}\right)}
		\begin{rowequmat}{ccc}
			1 & 0 & 0 \\ 
			0 & 1 & 0 \\ 
			0 & 0 & 1 \\ 
			0 & 0 & 0 \\ 
			0 & 0 & 0
		\end{rowequmat}
	\end{gather*}
	La forma ridotta della \textbf{matrice} $\boldsymbol{D}$ è la seguente:
	\begin{gather*}
		\begin{rowequmat}{ccc}
			2  & 0  & 1 \\
			-1 & 2 & 2 	\\
			2  & 0  & 2
		\end{rowequmat} \xlongrightarrow{E_{1,3}\left(-1\right)}
		\begin{rowequmat}{ccc}
			2  & 0  & 1 \\
			-1 & 2 & 2 	\\
			0  & 0  & 1
		\end{rowequmat} \xlongrightarrow{E_{1,2}\left(\frac{1}{2}\right)}
		\begin{rowequmat}{ccc}
			2  & 0  & 1 \\ [0.3em]
			0  & 2  & \frac{5}{2} \\ [0.3em]
			0  & 0  & 1
		\end{rowequmat} \\
		\\
		\xlongrightarrow[E_{3,1}\left(-1\right)]{E_{3,2}\left(-\frac{5}{2}\right)}
		\begin{rowequmat}{ccc}
			2  & 0  & 0 \\ 
			0  & 2  & 0 \\ 
			0  & 0  & 1
		\end{rowequmat} \xlongrightarrow[E_{2}\left(\frac{1}{2}\right)]{E_{1}\left(\frac{1}{2}\right)}
		\begin{rowequmat}{ccc}
			1  & 0  & 0 \\ 
			0  & 1  & 0 \\ 
			0  & 0  & 1
		\end{rowequmat}
	\end{gather*}\newpage
	
	\subsection{Soluzione \emph{b}}\label{soluzione b}
	
	Date le forme ridotte $A'$, $B'$, $C'$, $D'$ delle rispettive matrici $A$, $B$, $C$ e $D$, il loro rango corrisponde al numero delle colonne dominanti, quindi:
	\begin{equation*}
		\begin{array}{lllll}
			A' & = & \begin{rowequmat}{ccccc}
				0 & 1 & 0 & 0 & 0 \\
				0 & 0 & 1 & 0 & 0 \\
				0 & 0 & 0 & 1 & 0 \\
				0 & 0 & 0 & 0 & 1
			\end{rowequmat} & \longrightarrow & \rk\left(A\right) = 4 \\
			\\
			B' & = & \begin{rowequmat}{ccccccc}
				1 & 0 & 0 & 0 & 0 & 0 & 1 \\
				0 & 1 & 0 & 0 & 0 & 1 & 0 \\
				0 & 0 & 1 & 0 & 1 & 0 & 0 \\
				0 & 0 & 0 & 1 & 0 & 0 & 0
			\end{rowequmat} & \longrightarrow & \rk\left(B\right) = 4 \\
			\\
			C' & = & \begin{rowequmat}{ccc}
				1 & 0 & 0 \\ 
				0 & 1 & 0 \\ 
				0 & 0 & 1 \\ 
				0 & 0 & 0 \\ 
				0 & 0 & 0
			\end{rowequmat} & \longrightarrow & \rk\left(C\right) = 3 \\
			\\
			D' & = & \begin{rowequmat}{ccc}
				1  & 0  & 0 \\ 
				0  & 1  & 0 \\ 
				0  & 0  & 1
			\end{rowequmat} & \longrightarrow & \rk\left(D\right) = 3
		\end{array}
	\end{equation*}\newpage
	
	\subsection{Soluzione \emph{c}}
	
	Si scrivano i sistemi lineari per cui le matrici indicate sopra sono le corrispondenti matrici aumentate, e si usi il Teorema di Rouché-Capelli per decidere se ognuno di questi sistemi ha o non ha soluzioni.\newline
	
	\noindent
	I sistemi lineari delle corrispondenti matrici aumentate sono:
	\begin{equation*}
		\begin{array}{lllll}
			A & = & \begin{rowequmat}{ccccc}
				0 & 0 & -1 & 4 & 0 \\
				0 & 2 & 0 & -2 & 0 \\
				0 & 0 & 3 & 0 & 3 \\
				0 & 0 & 0 & 0 & 5
			\end{rowequmat} & \longrightarrow & \begin{cases}
				-x_{3} + 4x_{4} = 0 \\
				2x_{2} - 2x_{4} = 0 \\
				3x_{3} = 3 \\
				0 = 5
			\end{cases} \\
			\\
			B & = & \begin{rowequmat}{ccccccc}
				0 & 0 & 0 & 1 & 0 & 0 & 0 \\
				0 & 0 & 1 & 0 & 1 & 0 & 0 \\
				0 & 1 & 0 & 1 & 0 & 1 & 0 \\
				1 & 0 & 1 & 0 & 1 & 0 & 1
			\end{rowequmat} & \longrightarrow & \begin{cases}
				x_{4} = 0 \\
				x_{3} + x_{5} = 0 \\
				x_{2} + x_{4} + x_{6} = 0 \\
				x_{1} + x_{3} + x_{5} = 1
			\end{cases} \\
			\\
			C & = & \begin{rowequmat}{ccc}
				2 & 0 & 1 \\
				-4 & 0 & -2 \\
				0 & 3 & 2 \\
				2 & -2 & 3 \\
				4 & 3 & 4
			\end{rowequmat} & \longrightarrow & \begin{cases}
				2x_{1} = 1 \\
				-4x_{1} = -2 \\
				3x_{2} = 2 \\
				2x_{1} - 2x_{2} = 3 \\
				4x_{1} + 3x_{2} = 4
			\end{cases} \\
			\\
			D & = & \begin{rowequmat}{ccc}
				2 & 0 & 1 \\
				-1 & 2 & 2 \\
				2 & 0 & 2
			\end{rowequmat} & \longrightarrow & \begin{cases}
				2x_{1} = 1 \\
				-x_{1} + 2x_{2} = 2 \\
				2x_{1} = 2
			\end{cases}
		\end{array}
	\end{equation*}
	Si applica il teorema di Rouché-Capelli per decidere se ogni sistema ha o non ha soluzioni.\newpage
	
	\noindent
	\begin{proof}[\textbf{Dimostrazione soluzioni matrice} $\boldsymbol{A}$]
		Considerando la matrice aumentata $\left(A|b\right)$ con il rango pari a $4$:
		\begin{equation*}
			\left(A|b\right) = \begin{rowequmat}{cccc|c}
				0 & 0 & -1 & 4 & 0 \\
				0 & 2 & 0 & -2 & 0 \\
				0 & 0 & 3 & 0 & 3 \\
				0 & 0 & 0 & 0 & 5
			\end{rowequmat}
		\end{equation*}
		Si esegue lo studio del rango della matrice incompleta $A$:
		\begin{equation*}
			A = \begin{rowequmat}{cccc}
				0 & 0 & -1 & 4 \\
				0 & 2 & 0 & -2 \\
				0 & 0 & 3 & 0 \\
				0 & 0 & 0 & 0
			\end{rowequmat}
		\end{equation*}
		Si procede con l'Eliminazione di Gauss per ottenere la forma ridotta e infine si calcola il rango tramite le colonne dominanti:
		\begin{gather*}
			\begin{rowequmat}{cccc}
				0 & 0 & -1 & 4 \\
				0 & 2 & 0 & -2 \\
				0 & 0 & 3 & 0 \\
				0 & 0 & 0 & 0
			\end{rowequmat} \xlongrightarrow[E_{2,3}]{E_{1,2}}
			\begin{rowequmat}{cccc}
				0 & 2 & 0 & -2 \\
				0 & 0 & 3 & 0 \\
				0 & 0 & -1 & 4 \\
				0 & 0 & 0 & 0
			\end{rowequmat}\xlongrightarrow{E_{2,3}\left(\frac{1}{3}\right)}
			\begin{rowequmat}{cccc}
				0 & 2 & 0 & -2 \\
				0 & 0 & 3 & 0 \\
				0 & 0 & 0 & 4 \\
				0 & 0 & 0 & 0
			\end{rowequmat} \\
			\\
			\xlongrightarrow{E_{3,1}\left(\frac{1}{2}\right)}
			\begin{rowequmat}{cccc}
				0 & 2 & 0 & 0 \\
				0 & 0 & 3 & 0 \\
				0 & 0 & 0 & 4 \\
				0 & 0 & 0 & 0
			\end{rowequmat} \xlongrightarrow[E_{3}\left(\frac{1}{4}\right)]{E_{1}\left(\frac{1}{2}\right), E_{2}\left(\frac{1}{3}\right)}
			\begin{rowequmat}{cccc}
				0 & 1 & 0 & 0 \\
				0 & 0 & 1 & 0 \\
				0 & 0 & 0 & 1 \\
				0 & 0 & 0 & 0
			\end{rowequmat}
		\end{gather*}
		Le colonne dominanti sono $3$, quindi il rango della matrice incompleta $A$ è $3$ ($\rk\left(A\right) = 3$). Il teorema di Rouché-Capelli afferma che se:
		\begin{equation*}
			\rk\left(A\right) < \rk\left(A|b\right)
		\end{equation*}
		Ovvero se il rango della matrice incompleta è minore del rango della matrice completa, allora il sistema è impossibile da risolvere, cioè non ammette soluzioni.
	\end{proof}\newpage

	\begin{proof}[\textbf{Dimostrazione soluzioni matrice} $\boldsymbol{B}$]
		Considerando la matrice aumentata $\left(B|b\right)$ con il rango pari a $4$:
		\begin{equation*}
			\left(B|b\right) = \begin{rowequmat}{cccccc|c}
				0 & 0 & 0 & 1 & 0 & 0 & 0 \\
				0 & 0 & 1 & 0 & 1 & 0 & 0 \\
				0 & 1 & 0 & 1 & 0 & 1 & 0 \\
				1 & 0 & 1 & 0 & 1 & 0 & 1
			\end{rowequmat}
		\end{equation*}
		Si esegue lo studio del rango della matrice incompleta $B$:
		\begin{equation*}
			B = \begin{rowequmat}{cccccc}
				0 & 0 & 0 & 1 & 0 & 0 \\
				0 & 0 & 1 & 0 & 1 & 0 \\
				0 & 1 & 0 & 1 & 0 & 1 \\
				1 & 0 & 1 & 0 & 1 & 0
			\end{rowequmat}
		\end{equation*}
		Si procede con l'Eliminazione di Gauss per ottenere la forma ridotta e infine si calcola il rango tramite le colonne dominanti:
		\begin{gather*}
			\begin{rowequmat}{cccccc}
				0 & 0 & 0 & 1 & 0 & 0 \\
				0 & 0 & 1 & 0 & 1 & 0 \\
				0 & 1 & 0 & 1 & 0 & 1 \\
				1 & 0 & 1 & 0 & 1 & 0
			\end{rowequmat} \xlongrightarrow[E_{3,2}]{E_{4,1}}
			\begin{rowequmat}{cccccc}
				1 & 0 & 1 & 0 & 1 & 0 \\
				0 & 1 & 0 & 1 & 0 & 1 \\
				0 & 0 & 1 & 0 & 1 & 0 \\
				0 & 0 & 0 & 1 & 0 & 0
			\end{rowequmat} \xlongrightarrow[E_{3,1}\left(-1\right)]{E_{4,2}\left(-1\right)}
			\begin{rowequmat}{cccccc}
				1 & 0 & 0 & 0 & 0 & 0 \\
				0 & 1 & 0 & 0 & 0 & 1 \\
				0 & 0 & 1 & 0 & 1 & 0 \\
				0 & 0 & 0 & 1 & 0 & 0
			\end{rowequmat}
		\end{gather*}
		Le colonne dominanti sono $4$, quindi il rango della matrice incompleta $B$ è $4$ ($\rk\left(B\right) = 4$). Il teorema di Rouché-Capelli afferma che se:
		\begin{equation*}
			\rk\left(B\right) = \rk\left(B|b\right)
		\end{equation*}
		Ovvero se il rango della matrice incompleta è uguale al rango della matrice completa, allora il sistema è compatibile, cioè ammette una o infinite soluzioni. In particolare, dato $n$ come il numero di incognite, cioè $6$, allora:
		\begin{equation*}
			\rk\left(B\right) = \rk\left(B|b\right) < n \Longrightarrow 4 = 4 < 6 \textcolor{Green4}{\textbf{ \checkmark}}
		\end{equation*}
		Il sistema ammette infinite soluzioni.
	\end{proof}\newpage
	
	\begin{proof}[\textbf{Dimostrazione soluzioni matrice} $\boldsymbol{C}$]
		Considerando la matrice aumentata $\left(C|b\right)$ con il rango pari a $3$:
		\begin{equation*}
			\left(C|b\right) = \begin{rowequmat}{cc|c}
				2 & 0 & 1 \\
				-4 & 0 & -2 \\
				0 & 3 & 2 \\
				2 & -2 & 3 \\
				4 & 3 & 4
			\end{rowequmat}
		\end{equation*}
		Si esegue lo studio del rango della matrice incompleta $C$:
		\begin{equation*}
			C = \begin{rowequmat}{cc}
				2 & 0 \\
				-4 & 0 \\
				0 & 3 \\
				2 & -2 \\
				4 & 3
			\end{rowequmat}
		\end{equation*}
		Si procede con l'Eliminazione di Gauss per ottenere la forma ridotta e infine si calcola il rango tramite le colonne dominanti:
		\begin{gather*}
			\begin{rowequmat}{cc}
				2 & 0 \\
				-4 & 0 \\
				0 & 3 \\
				2 & -2 \\
				4 & 3
			\end{rowequmat} \xlongrightarrow[E_{2,5}\left(-1\right), E_{1,3}\left(2\right)]{E_{3,2}, E_{3,5}\left(1\right)}
			\begin{rowequmat}{cc}
				2 & 0 \\
				0 & 3 \\
				0 & 0 \\
				2 & -2 \\
				0 & 0
			\end{rowequmat} \xlongrightarrow[E_{2,4}\left(\frac{2}{3}\right)]{E_{1,4}\left(-1\right)}
			\begin{rowequmat}{cc}
				2 & 0 \\
				0 & 3 \\
				0 & 0 \\
				0 & 0 \\
				0 & 0
			\end{rowequmat} \xlongrightarrow[E_{2}\left(\frac{1}{3}\right)]{E_{1}\left(\frac{1}{2}\right)}
			\begin{rowequmat}{cc}
				1 & 0 \\
				0 & 1 \\
				0 & 0 \\
				0 & 0 \\
				0 & 0
			\end{rowequmat}
		\end{gather*}
		Le colonne dominanti sono $2$, quindi il rango della matrice incompleta $C$ è $2$ ($\rk\left(C\right) = 2$). Il teorema di Rouché-Capelli afferma che se:
		\begin{equation*}
			\rk\left(C\right) < \rk\left(C|b\right)
		\end{equation*}
		Ovvero se il rango della matrice incompleta è minore del rango della matrice completa, allora il sistema è impossibile da risolvere, cioè non ammette soluzioni.
	\end{proof}\newpage

	\begin{proof}[\textbf{Dimostrazione soluzioni matrice} $\boldsymbol{D}$]
		Considerando la matrice aumentata $\left(D|b\right)$ con il rango pari a $3$:
		\begin{equation*}
			\left(D|b\right) = \begin{rowequmat}{cc|c}
				2 & 0 & 1 \\
				-1 & 2 & 2 \\
				2 & 0 & 2
			\end{rowequmat}
		\end{equation*}
		Si esegue lo studio del rango della matrice incompleta $D$:
		\begin{equation*}
			D = \begin{rowequmat}{cc}
				2 & 0 \\
				-1 & 2 \\
				2 & 0
			\end{rowequmat}
		\end{equation*}
		Si procede con l'Eliminazione di Gauss per ottenere la forma ridotta e infine si calcola il rango tramite le colonne dominanti:
		\begin{gather*}
			\begin{rowequmat}{cc}
				2 & 0 \\
				-1 & 2 \\
				2 & 0
			\end{rowequmat} \xlongrightarrow[E_{1,2}\left(\frac{1}{2}\right)]{E_{1,3}\left(-1\right)}
			\begin{rowequmat}{cc}
				2 & 0 \\
				0 & 2 \\
				0 & 0
			\end{rowequmat} \xlongrightarrow[E_{2}\left(\frac{1}{2}\right)]{E_{1}\left(\frac{1}{2}\right)}
			\begin{rowequmat}{cc}
				1 & 0 \\
				0 & 1 \\
				0 & 0
			\end{rowequmat}
		\end{gather*}
		Le colonne dominanti sono $2$, quindi il rango della matrice incompleta $D$ è $2$ ($\rk\left(D\right) = 2$). Il teorema di Rouché-Capelli afferma che se:
		\begin{equation*}
			\rk\left(D\right) < \rk\left(D|b\right)
		\end{equation*}
		Ovvero se il rango della matrice incompleta è minore del rango della matrice completa, allora il sistema è impossibile da risolvere, cioè non ammette soluzioni.
	\end{proof}\newpage

	\subsection{Soluzione \emph{d}}
	
	Si trovino tutte le soluzioni del sistema di equazioni lineari:
	\begin{equation*}
		\left(D|b\right) = D\begin{rowequmat}{c}
			x \\
			y \\
			z
		\end{rowequmat} = \begin{rowequmat}{c}
			1 \\
			-1 \\
			3
		\end{rowequmat}
	\end{equation*}
	Si scrive la matrice aumentata:
	\begin{equation*}
		\left(D|b\right) = \begin{rowequmat}{ccc|c}
			2 & 0 & 1 & 1 \\
			-1 & 2 & 2 & -1 \\
			2 & 0 & 2 & 3
		\end{rowequmat}
	\end{equation*}
	E si esegue l'Eliminazione di Gauss per ottenere una forma ridotta:
	\begin{gather*}
		\begin{rowequmat}{ccc|c}
			2 & 0 & 1 & 1 \\
			-1 & 2 & 2 & -1 \\
			2 & 0 & 2 & 3
		\end{rowequmat} \xlongrightarrow[E_{1,3}\left(-1\right)]{E_{3,2}\left(\frac{1}{2}\right)}
		\begin{rowequmat}{ccc|c}
			2 & 0 & 1 & 1 \\ [0.3em]
			0 & 2 & 3 & \frac{1}{2} \\ [0.3em]
			0 & 0 & 1 & 2
		\end{rowequmat} \xlongrightarrow[E_{3,1}\left(-1\right)]{E_{3,2}\left(-3\right)}
		\begin{rowequmat}{ccc|c}
			2 & 0 & 0 & -1 \\ [0.3em]
			0 & 2 & 0 & -\frac{11}{2} \\ [0.3em]
			0 & 0 & 1 & 2
		\end{rowequmat} \\
		\\
		\xlongrightarrow[E_{2}\left(\frac{1}{2}\right)]{E_{1}\left(\frac{1}{2}\right)}
		\begin{rowequmat}{ccc|c}
			1 & 0 & 0 & -\frac{1}{2} \\ [0.3em]
			0 & 1 & 0 & -\frac{11}{4} \\ [0.3em]
			0 & 0 & 1 & 2
		\end{rowequmat}
	\end{gather*}
	Il rango della matrice è pari al numero di colonne dominanti:
	\begin{equation*}
		\rk\left(D|b\right) = 3
	\end{equation*}
	Il rango della matrice incompleta $D$ è pari a $3$ (calcolato nel paragrafo~\ref{soluzione b}). Quindi, grazie al Teorema di Rouché-Capelli si ha:
	\begin{equation*}
		\rk\left(D|b\right) = \rk\left(D\right) = n \Longrightarrow 3 = 3 = 3 \textcolor{Green4}{\textbf{ \checkmark}}
	\end{equation*}
	Dove $n$ rappresenta il numero di incognite. Quindi, il sistema ammette un'unica soluzione che è quella trovata grazie all'Eliminazione di Gauss:
	\begin{equation*}
		\begin{rowequmat}{ccc|c}
			1 & 0 & 0 & -\frac{1}{2} \\ [0.3em]
			0 & 1 & 0 & -\frac{11}{4} \\ [0.3em]
			0 & 0 & 1 & 2
		\end{rowequmat} \Longrightarrow
		\begin{cases}
			x_{1} = -\frac{1}{2} \vspace{0.3em} \\
			x_{2} = -\frac{11}{4} \vspace{0.3em} \\
			x_{3} = 2
		\end{cases}
	\end{equation*}
	Il vettore soluzione:
	\begin{equation*}
		\begin{rowequmat}{c}
			x \\
			y \\
			z
		\end{rowequmat}
		= \begin{rowequmat}{c}
			-\frac{1}{2} \\ [0.3em]
			-\frac{11}{4} \\ [0.3em]
			2
		\end{rowequmat}
	\end{equation*}\newpage

	\section{Soluzione esercizio 3}
	
	Per ogni parametro $t$ in $\mathbb{R}$ si consideri la matrice:
	\begin{equation*}
		A_{t} \coloneq \begin{rowequmat}{cccc}
			1 & t & -1 & t \\
			2 & 2t & -1 & 3t+1 \\
			1 & -1 & -t & 2
		\end{rowequmat}
	\end{equation*}
	
	\subsection{Soluzione \emph{a}}
	
	Si calcola il rango di $A_{t}$ con $t = -1$. La matrice $A_{-1}$ è:
	\begin{equation*}
		A_{-1} = \begin{rowequmat}{cccc}
			1 & -1 & -1 & -1 \\
			2 & -2 & -1 & -2 \\
			1 & -1 &  1 &  2
		\end{rowequmat}
	\end{equation*}
	L'Eliminazione di Gauss porta alla seguente forma ridotta:
	\begin{gather*}
		\begin{rowequmat}{cccc}
			1 & -1 & -1 & -1 \\
			2 & -2 & -1 & -2 \\
			1 & -1 &  1 &  2
		\end{rowequmat} \xlongrightarrow[E_{1,2}\left(-2\right)]{E_{1,3}\left(-1\right)}
		\begin{rowequmat}{cccc}
			1 & -1 & -1 & -1 \\
			0 & 0 & 1 & 0 \\
			0 & 0 &  2 &  3
		\end{rowequmat} \xlongrightarrow[E_{2,1}\left(1\right)]{E_{2,3}\left(-2\right)}
		\begin{rowequmat}{cccc}
			1 & -1 &  0 & -1 \\
			0 &  0 &  1 &  0 \\
			0 &  0 &  0 &  3
		\end{rowequmat} \\
		\\
		\xlongrightarrow[E_{3,1}\left(1\right)]{E_{1,3}\left(2\right)}
		\begin{rowequmat}{cccc}
			1 & -1 &  0 &  0 \\
			0 &  0 &  1 &  0 \\
			0 &  0 &  0 &  1
		\end{rowequmat}
	\end{gather*}
	Il rango di $A_{-1}$ è pari a $3$ (numero delle colonne dominanti):
	\begin{equation*}
		\rk\left(A_{-1}\right) = 3
	\end{equation*}

	\subsection{Soluzione \emph{b}}
	
	Si calcola il rango di $A_{t}$ per ogni valore di $t$:
	\begin{equation*}
		A_{t} \coloneq \begin{rowequmat}{cccc}
			1 & t & -1 & t \\
			2 & 2t & -1 & 3t+1 \\
			1 & -1 & -t & 2
		\end{rowequmat}
	\end{equation*}
	Si applica l'Eliminazione di Gauss, che porta alla seguente forma ridotta:
	\begin{gather*}
		\begin{rowequmat}{cccc}
			1 & t & -1 & t \\
			2 & 2t & -1 & 3t+1 \\
			1 & -1 & -t & 2
		\end{rowequmat} \xlongrightarrow[E_{1,3}\left(-1\right)]{E_{1,2}\left(-2\right)}
		\begin{rowequmat}{cccc}
			1 & t & -1 & t \\
			0 & 0 &  1 & t+1 \\
			0 & -1-t & -t+1 & 2-t
		\end{rowequmat} \\
		\\
		\xlongrightarrow{E_{2,3}}
		\begin{rowequmat}{cccc}
			1 & t & -1 & t \\
			0 & -1-t & -t+1 & 2-t \\
			0 & 0 &  1 & t+1
		\end{rowequmat} \xlongrightarrow{E_{3,1}\left(1\right)}
		\begin{rowequmat}{cccc}
			1 & t & 0 & 2t+1 \\
			0 & -1-t & -t+1 & 2-t \\
			0 & 0 &  1 & t+1
		\end{rowequmat}
	\end{gather*}
	Il rango:
	\begin{equation*}
		\rk\left(A_{t}\right) = \begin{cases}
			3 & t = -1 \\
			3 & t \ne -1
		\end{cases}
	\end{equation*}
	Quindi il rango rimane a $3$ per ogni valore di $t \in \mathbb{R}$.\newpage
	
	\subsection{Soluzione \emph{c}}
	
	Supponiamo che la matrice $A_{t}$ sia la matrice aumentata di un sistema lineare su $\mathbb{R}$. Per quali valori di $t$ il sistema avrà soluzione?\newline
	
	\noindent
	Per rispondere a questa domanda, è possibile sfruttare il teorema di Rouché-Capelli. Quindi, data la matrice aumentata $\left(A_{t}|b\right)$ e il suo sistema corrispondente:
	\begin{equation*}
		\begin{rowequmat}{ccc|c}
			1 & t & -1 & t \\
			2 & 2t & -1 & 3t+1 \\
			1 & -1 & -t & 2
		\end{rowequmat} \Longrightarrow
		\begin{cases}
			x_{1} + tx_{2} - x_{3} = t \\
			2x_{1} + 2tx_{2} - x_{3} = 3t+1 \\
			x_{1} - x_{2} - tx_{3} = 2
		\end{cases}
	\end{equation*}
	La corrispondente matrice incompleta $A_{t}$ sarà composta da:
	\begin{equation*}
		\begin{rowequmat}{ccc}
			1 & t & -1  \\
			2 & 2t & -1 \\
			1 & -1 & -t
		\end{rowequmat}
	\end{equation*}
	Si calcola il rango della matrice incompleta tramite l'Eliminazione di Gauss:
	\begin{gather*}
		\begin{rowequmat}{ccc}
			1 & t & -1  \\
			2 & 2t & -1 \\
			1 & -1 & -t
		\end{rowequmat} \xlongrightarrow[E_{1,3}\left(-1\right)]{E_{1,2}\left(-2\right)}
		\begin{rowequmat}{ccc}
			1 & t & -1  \\
			0 & 0 &  1 \\
			0 & -1-t & 1-t
		\end{rowequmat} \xlongrightarrow[E_{3,1}\left(1\right)]{E_{2,3}}
		\begin{rowequmat}{ccc}
			1 & t & 0  \\
			0 & -1-t & 1-t \\
			0 & 0 &  1
		\end{rowequmat}
	\end{gather*}
	Il rango della matrice incompleta:
	\begin{equation*}
		\rk\left(A_{t}\right) = \begin{cases}
			2 & t = -1 \\
			3 & t \ne -1
		\end{cases}
	\end{equation*}
	Quindi, grazie al teorema di Rouché-Capelli è possibile affermare che il sistema lineare:
	\begin{equation*}
		\begin{cases}
			x_{1} + tx_{2} - x_{3} = t \\
			2x_{1} + 2tx_{2} - x_{3} = 3t+1 \\
			x_{1} - x_{2} - tx_{3} = 2
		\end{cases}
	\end{equation*}
	Avrà soluzione \textbf{solo} nel caso in cui $t$ sia diverso da $-1$, altrimenti il sistema non avrà soluzioni e sarà impossibile da risolvere.
	\begin{equation*}
		\begin{array}{lllllll}
			t = -1 	& \Longrightarrow & \rk\left(A_{t}\right) = 2 & \Longrightarrow & \rk\left(A_{t}\right) < \rk\left(A_{t}|b\right) & \Longrightarrow & \cancel{\exists} \text{ soluzioni} \\
			t \ne -1& \Longrightarrow & \rk\left(A_{t}\right) = 3 & \Longrightarrow & \rk\left(A_{t}\right) = \rk\left(A_{t}|b\right) & \Longrightarrow & \exists \text{ soluzioni} \\
		\end{array}
	\end{equation*}
	In particolare, sempre utilizzando il teorema di Rouché-Capelli, dato che il numero di incognite è pari a $3$ e il rango della matrice è $3$, esiste un'unica soluzione. Per esempio, prendendo $t = 0$ e sostituendo tale valore nella matrice $\left(A_{t}|b\right)$ con $A_{t}$ ridotto:
	\begin{gather*}
		\begin{rowequmat}{ccc|c}
			1 & t & 0 & t \\
			0 & -1-t & 1-t & 3t+1 \\
			0 & 0 &  1 & 2
		\end{rowequmat} \xlongrightarrow{\text{sostituzione}}
		\begin{rowequmat}{ccc|c}
			1 & 0 & 0 & 0 \\
			0 & -1 & 1 & 1 \\
			0 & 0 &  1 & 2
		\end{rowequmat} \\
		\\
		\xlongrightarrow{\text{sistema}}
		\begin{cases}
			x_{1} = 0 \\
			-x_{2} + x_{3} = 1 \\
			x_{3} = 2
		\end{cases} \longrightarrow
		\begin{cases}
			x_{1} = 0 \\
			x_{2} = 1\\
			x_{3} = 2
		\end{cases}
	\end{gather*}
	L'unica soluzione del sistema sarà:
	\begin{equation*}
		\begin{rowequmat}{c}
			0 \\
			1 \\
			2
		\end{rowequmat} \hspace{2em} \text{con } t \ne -1
	\end{equation*}\newpage

	\section{Soluzione esercizio 4}
	
	Si risolva la seguente equazione nell'insieme di numeri complessi $x^{4} + 1 = 0$:
	\begin{equation*}
		\begin{array}{rll}
			x^{4} + 1 & = & 0 \\
			x^{4} & = & -1 \\
			x & = & \sqrt[4]{-1} \\
			\\
			& \downarrow & \text{utilizzando la forma polare } r\left(\cos\theta + i \sin \theta\right) \\
			\\
			& = & \sqrt[4]{1 \left(\cos\left(\pi\right) + i \sin\left(\pi\right)\right)} \\
			\\
			& \downarrow & \text{calcolo della \emph{n}-esima radice usando } \sqrt[n]{r}\left(\cos\left(\frac{\theta+2k\pi}{n}\right) + i \sin\left(\frac{\theta+2k\pi}{n}\right)\right) \\
			\\
			& = & \sqrt[4]{1} \cdot \left(\cos\left(\dfrac{\pi + 2 k \pi}{4}\right) + i \sin\left(\dfrac{\pi + 2 k \pi}{4}\right)\right) \\
			\\
			& \downarrow & \text{sostituzione dei valori 0, 1, 2, 3 al posto di \emph{k}} \\
			\\
			& = & \begin{cases}
				x_{1} = \cancelto{1}{\sqrt[4]{1}} \cdot \left(\cos\left(\dfrac{\pi + 2 \cdot 0 \cdot \pi}{4}\right) + i \sin\left(\dfrac{\pi + 2 \cdot 0 \cdot \pi}{4}\right)\right)\\
				\\
				x_{2} = \cancelto{1}{\sqrt[4]{1}} \cdot \left(\cos\left(\dfrac{\pi + 2 \cdot 1 \cdot \pi}{4}\right) + i \sin\left(\dfrac{\pi + 2 \cdot 1 \cdot \pi}{4}\right)\right)\\
				\\
				x_{3} = \cancelto{1}{\sqrt[4]{1}} \cdot \left(\cos\left(\dfrac{\pi + 2 \cdot 2 \cdot \pi}{4}\right) + i \sin\left(\dfrac{\pi + 2 \cdot 2 \cdot \pi}{4}\right)\right)\\
				\\
				x_{4} = \cancelto{1}{\sqrt[4]{1}} \cdot \left(\cos\left(\dfrac{\pi + 2 \cdot 3 \cdot \pi}{4}\right) + i \sin\left(\dfrac{\pi + 2 \cdot 3 \cdot \pi}{4}\right)\right)
			\end{cases} \\
			\\
			& \downarrow & \text{qualche calcolo algebrico e fine dell'esercizio} \\
			\\
			& = & \begin{cases}
				x_{1} = \left(\cos\left(\dfrac{\pi}{4}\right) + i \sin\left(\dfrac{\pi}{4}\right)\right)\\
				\\
				x_{2} = \left(\cos\left(\dfrac{3\pi}{4}\right) + i \sin\left(\dfrac{3\pi}{4}\right)\right)\\
				\\
				x_{3} = \left(\cos\left(\dfrac{5\pi}{4}\right) + i \sin\left(\dfrac{5\pi}{4}\right)\right)\\
				\\
				x_{4} = \left(\cos\left(\dfrac{7\pi}{4}\right) + i \sin\left(\dfrac{7\pi}{4}\right)\right)
			\end{cases}
		\end{array}
	\end{equation*}\newpage

	\section{Soluzione esercizio 5}
	
	Si mostri che la trasposta $A^{T}$ e l'inversa $A^{-1}$ coincidono per $A = \begin{rowequmat}{cc}
		\cos\theta & -\sin\theta \\
		\sin\theta & \cos\theta
	\end{rowequmat}$ per ogni $\theta$.\newline

	\noindent
	La matrice trasposta:
	\begin{equation*}
		A = \begin{rowequmat}{cc}
			\cos\theta & -\sin\theta \\
			\sin\theta & \cos\theta
		\end{rowequmat} \longrightarrow
		A^{T} = \begin{rowequmat}{cc}
			\cos\theta & \sin\theta \\
			-\sin\theta & \cos\theta
		\end{rowequmat}
	\end{equation*}
	La matrice inversa:
	\begin{equation*}
		A = \begin{rowequmat}{cc}
			\cos\theta & -\sin\theta \\
			\sin\theta & \cos\theta
		\end{rowequmat}
	\end{equation*}
\end{document}