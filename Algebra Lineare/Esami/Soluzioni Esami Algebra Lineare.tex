\documentclass[a4paper]{article}
\usepackage[T1]{fontenc}			% pacchetto per \chapter
\usepackage[italian]{babel}
\usepackage[italian]{isodate}  		% formato delle date in italiano
\usepackage{graphicx}				% gestione delle immagini
\usepackage{amsfonts}
\usepackage{booktabs}				% tabelle di qualità superiore
\usepackage{amsmath}				% pacchetto matematica
\usepackage{mathtools}				% per sottolineare sotto le equazioni
\usepackage{stmaryrd} 				% per '\llbracket' e '\rrbracket'
\usepackage{amsthm}					% teoremi migliorati
\usepackage{enumitem}				% gestione delle liste
\usepackage{pifont}					% pacchetto con elenchi carini
\usepackage{enumitem}				% pacchetto per elenchi con lettere dell'alfabeto
\usepackage{cancel}					% per cancellare delle espressioni matematiche



\usepackage[x11names]{xcolor}		% pacchetto colori RGB
% Link ipertestuali per l'indice
\usepackage{xcolor}
\usepackage[linkcolor=black, citecolor=blue, urlcolor=cyan]{hyperref}
\hypersetup{
	colorlinks=true
}

\usepackage{tikz}
\newcommand{\MyTikzmark}[2]{%
	\tikz[overlay,remember picture,baseline] \node [anchor=base] (#1) {#2};%
}
\newcommand{\DrawVLine}[3][]{%
	\begin{tikzpicture}[overlay,remember picture]
		\draw[shorten <=0.3ex, #1] (#2.north) -- (#3.south);
	\end{tikzpicture}
}
\newcommand{\DrawHLine}[3][]{%
	\begin{tikzpicture}[overlay,remember picture]
		\draw[shorten <=0.2em, #1] (#2.west) -- (#3.east);
	\end{tikzpicture}
}


%\usepackage{showframe}				% visualizzazione bordi
%\usepackage{showkeys}				% visualizzazione etichetta

\newtheorem{theorem}{\textcolor{Red3}{\underline{Teorema}}}
\newtheorem{lemma}{Lemma}
\renewcommand{\qedsymbol}{QED}
\newcommand{\exec}[1]{\llbracket #1\:\rrbracket}
\newcommand{\dquotes}[1]{``#1''}
\newcommand{\longline}{\noindent\rule{\textwidth}{0.4pt}}
\newcommand{\circledtext}[1]{\raisebox{.5pt}{\textcircled{\raisebox{-.9pt}{#1}}}}

\newenvironment{rowequmat}[1]{\left(\array{@{}#1@{}}}{\endarray\right)}
\newenvironment{rowequmatbra}[1]{\left[\array{@{}#1@{}}}{\endarray\right]}

\begin{document}
	\author{VR443470}
	\title{Soluzioni Esami di Algebra Lineare}
	\date{\printdayoff\today}
	\maketitle
	
	\newpage
	
	% indice
	\tableofcontents
	
	\newpage
	
	\section{Esame del 20/06/2022}
	
	\subsection{Esercizio 1}
	
	(\textbf{6 punti}) Si consideri la seguente matrice:
	\begin{equation*}
		A = \begin{pmatrix}
			1	& 0		& 0		& 2 \\
			-2	& -1	& 1		& -5 \\
			1	& -a	& 2+a	& 2-a \\
			1+a	& 0		& 2		& \left(1+a\right)\left(-1+a\right)
		\end{pmatrix}
	\end{equation*}
	
	\longline
	
	\subsubsection{Punto a}
	
	\textcolor{Green4}{\emph{\textbf{Si calcoli, al variare di $a \in \mathbb{R}$, il rango $\mathrm{rk} \: A$ di $A$.}}}\newline
	
	\noindent
	Si applica l'eliminazione di Gauss per ottenere il rango della matrice:
	\begin{gather*}
		\begin{pmatrix}
			1	& 0		& 0		& 2 \\
			-2	& -1	& 1		& -5 \\
			1	& -a	& 2+a	& 2-a \\
			1+a	& 0		& 2		& \left(1+a\right)\left(-1+a\right)
		\end{pmatrix}
		\xrightarrow{E_{1,2}\left(2\right)}
		\begin{pmatrix}
			1	& 0		& 0		& 2 \\
			0	& -1	& 1		& -1 \\
			1	& -a	& 2+a	& 2-a \\
			1+a	& 0		& 2		& \left(1+a\right)\left(-1+a\right)
		\end{pmatrix} \\
		\\
		\xrightarrow{E_{1,3}\left(-1\right)}
		\begin{pmatrix}
			1	& 0		& 0		& 2 \\
			0	& -1	& 1		& -1 \\
			0	& -a	& 2+a	& -a \\
			1+a	& 0		& 2		& \left(1+a\right)\left(-1+a\right)
		\end{pmatrix}
		\xrightarrow{E_{1,4}\left(-a-1\right)}
		\begin{pmatrix}
			1	& 0		& 0		& 2 \\
			0	& -1	& 1		& -1 \\
			0	& -a	& 2+a	& -a \\
			0	& 0		& 2		& a^{2} -2a -3
		\end{pmatrix} \\
		\\
		\xrightarrow{E_{2,3}\left(-a\right)}
		\begin{pmatrix}
			1	& 0		& 0		& 2 \\
			0	& -1	& 1		& -1 \\
			0	& 0		& 2		& 0 \\
			0	& 0		& 2		& a^{2} -2a -3
		\end{pmatrix}
		\xrightarrow{E_{3,4}\left(-1\right)}
		\begin{pmatrix}
			1	& 0		& 0		& 2 \\
			0	& -1	& 1		& -1 \\
			0	& 0		& 2		& 0 \\
			0	& 0		& 0		& a^{2} -2a -3
		\end{pmatrix}
	\end{gather*}
	Il rango della matrice è influenzato solo dall'espressione $a^{2} - 2a - 3$. Quindi, nel caso di:
	\begin{equation*}
		\mathrm{rk}\left(A\right) =
		\begin{cases}
			3	& a^{2} - 2a -3 = 0\\
			4	& a^{2} - 2a -3 \ne 0
		\end{cases}
	\end{equation*}\newpage
	
	\subsubsection{Punto b}
	
	\textcolor{Green4}{\emph{\textbf{Si calcoli il determinante $\det\left(A\right)$ di $A$.}}}\newline
	
	\noindent
	Per velocizzare i calcoli, si utilizza il metodo di Gauss Jordan\footnote{Approfondimento: \href{https://www.youmath.it/forum/algebra-lineare/52861-determinante-con-eliminazione-di-gauss.html}{YouMath}}. Per il calcolo del determinante, si ricordano le seguenti regole:
	\begin{itemize}
		\item Lo scambio di una riga cambia il segno del determinante (quindi lo moltiplica per $-1$);
		
		\item La moltiplicazione di una riga per uno scalare non nullo, provoca la moltiplicazione dell'inverso di esso al determinante della matrice. Quindi, data l'operazione $E_{i}\left(\alpha\right)$, il determinante viene moltiplicato per $\frac{1}{\alpha}$;
		
		\item La moltiplicazione di una riga per uno scalare e la successiva somma, non cambia il determinante.
	\end{itemize}
	Quindi, si ottiene il determinante della matrice ridotta $A'$ moltiplicando la diagonale:
	\begin{gather*}
		A' = \begin{pmatrix}
			1	& 0		& 0		& 2 \\
			0	& -1	& 1		& -1 \\
			0	& 0		& 2		& 0 \\
			0	& 0		& 0		& a^{2} -a -2
		\end{pmatrix} \\
		\\
		\det\left(A'\right) = 1 \cdot \left(-1\right) \cdot 2 \cdot \left(a^{2}-2a-3\right) = -2a^{2} + 4a +6
	\end{gather*}
	E controllando le operazioni eseguite al punto precedente, è possibile notare che non è stato effettuato nessuno scambio di righe e nessuna moltplicazione + somma. Quindi il determinante è lo stesso:
	\begin{equation*}
		\det\left(A\right) = \det\left(A'\right) = -2a^{2} + 4a +6
	\end{equation*}\newpage
	
	\subsubsection{Punto c}
	
	\textcolor{Green4}{\textbf{\emph{Si determino i valori di $a \in \mathbb{R}$ tali che $A$ possiede una inversa.}}}\newline
	
	\noindent
	La matrice $A$ possiede un'inversa se e solo se il suo determinante è diverso da zero. Quindi, risolvendo l'equazione del determinante, si può capire per quali valori di $a$, la matrice $A$ ammette inversa:
	\begin{equation*}
		-2a^{2} + 4a + 6 = 0\longrightarrow \dfrac{-b \pm \sqrt{b^{2} - 4ac}}{2a} \longrightarrow
		\dfrac{-4 \pm \sqrt{16 - 4 \cdot -2 \cdot 6}}{2 \cdot -2} = \dfrac{-4 \pm 8}{-4}
	\end{equation*}
	Le soluzioni che azzerano l'equazione sono:
	\begin{gather*}
		a_{0} = -1 \\
		a_{1} = 3
	\end{gather*}
	Si conclude dicendo che:
	\begin{equation*}
		\det\left(A\right) = \begin{cases}
			0										& a = -1 \lor a = 3 \\
			\mathbb{R} \setminus \left\{0\right\}	& \text{altrimenti}
		\end{cases}
	\end{equation*}
	La matrice $A$ possiede una inversa se e solo se il determinante è diverso da zero. Il determinante è diverso da zero se e solo se $a$ è diverso da $-1$ e da $3$. Quindi, $A$ possiede una inversa per i valori $a \in \mathbb{R} \setminus \left\{-1, 3\right\}$.\newpage

	\subsection{Esercizio 2}
	
	(\textbf{12 punti}) Si consideri la seguente matrice:
	\begin{equation*}
		B = \begin{pmatrix}
			-1	& 0		& 0		& 0 \\
			0	& 2		& -3	& 0 \\
			0	& 0		& -1	& 0 \\
			0	& 0		& 5		& 2
		\end{pmatrix}
	\end{equation*}
	
	\longline
	
	\subsubsection{Punto a}

	\textcolor{Green4}{\textbf{\emph{Si calcolino tutti gli autovalori di $B$ su $\mathbb{R}$ e si trovino delle basi dei loro autospazi.}}}\newline

	\noindent
	Si calcola il polinomio caratteristico associato alla matrice $B$:
	\begin{equation*}
		\begin{array}{lll}
			p_{B}\left(\lambda\right) &=& \det\left(B - \lambda \mathrm{Id}_{4}\right) \\ [.7em]
			&=& \det\left[
				\begin{pmatrix}
					-1	& 0		& 0		& 0 \\
					0	& 2		& -3	& 0 \\
					0	& 0		& -1	& 0 \\
					0	& 0		& 5		& 2
				\end{pmatrix} -
				\lambda \begin{pmatrix}
					1	& 0		& 0		& 0 \\
					0	& 1		& 0		& 0 \\
					0	& 0		& 1		& 0 \\
					0	& 0		& 0		& 1
				\end{pmatrix}
			\right] \\ [2.5em]
			&=& \det\left[
				\begin{pmatrix}
					-1	& 0		& 0		& 0 \\
					0	& 2		& -3	& 0 \\
					0	& 0		& -1	& 0 \\
					0	& 0		& 5		& 2
				\end{pmatrix} -
				\begin{pmatrix}
					\lambda	& 0			& 0			& 0 \\
					0		& \lambda	& 0			& 0 \\
					0		& 0			& \lambda	& 0 \\
					0		& 0			& 0			& \lambda
				\end{pmatrix}
			\right] \\ [2.5em]
			&=& \det \begin{pmatrix}
				-1-\lambda	& 0			& 0				& 0 \\
					0		& 2-\lambda	& -3			& 0 \\
					0		& 0			& -1-\lambda	& 0 \\
					0		& 0			& 5				& 2-\lambda
			\end{pmatrix} \\ [2.5em]
		\end{array}
	\end{equation*}
	Si utilizzano gli sviluppi di Laplace per calcolare il determinante della matrice. Si sceglie lo sviluppo per righe partendo dalla riga $4$ e rimando sulla colonna $4$:
	\begin{gather*}
		\begin{rowequmat}{cccc}
				-1-\lambda	& 0			& 0				& 0 \\
					0		& 2-\lambda	& -3			& 0 \\
					0		& 0			& -1-\lambda	& 0 \\
					0		& 0			& 5				& 2-\lambda
		\end{rowequmat} \\
		\\
		\begin{array}{lll}
			b_{4,4} = 2-\lambda &\longrightarrow& \left(-1\right)^{4+4} \cdot \left(2-\lambda\right) \cdot \det
			\begin{rowequmat}{ccc}
				-1-\lambda	& 0			& 0			 \\
					0		& 2-\lambda	& -3		 \\
					0		& 0			& -1-\lambda 
			\end{rowequmat} \\[2.5em]
			&\longrightarrow& 1 \cdot \left(2-\lambda\right) \cdot \left(-1-\lambda\right) \cdot \left(2-\lambda\right) \cdot \left(-1-\lambda\right) \\ [1em]
			b_{3,4} = 0 		&\longrightarrow& \left(-1\right)^{3+4} \cdot \left(0\right) \cdot \det
			\begin{rowequmat}{ccc}
				-1-\lambda	& 0			& 0			 \\
					0		& 2-\lambda	& -3		 \\
					0		& 0			& 5 		 
			\end{rowequmat} = 0 \\[2.5em]
			b_{2,4} = 0 		&\longrightarrow& \left(-1\right)^{2+4} \cdot \left(0\right) \cdot \det
			\begin{rowequmat}{ccc}
				-1-\lambda	& 0			& 0				\\
					0		& 0			& -1-\lambda	\\
					0		& 0			& 5				
			\end{rowequmat} = 0 \\[2.5em]
			b_{1,4} = 0 		&\longrightarrow& \left(-1\right)^{1+4} \cdot \left(0\right) \cdot \det
			\begin{rowequmat}{ccc}
					0		& 2-\lambda	& -3			\\
					0		& 0			& -1-\lambda	\\
					0		& 0			& 5				
			\end{rowequmat} = 0 \\[2.5em]
		\end{array}
	\end{gather*}
	Quindi, il determinante della matrice è:
	\begin{equation*}
		p_{B}\left(\lambda\right) = \det\left(B - \lambda \mathrm{Id}_{4}\right) = \left(-1-\lambda\right) \cdot \left(2-\lambda\right) \cdot \left(-1-\lambda\right) \cdot \left(2-\lambda\right)
	\end{equation*}
	Per continuare il calcolo del polinomio caratteristico, si cercano gli zeri, ovvero tutti quei valori tale che $p_{B}\left(\lambda\right) = 0$. Banalmente, i valori sono:
	\begin{equation*}
		\begin{array}{lll}
			\lambda_{1} = -1 &\longrightarrow& \left(-1-\left(-1\right)\right) \cdot \left(2-\left(-1\right)\right) \cdot \left(-1-\left(-1\right)\right) \cdot \left(2-\left(-1\right)\right) = 0 \\[.5em]
			\lambda_{2} = 2  &\longrightarrow& \left(-1-\left(2\right)\right) \cdot \left(2-\left(2\right)\right) \cdot \left(-1-\left(2\right)\right) \cdot \left(2-\left(2\right)\right) = 0
		\end{array}
	\end{equation*}
	Si conclude affermando che gli autovalori di $B$ sono $\lambda_{1} = -1$ e $\lambda_{2} = 2$.\newpage

	\noindent
	Per trovare delle basi dei loro autospazi, è necessario sostituire ogni $\lambda$ trovato, nella matrice calcolata precedentemente. Quindi, il primo autospazio con il primo autovalore $\lambda_{1} = -1$:
	\begin{equation*}
		B - \lambda_{1} \mathrm{Id}_{4} = B - \left(-1\right) \mathrm{Id}_{4} =
		\begin{rowequmat}{cccc}
				0	& 0		& 0		& 0 \\
				0	& 3		& -3	& 0 \\
				0	& 0		& 0		& 0 \\
				0	& 0		& 5		& 3
		\end{rowequmat}
	\end{equation*}
	Si cerca la forma ridotta della matrice, eseguendo l'eliminazione di Gauss:
	\begin{equation*}
		\begin{rowequmat}{cccc}
			0	& 0		& 0		& 0 \\
			0	& 3		& -3	& 0 \\
			0	& 0		& 0		& 0 \\
			0	& 0		& 5		& 3
		\end{rowequmat} \xlongrightarrow[E_{4,2}]{E_{2,1}}
		\begin{rowequmat}{cccc}
			0	& 3		& -3	& 0 \\
			0	& 0		& 5		& 3 \\
			0	& 0		& 0		& 0 \\
			0	& 0		& 0		& 0 
		\end{rowequmat}
	\end{equation*}
	Si ottiene il sistema lineare:
	\begin{equation*}
		\begin{cases}
			3x_{2} -3x_{3} = 0 \\
			5x_{3} +3x_{4} = 0	
		\end{cases}
		\longrightarrow
		\begin{cases}
			x_{3} = x_{2} \\
			x_{4} = -\frac{5}{3}x_{2}
		\end{cases}
	\end{equation*}
	Quindi, il sistema generale è:
	\begin{equation*}
		\begin{cases}
			x_{1} = x_{1} \\
			x_{2} = x_{2} \\
			x_{3} = x_{2} \\
			x_{4} = -\frac{5}{3}x_{2}
		\end{cases}
	\end{equation*}
	L'autospazio generale è:
	\begin{equation*}
		\left\{
			\left. \begin{pmatrix}
				x_{1} \\
				x_{2} \\
				x_{2} \\
				-\frac{5}{3}x_{2}
			\end{pmatrix} \: \right| \:
			x_{1}, x_{2} \in \mathbb{R}
		\right\} = \left\{x_{1}
		\begin{pmatrix}
			1 \\
			0 \\
			0 \\
			0
		\end{pmatrix} + x_{2}
		\begin{rowequmat}{c}
			0 \\ [.3em]
			1 \\ [.3em]
			1 \\ [.3em]
			-\frac{5}{3}
		\end{rowequmat}
		\right\}
	\end{equation*}
	Dove il seguente insieme è una base:
	\begin{equation*}
		\left\{
			\begin{pmatrix}
				1 \\
				0 \\
				0 \\
				0
			\end{pmatrix} ,
			\begin{rowequmat}{c}
				0 \\ [.3em]
				1 \\ [.3em]
				1 \\ [.3em]
				-\frac{5}{3}
			\end{rowequmat}
		\right\}
	\end{equation*}\newpage

	\noindent
	Il secondo autospazio con il secondo autovalore $\lambda = 2$:
	\begin{equation*}
		B - \lambda_{2} \mathrm{Id}_{4} = B - \left(2\right) \mathrm{Id}_{4} =
		\begin{rowequmat}{cccc}
				-3	& 0		& 0		& 0 \\
				0	& 0		& -3	& 0 \\
				0	& 0		& -3	& 0 \\
				0	& 0		& 5		& 0
		\end{rowequmat}
	\end{equation*}
	Si cerca la forma ridotta della matrice, eseguendo l'eliminazione di Gauss:
	\begin{equation*}
		\begin{rowequmat}{cccc}
			-3	& 0		& 0		& 0 \\
			0	& 0		& -3	& 0 \\
			0	& 0		& -3	& 0 \\
			0	& 0		& 5		& 0
		\end{rowequmat}
		\xlongrightarrow[E_{4,2}\left(\frac{3}{5}\right)]{E_{2,3}\left(-1\right)}
		\begin{rowequmat}{cccc}
			-3	& 0		& 0		& 0 \\
			0	& 0		& 0		& 0 \\
			0	& 0		& 0		& 0 \\
			0	& 0		& 5		& 0
		\end{rowequmat}
		\xlongrightarrow[E_{1}\left(-\frac{1}{3}\right)]{E_{4,2}, \: E_{2}\left(\frac{1}{5}\right)}
		\begin{rowequmat}{cccc}
			1	& 0		& 0		& 0 \\
			0	& 0		& 1		& 0 \\
			0	& 0		& 0		& 0 \\
			0	& 0		& 0		& 0
		\end{rowequmat}
	\end{equation*}
	Si ottiene il sistema lineare:
	\begin{equation*}
		\begin{cases}
			x_{1} = 0 \\
			x_{3} = 0
		\end{cases}
	\end{equation*}
	Quindi, il sistema generale è:
	\begin{equation*}
		\begin{cases}
			x_{1} = 0		\\
			x_{2} = x_{2} 	\\
			x_{3} = 0		\\
			x_{4} = x_{4}
		\end{cases}
	\end{equation*}
	L'autospazio generale è:
	\begin{equation*}
		\left\{
			\left. \begin{pmatrix}
				0		\\
				x_{2} 	\\
				0		\\
				x_{4}
			\end{pmatrix} \: \right| \:
			x_{2}, x_{4} \in \mathbb{R}
		\right\} =
		\left\{
			x_{2} \begin{pmatrix}
				0 \\
				1 \\
				0 \\
				0
			\end{pmatrix}
			+
			x_{4} \begin{pmatrix}
				0 \\
				0 \\
				0 \\
				1
			\end{pmatrix}
		\right\}
	\end{equation*}
	Dove il seguente insieme è una base:
	\begin{equation*}
		\left\{
			\begin{pmatrix}
				0 \\
				1 \\
				0 \\
				0
			\end{pmatrix},
			\begin{pmatrix}
				0 \\
				0 \\
				0 \\
				1
			\end{pmatrix}
		\right\}
	\end{equation*}\newpage

	\subsubsection{Punto b}

	\textcolor{Green4}{\textbf{\emph{Si verifichi che la matrice $B$ è diagonalizzabile e si scrivano la matrice diagonale $D$ e la matrice invertibile $S$ tali che $B = SDS^{-1}$.}}}\newline

	\noindent
	Una matrice $B$ è diagonalizzabile se rispetta due condizioni:
	\begin{enumerate}
		\item La somma delle molteplicità algebriche degli autovalori della matrice è uguale all'ordine della matrice;
		\item La molteplicità geometrica di ciascun autovalore coincide con la relativa molteplicità algebrica.
	\end{enumerate}
	Gli autovalori di $B$ sono $\lambda_{1}=-1$ e $\lambda_{2} = 2$. Ma dato che le soluzioni derivano da:
	\begin{equation*}
		p_{B}\left(\lambda\right) = \det\left(B - \lambda \mathrm{Id}_{4}\right) = \left(-1-\lambda\right) \cdot \left(2-\lambda\right) \cdot \left(-1-\lambda\right) \cdot \left(2-\lambda\right)
	\end{equation*}
	È immediato vedere come ogni soluzione trovata annulli due volte il polinomio caratteristico. Quindi le molteplicità algebriche sono:
	\begin{gather*}
		m_{1} = 2 \\
		m_{2} = 2
	\end{gather*}
	La prima condizione è soddisfatta. La seconda è la verifica della molteplicità geometrica. Quest'ultima è possibile verificarla con la seguente formula:
	\begin{equation*}
		m_{g}\left(\lambda\right) = n - \mathrm{rk}\left(A - \lambda\mathrm{Id}_{n}\right)
	\end{equation*}
	Dunque, si calcola il rango di tutte le matrici trovate sostituendo gli autovalori ottenuti:
	\begin{equation*}
		\begin{array}{lll}
			m_{1}\left(-1\right) &=& 4 - \mathrm{rk}\left(A - \left(-1\right)\mathrm{Id}_{4}\right) = 4 - 2 = 2 \\
			m_{2}\left( 2\right) &=& 4 - \mathrm{rk}\left(A - 2\mathrm{Id}_{4}\right) = 4 - 2 = 2
		\end{array}
	\end{equation*}
	Ciascuna molteplicità geometrica corrisponde con la relativa molteplicità algebrica. Questo conferma che la matrice $B$ è diagonalizzabile.\newline

	\noindent
	La matrice diagonale $D$ ha gli autovalori di $B$ nella diagonale principale:
	\begin{equation*}
		D = \begin{pmatrix}
			-1 & 0 & 0 & 0 \\
			0 & -1 & 0 & 0 \\
			0 & 0 & 2 & 0 \\
			0 & 0 & 0 & 2 
		\end{pmatrix}
	\end{equation*}
	La matrice $S$ invece è composta dalle basi trovate, ovvero:
	\begin{equation*}
		S = \begin{rowequmat}{cccc}
			1 & 0				& 0 & 0 \\ [.3em]
			0 & 1				& 1 & 0 \\ [.3em]
			0 & 1				& 0 & 0 \\ [.3em]
			0 & -\frac{5}{3}	& 0 & 1
		\end{rowequmat}
	\end{equation*}\newpage
	
	\noindent
	La matrice inversa si trova affiancando a destra la matrice identità ed eseguendo EG:
	\begin{gather*}
		\left(S \: | \: \mathrm{Id_{4}}\right) = 
		\begin{rowequmat}{cccc|cccc}
			1 & 0				& 0 & 0 & 1 & 0 & 0 & 0 \\ [.3em]
			0 & 1				& 1 & 0 & 0 & 1 & 0 & 0 \\ [.3em]
			0 & 1				& 0 & 0 & 0 & 0 & 1 & 0 \\ [.3em]
			0 & -\frac{5}{3}	& 0 & 1 & 0 & 0 & 0 & 1
		\end{rowequmat}
		\xlongrightarrow[E_{2,3}\left(-1\right)]{E_{3,2}}
		\begin{rowequmat}{cccc|cccc}
			1 & 0				& 0 & 0 & 1 & 0 & 0 & 0 \\ [.3em]
			0 & 1				& 0 & 0 & 0 & 0 & 1 & 0 \\ [.3em]
			0 & 0				& 1 & 0 & 0 & 1 & -1 & 0 \\ [.3em]
			0 & -\frac{5}{3}	& 0 & 1 & 0 & 0 & 0 & 1
		\end{rowequmat} \\
		\\
		\xlongrightarrow{E_{2,4}\left(\frac{5}{3}\right)}
		\begin{rowequmat}{cccc|cccc}
			1 & 0	& 0 & 0 & 1 & 0 & 0 & 0 \\ [.3em]
			0 & 1	& 0 & 0 & 0 & 0 & 1 & 0 \\ [.3em]
			0 & 0	& 1 & 0 & 0 & 1 & -1 & 0 \\ [.3em]
			0 & 0	& 0 & 1 & 0 & 0 & \frac{5}{3} & 1
		\end{rowequmat}\\
		\\
		S^{-1} = 
		\begin{rowequmat}{cccc}
			1 & 0 & 0 & 0 \\ [.3em]
			0 & 0 & 1 & 0 \\ [.3em]
			0 & 1 & -1 & 0 \\ [.3em]
			0 & 0 & \frac{5}{3} & 1
		\end{rowequmat}
	\end{gather*}
	Si verifica la correttezza:
	\begin{equation*}
		\begin{array}{lll}
			B = SDS^{-1} &=&
			\begin{rowequmat}{cccc}
				1 & 0				& 0 & 0 \\ [.3em]
				0 & 1				& 1 & 0 \\ [.3em]
				0 & 1				& 0 & 0 \\ [.3em]
				0 & -\frac{5}{3}	& 0 & 1
			\end{rowequmat}
			\begin{rowequmat}{cccc}
				-1 & 0 & 0 & 0 \\ [.3em]
				0 & -1 & 0 & 0 \\ [.3em]
				0 & 0 & 2 & 0 \\ [.3em]
				0 & 0 & 0 & 2 
			\end{rowequmat}
			\begin{rowequmat}{cccc}
				1 & 0 & 0 & 0 \\ [.3em]
				0 & 0 & 1 & 0 \\ [.3em]
				0 & 1 & -1 & 0 \\ [.3em]
				0 & 0 & \frac{5}{3} & 1
			\end{rowequmat} \\[3em]
			%
			&=&
			\begin{rowequmat}{cccc}
				1 & 0				& 0 & 0 \\ [.3em]
				0 & 1				& 1 & 0 \\ [.3em]
				0 & 1				& 0 & 0 \\ [.3em]
				0 & -\frac{5}{3}	& 0 & 1
			\end{rowequmat}
			\begin{rowequmat}{cccc}
				-1 & 0 & 0 & 0 \\ [.3em]
				0 & 0 & -1 & 0 \\ [.3em]
				0 & 2 & -2 & 0 \\ [.3em]
				0 & 0 & \frac{10}{3} & 2
			\end{rowequmat} \\[3em]
			%
			&=&
			\begin{rowequmat}{cccc}
				-1 & 0 & 0 & 0 \\ [.3em]
				0 & 2 & -3 & 0 \\ [.3em]
				0 & 0 & -1 & 0 \\ [.3em]
				0 & 0 & 5 & 2
			\end{rowequmat}
		\end{array}
	\end{equation*}\newpage

	\subsubsection{Punto c}

	\textcolor{Green4}{\textbf{\emph{Utilizzando la diagonalizzazione, si calcoli il prodotto $B^{5}$.}}}\newline

	\noindent
	Si calcola:
	\begin{equation*}
		B^{5} = \left(SDS^{-1}\right)^{5} = S D^{5} S^{-1}
	\end{equation*}
	Quindi, le matrici mutano in:
	\begin{equation*}
		\begin{array}{lll}
			B = \left(SDS^{-1}\right)^{5} &=&
			\begin{rowequmat}{cccc}
				1 & 0				& 0 & 0 \\ [.3em]
				0 & 1				& 1 & 0 \\ [.3em]
				0 & 1				& 0 & 0 \\ [.3em]
				0 & -\frac{5}{3}	& 0 & 1
			\end{rowequmat}
			\begin{rowequmat}{cccc}
				-1 & 0 & 0 & 0 \\ [.3em]
				0 & -1 & 0 & 0 \\ [.3em]
				0 & 0 & 32 & 0 \\ [.3em]
				0 & 0 & 0 & 32 
			\end{rowequmat}
			\begin{rowequmat}{cccc}
				1 & 0 & 0 & 0 \\ [.3em]
				0 & 0 & 1 & 0 \\ [.3em]
				0 & 1 & -1 & 0 \\ [.3em]
				0 & 0 & \frac{5}{3} & 1
			\end{rowequmat} \\[3em]
			%
			&=&
			\begin{rowequmat}{cccc}
				1 & 0				& 0 & 0 \\ [.3em]
				0 & 1				& 1 & 0 \\ [.3em]
				0 & 1				& 0 & 0 \\ [.3em]
				0 & -\frac{5}{3}	& 0 & 1
			\end{rowequmat}
			\begin{rowequmat}{cccc}
				-1 & 0 & 0 & 0 \\ [.3em]
				0 & 0 & -1 & 0 \\ [.3em]
				0 & 32 & -32 & 0 \\ [.3em]
				0 & 0 & \frac{160}{3} & 32
			\end{rowequmat} \\[3em]
			%
			&=&
			\begin{rowequmat}{cccc}
				-1 & 0 & 0 & 0 \\ [.3em]
				0 & 32 & -33 & 0 \\ [.3em]
				0 & 0 & -1 & 0 \\ [.3em]
				0 & 0 & 55 & 32
			\end{rowequmat}
		\end{array}
	\end{equation*}\newpage

	\subsection{Esercizio 3}

	(\textbf{8 punti}) Si consideri la seguente matrice:
	\begin{equation*}
		M = \begin{pmatrix}
			i	& 2i+1	\\
			-1	& 2		\\
			-i	& 2i
		\end{pmatrix}
	\end{equation*}
	
	\longline

	\subsubsection{Punto a}

	\textcolor{Green4}{\textbf{\emph{Si calcoli la $H$-trasposta $M^{H}$ di $M$.}}}\newline

	\noindent
	La matrice $H$-trasposta di $M$, si ottiene invertendo le righe con le colonne ed eseguendo l'operazione di coniugazione (cambiare di segno), la quale influisce solo sui numeri immaginari:
	\begin{equation*}
		\begin{array}{lll}
			M &=& \begin{pmatrix}
				i	& 2i+1	\\
				-1	& 2		\\
				-i	& 2i
			\end{pmatrix} \\ [2	em]
			M^{T} &=& \begin{pmatrix}
				i 		& -1	& -i \\
				2i+1	& 2		& 2i
			\end{pmatrix} \\ [2em]
			M^{H} &=& \begin{pmatrix}
				-i 		& -1	& i \\
				-2i+1	& 2		& -2i
			\end{pmatrix}
		\end{array}
	\end{equation*}

	\longline

	\subsubsection{Punto b}

	\textcolor{Green4}{\textbf{\emph{Si determinino una base di $C\left(M\right)$ e una base di $N\left(M^{H}\right)$ su $\mathbb{C}$.}}}\newline

	\noindent
	Si applica l'eliminazione di Gauss per ottenere una forma ridotta:
	\begin{equation*}
		M' =
		\begin{pmatrix}
			i	& 2i+1	\\
			-1	& 2		\\
			-i	& 2i
		\end{pmatrix}
		\xlongrightarrow[E_{1,2}\left(\frac{1}{i}\right)]{E_{1,3}\left(1\right)}
		\begin{rowequmat}{cc}
			i	& 2i+1 \\ [.3em]
			0	& 4+\frac{1}{i} \\ [.3em]
			0	& 4i+1
		\end{rowequmat}
		\xlongrightarrow{E_{2,3}\left(-i\right)}
		\begin{rowequmat}{cc}
			i	& 2i+1 \\ [.3em]
			0	& 4+\frac{1}{i} \\ [.3em]
			0	& 0
		\end{rowequmat}
	\end{equation*}
	Il numero di \emph{pivot}, in questo caso 2, rappresenta il rango della matrice ($\mathrm{rk}\left(M\right) = 2$), ovvero il numero di vettori colonna linearmente indipendenti. In altre parole, rappresenta la dimensione dello spazio generato dai vettori considerati inizialmente.

	Quindi, è possibile affermare che i vettori colonna della matrice non ridotta $M$, i quali corrispondono ai vettori colonna della matrice ridotta $M'$ (soprastante) che contengono i pivot, costituiscono una base dello spazio generato del sistema di generatori. Per cui, una base di $C\left(M\right)$:
	\begin{equation*}
		\dim\left(M\right) = \mathrm{rk}\left(M\right) \longrightarrow 2 = 2 \Longrightarrow
		C\left(M\right) = \left\{
			\begin{pmatrix}
				i  \\
				-1 \\
				-i
			\end{pmatrix},
			\begin{pmatrix}
				2i+1 \\
				2 	 \\
				2i
			\end{pmatrix}
		\right\}
	\end{equation*}\newpage

	\noindent
	Si prosegue l'esercizio calcolando una base della nullità di $M^{H}$. Quindi, si esegue l'eliminazione di Gauss:
	\begin{equation*}
		\begin{pmatrix}
			-i 		& -1	& i \\
			-2i+1	& 2		& -2i
		\end{pmatrix}
		\xlongrightarrow{E_{1,2}\left(-2-i\right)}
		\begin{pmatrix}
			-i 	& -1	& i \\
			0	& 4+i	& 1-4i
		\end{pmatrix}
	\end{equation*}
	Si ottiene il seguente sistema lineare:
	\begin{gather*}
		\begin{cases}
			-ix -y +iz = 0 \\
			\left(4+i\right)y + \left(1-4i\right) z = 0
		\end{cases}
		\longrightarrow
		\begin{cases}
			-ix -y +iz = 0 \\
			\\
			y = -\dfrac{\left(1-4i\right)}{\left(4+i\right)} z
		\end{cases}\\
		\\
		\dfrac{\left(1-4i\right)}{\left(4+i\right)} = \dfrac{\left(1-4i\right)\left(4-i\right)}{\left(4+i\right)\left(4-i\right)} = \dfrac{4-i-16i+4i^{2}}{16-4i+4i-i^2} = \dfrac{-17i}{17} = -i \\
		\\
		\begin{cases}
			-ix -y +iz = 0 \\
			y = -\left(-i\right) z
		\end{cases}
		\longrightarrow
		\begin{cases}
			-ix -iz +iz = 0 \\
			y = i z
		\end{cases}
		\longrightarrow
		\begin{cases}
			x = 0 \\
			y = i z
		\end{cases}\\
	\end{gather*}
	Dopo alcune semplificazioni, il risultato generale è:
	\begin{equation*}
		\begin{cases}
			x = 0 \\
			y = iz \\
			z = z
		\end{cases}
	\end{equation*}
	Dunque $N\left(M^{H}\right)$ è:
	\begin{equation*}
		N\left(M^{H}\right) = \left\{
			\left.
				z
				\begin{pmatrix}
					0 \\
					i \\
					1
				\end{pmatrix} \:
			\right| \: z \in \mathbb{C}
		\right\}
	\end{equation*}
	Per cui, si ha una base:
	\begin{equation*}
		\left\{
			\begin{pmatrix}
				0 \\
				i \\
				1
			\end{pmatrix}
		\right\}
	\end{equation*}

	\longline

	\subsubsection{Punto c}

	\textcolor{Green4}{\textbf{\emph{Si scriva una base di $\mathbb{C}^{3}$ che contiene le colonne di $M$.}}}\newline

	\noindent
	Dato che per definizione:
	\begin{equation*}
		\mathbb{C}^{3} = C\left(M\right) + N\left(M^{H}\right)
	\end{equation*}
	Allora una base di $\mathbb{C}^{3}$ è:
	\begin{equation*}
		\left\{
			\begin{pmatrix}
				i  \\
				-1 \\
				-i
			\end{pmatrix},
			\begin{pmatrix}
				2i+1 \\
				2 	 \\
				2i
			\end{pmatrix},
			\begin{pmatrix}
				0 \\
				i \\
				1
			\end{pmatrix}
		\right\}
	\end{equation*}\newpage

	\subsection{Esercizio 4}

	(\textbf{4 punti}) Vero o falso? Si motivi la risposta!

	\longline

	\subsubsection{Punto a}

	\textcolor{Green4}{\textbf{\emph{Il sistema omogeneo $Ax = 0$ ammette soltanto la soluzione banale}} $x = \begin{pmatrix}
		0 \\ 0 \\ 0
	\end{pmatrix}$ \textbf{\emph{dove}} $A = \begin{pmatrix}
		1 & 0 & 0 \\ 0 & 1 & 0
	\end{pmatrix}$}\newline

	\noindent
	Il sistema omogeneo $Ax = 0$ non ammette soltanto la soluzione banale. Per dimostrare ciò, si utilizza il teorema di Rouché-Capelli. Quindi, si calcola il rango delle matrici ridotte e aumentate:
	\begin{equation*}
		\begin{array}{rllll}
			A &=& \begin{pmatrix}
				1 & 0 & 0 \\ 0 & 1 & 0
			\end{pmatrix} &\longrightarrow& \mathrm{rk}\left(A\right) = 2 \\ [2em]
			\left(A \: | \: \mathbf{0}\right) &=& \begin{rowequmat}{ccc|c}
				1 & 0 & 0 & 0 \\
				0 & 1 & 0 & 0
			\end{rowequmat} &\longrightarrow& \mathrm{rk}\left(A \: | \: \mathbf{0}\right) = 2
		\end{array}
	\end{equation*}
	Il rango di entrambe le matrici è uguale a due, quindi, secondo il teorema, esistono una o infinite soluzioni. Precisamente:
	\begin{equation*}
		\mathrm{rk}\left(A\right) = \mathrm{\left(A \: | \: \mathbf{0}\right)} < n
	\end{equation*}
	Dove $n$ indica il numero di incognite, in questo caso 3 ($x,y,z$). Andando a sostituire i valori:
	\begin{equation*}
		2 = 2 < 3
	\end{equation*}
	La condizione è rispettata, quindi secondo il teorema di Rouché-Capelli, il sistema $Ax = 0$ ammette infinite soluzioni, precisamente $\infty^{n - \mathrm{rk}\left(A\right)} \rightarrow \infty^{3-2} = \infty$.
	
	\noindent
	In generale, la forma generale della soluzione deve essere:
	\begin{equation*}
		A x = 0 \Longrightarrow
		\begin{pmatrix}
			1 & 0 & 0 \\ 0 & 1 & 0
		\end{pmatrix}
		x\begin{pmatrix}
			0 \\ 0 \\ 1
		\end{pmatrix} = \mathbf{0}
	\end{equation*}\newpage

	\subsubsection{Punto b}

	\textcolor{Green4}{\textbf{\emph{L'insieme }} $\left\{\begin{pmatrix}
		-1 \\ 0
	\end{pmatrix}, \begin{pmatrix}
		2 \\ 1
	\end{pmatrix}, \begin{pmatrix}
		1 \\ 2
	\end{pmatrix}\right\}$ \textbf{\emph{è linearmente dipendente.}}}\newline

	\noindent
	Per verificarlo, si prendono tre generici scalari $a,b,c \in \mathbb{R}$ e si moltiplicano per i vettori:
	\begin{equation*}
		av_{1} + bv_{2} + cv_{3} = \mathbf{0}
	\end{equation*}
	Sostituendo i valori:
	\begin{equation*}
		a
		\begin{pmatrix}
			-1 \\ 0
		\end{pmatrix}
		+ b
		\begin{pmatrix}
			2 \\ 1
		\end{pmatrix}
		+ c
		\begin{pmatrix}
			1 \\ 2
		\end{pmatrix}
		=
		\begin{pmatrix}
			0 \\ 0
		\end{pmatrix}
	\end{equation*}
	Eseguendo i calcoli:
	\begin{equation*}
		\begin{pmatrix}
			-a + 2b + c \\
			b + 2c
		\end{pmatrix} \Longrightarrow
		\begin{cases}
			-a + 2b + c = 0 \\
			b + 2c = 0
		\end{cases} \longrightarrow
		\begin{cases}
			-a + 2\left(-2c\right) + c = 0 \\
			b = -2c
		\end{cases} \longrightarrow
		\begin{cases}
			c = -\frac{a}{3} \\
			b = -2c
		\end{cases}
	\end{equation*}
	È evidente che i tre vettori sono linearmente dipendenti. $b$ dipende da $c$ e $c$ dipende da $a$.

	\longline

	\subsection{Esercizio 5}

	(\textbf{1 punto}) Sia $V$ uno spazio vettoriale su $\mathbb{K}$. Si dimostri la seguente affermazione: se almeno uno dei vettori $v_{1}, \cdots, v_{2}$ è combinazione lineare dei rimanenti, allora $\left\{v_{1}, \cdots, v_{2}\right\}$ non è linearmente indipendente.\newline

	\noindent
	Dimostrazione lasciata al lettore.\newpage

	\section{Esame del 15/07/2022}

	\subsection{Esercizio 1}

	(\textbf{6 punti}) Si consideri la seguente matrice:
	\begin{equation*}
		A = \begin{pmatrix}
			1 &	1	& 1 & 0 \\
			1 &	1	& k & k \\
			1 &	k-1	& 0 & 0 \\
			1 &	k-1	& k & k 
		\end{pmatrix}
	\end{equation*}
	
	\longline

	\subsubsection{Parte a}

	\textcolor{Green4}{\textbf{\emph{Si studi $\det\left(A\right)$ al variare di $k$.}}}\newline

	\noindent
	Si procede con l'eliminazione di Gauss:
	\begin{gather*}
		\begin{pmatrix}
			1 &	1	& 1 & 0 \\
			1 &	1	& k & k \\
			1 &	k-1	& 0 & 0 \\
			1 &	k-1	& k & k 
		\end{pmatrix}
		\xlongrightarrow{E_{2,1}\left(-1\right)}
		\begin{pmatrix}
			0 &	0	& 1-k & -k \\
			1 &	1	& k & k \\
			1 &	k-1	& 0 & 0 \\
			1 &	k-1	& k & k 
		\end{pmatrix} \\
		\\
		\xlongrightarrow[E_{4,3}\left(-1\right)]{E_{4,2}\left(-1\right)}
		\begin{pmatrix}
			0 &	0	& 1-k & -k \\
			0 &	-k+2 & 0 & 0 \\
			0 &	0	& -k & -k \\
			1 &	k-1	& k & k 
		\end{pmatrix}
		\xlongrightarrow[E_{4,3}\left(-1\right)]{E_{4,1}}
		\begin{pmatrix}
			1 &	k-1	& k & k \\
			0 &	-k+2 & 0 & 0 \\
			0 &	0	& -1 & 0 \\
			0 &	0	& 1-k & -k 
		\end{pmatrix}\\
		\\
		\xlongrightarrow[E_{3,4}\left(-k\right)]{E_{3,4}\left(1\right)}
		\begin{pmatrix}
			1 &	k-1	& k & k \\
			0 &	-k+2 & 0 & 0 \\
			0 &	0	& -1 & 0 \\
			0 &	0	&  0 & -k 
		\end{pmatrix}
		\xrightarrow{E_{3}\left(-1\right)}
		\begin{pmatrix}
			1 &	k-1	& k & k \\
			0 &	-k+2 & 0 & 0 \\
			0 &	0	& 1 & 0 \\
			0 &	0	&  0 & -k 
		\end{pmatrix}
	\end{gather*}
	Il determinante della matrice ridotta è:
	\begin{equation*}
		\det\left(A\right) = 1 \cdot \left(-k+2\right) \cdot 1 \cdot \left(-k\right) = k^{2} - 2k = k\left(k-2\right)
	\end{equation*}

	\longline

	\subsubsection{Parte b}

	\textcolor{Green4}{\textbf{\emph{Si studi $\mathrm{rk}\left(A\right)$ al variare di $k$.}}}\newline

	\noindent
	Il rango corrisponde al numero di \emph{pivot} della matrice ridotta, in questo caso, dipende da $k$, ovvero:
	\begin{equation*}
		\mathrm{rk}\left(A\right) = \begin{cases}
			3 & k = 0 \lor k = 2 \\
			4 & \text{altrimenti}
		\end{cases}
	\end{equation*}\newpage

	\subsubsection{Parte c}

	\textcolor{Green4}{\textbf{\emph{Si determini se $A$ è invertibile. Se sì, per quali valori di $k$?}}}\newline

	\noindent
	La matrice $A$ è invertibile se e solo se il determinante è diverso da zero. Dato che il determinante è $\det\left(A\right) = k\left(k-2\right)$, la matrice $A$ possiede inversa se e solo se $k \ne 0 \land k \ne 2$. In questo modo, il determinante non sarà mai nullo.\newpage

	\subsection{Esercizio 2}

	Si consideri la seguente matrice:
	\begin{equation*}
		B = \begin{rowequmat}{cc}
			1 & 0 \\
			4 & \frac{1}{2}
		\end{rowequmat}
	\end{equation*}

	\longline

	\subsubsection{Parte a}

	\textcolor{Green4}{\textbf{\emph{Si calcolino tutti gli autovalori di $B$ su $\mathbb{R}$ e si trovino delle basi dei loro autospazi.}}}\newline

	\noindent
	Si calcola il polinomio caratteristico:
	\begin{equation*}
		\begin{array}{lll}
			p_{B}\left(\lambda\right) &=& \det\left(B-\lambda \mathrm{Id}_{2}\right) \\ [2em]
			&=& \det\left[\begin{rowequmat}{cc}
				1 & 0 \\
				4 & \frac{1}{2}
			\end{rowequmat} - \lambda \begin{pmatrix}
				1 & 0 \\ 0 & 1
			\end{pmatrix}\right] \\ [2em]
			&=& \det\left[\begin{rowequmat}{cc}
				1 & 0 \\
				4 & \frac{1}{2}
			\end{rowequmat} - \begin{pmatrix}
				\lambda & 0 \\ 0 & \lambda
			\end{pmatrix}\right] \\ [2em]
			&=& \det\left[\begin{rowequmat}{cc}
				1-\lambda & 0 \\
				4 & \frac{1}{2}-\lambda
			\end{rowequmat}\right]
		\end{array}
	\end{equation*}
	Si esegue l'eliminazione di Gauss per calcolare il determinante della matrice:
	\begin{equation*}
		\begin{rowequmat}{cc}
			1-\lambda & 0 \\
			4 & \frac{1}{2}-\lambda
		\end{rowequmat}
		\xlongrightarrow{E_{1,2}\left(-\frac{4}{1-\lambda}\right)}
		\begin{rowequmat}{cc}
			1-\lambda & 0 \\
			0 & \frac{1}{2}-\lambda
		\end{rowequmat}
	\end{equation*}
	Il determinante dunque è:
	\begin{equation*}
		\det\left(B - \lambda\mathrm{Id}_{2}\right) = \left(1-\lambda\right) \cdot \left(\dfrac{1}{2}-\lambda\right)
	\end{equation*}
	Adesso si cercano gli zeri, ovvero quei valori per cui il polinomio caratteristico è uguale a zero:
	\begin{equation*}
		\begin{array}{lll}
			\lambda_{1} = 1				&\longrightarrow& \left(1-1\right) \cdot \left(\dfrac{1}{2}-1\right) \\ [2em]
			\lambda_{2} = \dfrac{1}{2}	&\longrightarrow& \left(1-\dfrac{1}{2}\right) \cdot \left(\dfrac{1}{2}-\dfrac{1}{2}\right)
		\end{array}
	\end{equation*}
	Adesso che sono stati trovati gli autovalori, si trovano le basi degli autospazi.\newpage

	\noindent
	Sostituzione del valore $\lambda_{1} = 1$:
	\begin{equation*}
		p_{B}\left(1\right) = \det\left(B - 1\mathrm{Id}_{2}\right) = 
		\begin{pmatrix}
			1-1 & 0 \\
			4 & \frac{1}{2}-1
		\end{pmatrix} = \begin{pmatrix}
			0 & 0 \\
			4 & -\frac{1}{2}
		\end{pmatrix}
	\end{equation*}
	L'eliminazione di Gauss non è necessaria, al massimo uno scambio di righe:
	\begin{equation*}
		\begin{pmatrix}
			0 & 0 \\
			4 & -\frac{1}{2}
		\end{pmatrix}
		\xlongrightarrow{E_{2,1}\left(1\right)}
		\begin{pmatrix}
			4 & -\frac{1}{2} \\
			0 & 0
		\end{pmatrix}
	\end{equation*}
	Si ottiene il sistema lineare:
	\begin{equation*}
		\begin{cases}
			4x_{1} - \dfrac{1}{2}x_{2} = 0
		\end{cases}
		\longrightarrow
		\begin{cases}
			x_{1} = \dfrac{1}{8}x_{2}
		\end{cases}
	\end{equation*}
	Quindi, il sistema generale:
	\begin{equation*}
		\begin{cases}
			x_{1} = \dfrac{1}{8}x_{2} \\
			\\
			x_{2} = x_{2}
		\end{cases}
	\end{equation*}
	L'autospazio generale è:
	\begin{equation*}
		\left\{ \left. \begin{rowequmat}{c}
			\dfrac{1}{8}x_{2} \\ [.3em]
			x_{2}
		\end{rowequmat} \: \right| \: x_{2} \in \mathbb{R}\right\} =
		\left\{
			x_{2}
			\begin{rowequmat}{c}
				\frac{1}{8} \\ [.3em]
				1
			\end{rowequmat}
		\right\}
	\end{equation*}
	Una base dell'autospazio dell'autovalore $\lambda=1$ è:
	\begin{equation*}
		\left\{\begin{rowequmat}{c}
			\frac{1}{8} \\ [.3em]
			1
		\end{rowequmat}\right\}
	\end{equation*}\newpage

	\noindent
	Sostituzione del valore $\lambda=\dfrac{1}{2}$:
	\begin{equation*}
		p_{B}\left(\dfrac{1}{2}\right) = \det\left(B - \dfrac{1}{2}\mathrm{Id}_{2}\right) = \begin{rowequmat}{cc}
			1-\frac{1}{2} & 0 \\ [.3em]
			4 & \frac{1}{2} - \frac{1}{2}
		\end{rowequmat} =
		\begin{rowequmat}{cc}
			\frac{1}{2} & 0 \\ [.3em]
			4 & 0
		\end{rowequmat}
	\end{equation*}
	L'eliminazione di Gauss è inutile, infatti, il sistema lineare equivalente è:
	\begin{equation*}
		\begin{cases}
			\frac{1}{2}x_{1} = 0
			4x_{1} = 0
		\end{cases}
		\longrightarrow
		\begin{cases}
			x_{1} = 0
			x_{1} = 0
		\end{cases}
	\end{equation*}
	Per cui, il sistema generale:
	\begin{equation*}
		\begin{cases}
			x_{1} = 0 \\
			x_{2} = x_{2}
		\end{cases}
	\end{equation*}
	Quindi, l'autospazio generale:
	\begin{equation*}
		\left\{
			\left.
			\begin{pmatrix}
				0 \\ x_{2}
			\end{pmatrix} \: \right| \:
			x_{2} \in \mathbb{R}
		\right\} = \left\{
			x_{2} \begin{pmatrix}
				0 \\
				1
			\end{pmatrix}
		\right\}
	\end{equation*}
	Una base dell'autospazio dell'autovalore $\lambda=\frac{1}{2}$ è:
	\begin{equation*}
		\left\{\begin{pmatrix}
			0 \\ 1
		\end{pmatrix}\right\}
	\end{equation*}\newpage

	\subsubsection{Parte b}

	\textcolor{Green4}{\textbf{\emph{Si verifichi che la matrice $B$ è diagonalizzabile e si trovino la matrice diagonale $D$ e le matrici $S$, $S^{-1}$ tali che $B=SDS^{-1}$.}}}\newline

	\noindent
	Si verifica che la matrice $B$ è diagonalizzabile guardando le molteplicità algebriche e geometriche degli autovalori:
	\begin{equation*}
		\begin{array}{lll}
			\text{Molteplicità algebrica} &\longrightarrow& m_{1}\left(1\right) = 1, m_{2}\left(\dfrac{1}{2}\right) = 1 \\ [2em]
			\text{Molteplicità geometrica} &\longrightarrow& m_{1}\left(1\right) = 2 - \mathrm{rk}\left(B - 1\mathrm{Id}_{2}\right) = 2 - 1 = 1 \\ [1.5em]
			&& m_{2}\left(\dfrac{1}{2}\right) = 2 - 1 = 1
		\end{array}
	\end{equation*}
	La matrice è diagonalizzabile perché ogni molteplicità geometrica corrisponde alla rispettiva molteplicità algebrica.\newline

	\noindent
	La matrice diagonale $D$ si trova inserendo nella diagonale principale gli autovalori:
	\begin{equation*}
		D = \begin{rowequmat}{cc}
			1 & 0 \\ [.3em]
			0 & \frac{1}{2}
		\end{rowequmat}
	\end{equation*}
	La matrice $S$ è composta dalle basi trovate:
	\begin{equation*}
		S = \begin{rowequmat}{cc}
			\frac{1}{8}	& 0 \\ [.3em]
			1			& 1
		\end{rowequmat}
	\end{equation*}
	L'inversa di $S$, cioè $S^{-1}$, si ottiene affiancando la matrice identità a destra e ottenendo una forma ridotta a sinistra:
	\begin{equation*}
		\begin{rowequmat}{cc|cc}
			\frac{1}{8}	& 0 & 1 & 0 \\ [.3em]
			1			& 1 & 0 & 1
		\end{rowequmat}
		\xlongrightarrow[sub]{E_{1,2}\left(-8\right)}
		\begin{rowequmat}{cc|cc}
			\frac{1}{8}	& 0 & 1 & 0 \\ [.3em]
			0			& 1 & -8 & 1
		\end{rowequmat}
	\end{equation*}
	Quindi la matrice $S^{-1}$ è composta nel seguente modo:
	\begin{equation*}
		\begin{pmatrix}
			1 & 0 \\
			-8 & 1
		\end{pmatrix}
	\end{equation*}\newpage

	\subsection{Esercizio 3}

	\textcolor{Green4}{\textbf{\emph{Sia $f: \mathbb{R}^{3} \rightarrow \mathbb{R}^{3}$ l'applicazione lineare tale che $f\left(v\right) = \begin{pmatrix}
		x+2y-z \\
		2x+4y-2z \\
		-3x-6y+3z
	\end{pmatrix}$ per ogni $v=\begin{pmatrix}
		x \\ y \\ z
	\end{pmatrix} \in \mathbb{R}^{3}$.}}}

	\longline

	\subsubsection{Parte a}

	\textcolor{Green4}{\textbf{\emph{Si calcoli la matrice $M$ associata a $f$ rispetto alla base canonica.}}}\newline

	\noindent
	La base canonica:
	\begin{equation*}
		\mathcal{C}_{\mathbb{C}^{3}} = \left\{ \begin{pmatrix} 1 \\ 0 \\ 0 \end{pmatrix}, \begin{pmatrix} 0 \\ 1 \\ 0 \end{pmatrix}, \begin{pmatrix} 0 \\ 0 \\ 1 \end{pmatrix}\right\}
	\end{equation*}
	Si calcola la rispettiva matrice associata:
	\begin{equation*}
		\begin{array}{rll}
			f\begin{pmatrix}
				1 \\ 0 \\ 0
			\end{pmatrix} &=& \begin{pmatrix}
				1 + 2 \cdot 0 - 0 \\
				2 \cdot 1 + 4 \cdot 0 - 2 \cdot 0 \\
				-3 \cdot 1 - 6 \cdot 0 + 3 \cdot 0
			\end{pmatrix} = \begin{pmatrix}
				1 \\
				2 \\
				-3
			\end{pmatrix} \\ [2em]
			%
			f\begin{pmatrix}
				0 \\ 1 \\ 0
			\end{pmatrix} &=& \begin{pmatrix}
				0 + 2 \cdot 1 - 0 \\
				2 \cdot 0 + 4 \cdot 1 - 2 \cdot 0 \\
				-3 \cdot 0 - 6 \cdot 1 + 3 \cdot 0
			\end{pmatrix} = \begin{pmatrix}
				2 \\
				4 \\
				-6
			\end{pmatrix} \\ [2em]
			%
			f\begin{pmatrix}
				0 \\ 0 \\ 1
			\end{pmatrix} &=& \begin{pmatrix}
				0 + 2 \cdot 0 - 1 \\
				2 \cdot 0 + 4 \cdot 0 - 2 \cdot 1 \\
				- 3 \cdot 0 - 6 \cdot 0 + 3 \cdot 1
			\end{pmatrix} = \begin{pmatrix}
				-1 \\
				-2 \\
				3
			\end{pmatrix} \\ [2em]
			M &=& \begin{pmatrix}
				1	& 2		& -1 \\
				2	& 4		& -2 \\
				-3	& -6	& 3 
			\end{pmatrix}
		\end{array}
	\end{equation*}\newpage

	\subsubsection{Parte b}

	\textcolor{Green4}{\textbf{\emph{Si determinino la dimensione e una base dell'immagine $Im\left(f\right) = C\left(M\right)$ di $f$ e dello spazio nullo $N\left(f\right) = N\left(M\right)$ di $f$.}}}\newline

	\noindent
	Si utilizza l'eliminazione di Gauss per ottenere una forma ridotta, così da trovare una base e la dimensione di $C\left(M\right)$:
	\begin{equation*}
		\begin{pmatrix}
			1	& 2		& -1 \\
			2	& 4		& -2 \\
			-3	& -6	& 3 
		\end{pmatrix}
		\xlongrightarrow[E_{1,3}\left(3\right)]{E_{1,2}\left(-2\right)}
		\begin{pmatrix}
			1	& 2	& -1 \\
			0	& 0	& 0 \\
			0	& 0	& 0 
		\end{pmatrix}
	\end{equation*}
	Il rango della matrice risulta essere $\mathrm{rk}\left(M\right) = 1$. Esso corrisponde alla dimensione dello spazio generato dai vettori considerati inizialmente, ovvero:
	\begin{equation*}
		\dim\left(M\right) = \mathrm{rk}\left(M\right) \longrightarrow 1 = 1
	\end{equation*}
	Quindi, una base di $C\left(M\right)$ è:
	\begin{equation*}
		\left\{
			\begin{pmatrix}
				1 \\ 2 \\ -3
			\end{pmatrix}
		\right\}
	\end{equation*}
	Per trovare la dimensione della nullità di $f$, è necessario calcolare la dimensione del nucleo dell'applicazione lineare:
	\begin{equation*}
		M v = \mathbf{0}_{\mathbb{R}^{3}} \longrightarrow
		\begin{pmatrix}
			1	& 2		& -1 \\
			2	& 4		& -2 \\
			-3	& -6	& 3
		\end{pmatrix}
		\begin{pmatrix}
			x \\ y \\ z
		\end{pmatrix} =
		\begin{pmatrix}
			0 \\ 0 \\ 0
		\end{pmatrix}
	\end{equation*}
	Si cerca la forma ridotta della matrice aumentata:
	\begin{equation*}
		\begin{rowequmat}{ccc|c}
			1	& 2		& -1 & 0 \\
			2	& 4		& -2 & 0 \\
			-3	& -6	& 3  & 0
		\end{rowequmat}
		\xlongrightarrow[E_{1,3}\left(3\right)]{E_{1,2}\left(-2\right)}
		\begin{rowequmat}{ccc|c}
			1	& 2	& -1 	& 0 \\
			0	& 0	& 0 	& 0 \\
			0	& 0	& 0  	& 0
		\end{rowequmat}
	\end{equation*}
	Anche in questo caso il rango è pari ad 1. Quindi, grazie al teorema di Rouché Capelli, la dimensione del nucleo di $f$, ovvero della nullità di $f$ è:
	\begin{equation*}
		\dim\left(\ker\left(f\right)\right) = \dim\left(N\left(M\right)\right) = n - \mathrm{rk}\left(M\right) = 3-1 = 2
	\end{equation*}
	La dimensione è corretta, confermato dal teorema della somma delle dimensioni dell'applicazioni lineari $f$:\
	\begin{equation*}
		\dim\left(f\right) = \dim\left(\ker\left(f\right)\right) + \dim\left(\mathrm{Im}\left(f\right)\right)
	\end{equation*}
	Una base dipende dal sistema lineare equivalente:
	\begin{equation*}
		\begin{cases}
			x + 2y -z = 0
		\end{cases}
		\longrightarrow
		\begin{cases}
			x = -2y + z
		\end{cases}
	\end{equation*}
	In generale:
	\begin{equation*}
		\begin{cases}
			x = -2y + z \\
			y = y \\
			z = z
		\end{cases}
	\end{equation*}
	Quindi, le basi sono:
	\begin{equation*}
		N\left(M\right) = \left\{
			y \begin{pmatrix}
				-2 \\
				1 \\
				0
			\end{pmatrix} +
			z \begin{pmatrix}
				1 \\
				0 \\
				1
			\end{pmatrix}
		\right\} \longrightarrow \left\{
			\begin{pmatrix}
				-2 \\
				1 \\
				0
			\end{pmatrix},
			\begin{pmatrix}
				1 \\
				0 \\
				1
			\end{pmatrix}
		\right\}
	\end{equation*}

	\newpage
	\subsubsection{Parte c}

	\textcolor{Green4}{\textbf{\emph{Si dica se l'applicazione lineare $f$ è un isomorfismo.}}}\newline

	\noindent
	Per verificare l'isomorfismo, è necessario controllare che l'applicazione lineare sia invertibile. La matrice $M$ associata ad $f$, è invertibile se e solo se il suo determinante è diverso da zero e di rango massimo.

	Purtroppo, in questo caso, il determinante è uguale a 0 e dunque l'applicazione lineare $f$ non è isomorfa.

	\longline

	\subsubsection{Parte d}

	\textcolor{Green4}{\textbf{\emph{Si calcoli la matrice $N$ associata a $f$ rispetto alla base $\mathcal{B} = \left\{\begin{pmatrix}
		1 \\ 0 \\ 0
	\end{pmatrix},\begin{pmatrix}
		1 \\ 1 \\ 0
	\end{pmatrix},\begin{pmatrix}
		1 \\ 1 \\ 1
	\end{pmatrix}\right\}$ nel dominio e rispetto alla base canonica nel codominio.}}}\newline

	\noindent
	La matrice $N$ associata a $f$ rispetto alla base $\mathcal{B}$ è:
	\begin{equation*}
		\begin{array}{rll}
			f\begin{pmatrix}
				1 \\ 0 \\ 0
			\end{pmatrix} &=& \begin{pmatrix}
				1 + 2 \cdot 0 - 0 \\
				2 \cdot 1 + 4 \cdot 0 - 2 \cdot 0 \\
				-3 \cdot 1 - 6 \cdot 0 + 3 \cdot 0
			\end{pmatrix} = \begin{pmatrix}
				1 \\ 2 \\ -3
			\end{pmatrix} \\ [2em]
			%
			f\begin{pmatrix}
				1 \\ 1 \\ 0
			\end{pmatrix} &=& \begin{pmatrix}
				1 + 2 \cdot 1 - 0 \\
				2 \cdot 1 + 4 \cdot 1 - 2 \cdot 0 \\
				-3 \cdot 1 - 6 \cdot 1 + 3 \cdot 0
			\end{pmatrix} = \begin{pmatrix}
				3 \\ 6 \\ -9
			\end{pmatrix} \\ [2em]
			%
			f\begin{pmatrix}
				1 \\ 1 \\ 1
			\end{pmatrix} &=& \begin{pmatrix}
				1 + 2 \cdot 1 - 1 \\
				2 \cdot 1 + 4 \cdot 1 - 2 \cdot 1 \\
				-3 \cdot 1 - 6 \cdot 1 + 3 \cdot 1
			\end{pmatrix} = \begin{pmatrix}
				2 \\ 4 \\ -6
			\end{pmatrix} \\ [2em]
			N &=& \begin{pmatrix}
				1	& 3		& 2		\\
				2	& 6		& 4		\\
				-3	& -9	& -6
			\end{pmatrix}
		\end{array}
	\end{equation*}\newpage

	\subsection{Esercizio 4}

	(\textbf{4 punti}) Vero o falso? Si motivi la risposta!\newline

	\longline

	\subsubsection{Parte a}

	\textcolor{Green4}{\textbf{\emph{Il numero complesso $\dfrac{-3+6i}{2+i}$ in forma algebrica è $3i$.}}}\newline

	\noindent
	Dato il rapporto tra i numeri complessi, si moltiplica la frazione per il complesso coniugato al denominatore:
	\begin{equation*}
		\dfrac{-3+6i}{2+i} \cdot \dfrac{2-i}{2-i} = \dfrac{-6+12i+3i-6i^{2}}{4-2i+2i-i^{2}} = \dfrac{15i}{5} = 3i
	\end{equation*}
	Quindi la forma algebrica è $3i$ e la risposta all'esercizio è: vero!\newline

	\longline

	\subsubsection{Parte b}

	\textcolor{Green4}{\textbf{\emph{L'insieme $\left\{\begin{rowequmat}{c}
		\frac{2}{\sqrt{5}} \\ [.3em]
		\frac{1}{\sqrt{5}}
	\end{rowequmat}, \begin{pmatrix}
		i \\ 0
	\end{pmatrix}\right\}$ è una base ortonormale di $\mathbb{C}^{2}$.}}}\newline

	\noindent
	Per verificare che sia una base ortonormale, i vettori devono essere linearmente indipendenti. Quindi, la moltiplicazione dei due vettori, dovrebbe avere come risultato zero. Si verifica:
	\begin{equation*}
		\begin{pmatrix}
			\frac{2}{\sqrt{5}} & \frac{1}{\sqrt{5}}
		\end{pmatrix}^{H}
		\times
		\begin{pmatrix}
			i \\ 0
		\end{pmatrix} =
			\dfrac{2}{\sqrt{5}} \cdot i + \dfrac{1}{\sqrt{5}} \cdot 0 \ne 0
	\end{equation*}
	È stata effettuata una $H$-trasposta per consentire la moltiplicazione. La risposta all'esercizio è: falso! L'insieme non è una base ortonormale di $\mathbb{C}^{2}$.\newline

	\longline

	\subsection{Esercizio 5}

	(\textbf{1 punto}) Sia $M \in M_{m \times n}\left(\mathbb{K}\right)$ una matrice. Si dimostri la seguente affermazione: se $M$ ammette un'inversa destra $R \in M_{n \times m}\left(\mathbb{K}\right)$, allora il sistema lineare $Ax = b$ ammette soluzione per qualsiasi $b \in M_{m \times 1}\left(\mathbb{K}\right)$.\newline

	\noindent
	Dimostrazione lasciata al lettore.
	
	\newpage
	\section{Esame del 02/09/2022}

	\subsection{Esercizio 1}

	(\textbf{8 punti})

	\subsubsection{Punto a}

	\textcolor{Green4}{\textbf{\emph{Calcolare $z^{4}$ dove $z = 2\left(\cos\frac{\pi}{4} + i \sin\frac{\pi}{4}\right)$.}}}\newline

	\noindent
	Per calcolare l'elevazione a potenza, si scrive la radice quadrata:
	\begin{equation*}
		z = \sqrt[4]{2\left(\cos\frac{\pi}{4} + i \sin\frac{\pi}{4}\right)}
	\end{equation*}\newpage

	\noindent
	E si calcolata la $n$-esima radice quadrata di un numero complesso con $k = 0, ... n-1$ e la seguente formula:
	\begin{equation*}
		\begin{array}{rll}
			z &=& \sqrt[n]{r}\left(\cos\left(\dfrac{\theta + 2 k \pi}{n}\right) + i \sin\left(\dfrac{\theta + 2 k \pi}{n}\right) \right) \\ [2em]
			&\downarrow& \text{sostituzione} \\ [1.5em]
			&=& \sqrt[4]{2}\left(\cos\left(\dfrac{\dfrac{\pi}{4} + 2 k \pi}{4}\right) + i \sin\left(\dfrac{\dfrac{\pi}{4} + 2 k \pi}{4}\right)\right) \\ [2em]
			&\downarrow& \text{sostituzione dei valori }k \\ [1.5em]
			z_{1} &=& \sqrt[4]{2}\left(\cos\left(\dfrac{\dfrac{\pi}{4} + 2 \cdot 0 \cdot \pi}{4}\right) + i \sin\left(\dfrac{\dfrac{\pi}{4} + 2 \cdot 0 \cdot \pi}{4}\right)\right) \\ [2em]
			&=& \sqrt[4]{2}\left(\cos\left(\dfrac{\pi}{16}\right) + i \sin\left(\dfrac{\pi}{16}\right)\right) \\ [1.5em]
			z_{2} &=& \sqrt[4]{2}\left(\cos\left(\dfrac{\dfrac{\pi}{4} + 2 \cdot 1 \cdot \pi}{4}\right) + i \sin\left(\dfrac{\dfrac{\pi}{4} + 2 \cdot 1 \cdot \pi}{4}\right)\right) \\ [2em]
			&=& \sqrt[4]{2}\left(\cos\left(\dfrac{9\pi}{16}\right) + i \sin\left(\dfrac{9\pi}{16}\right)\right) \\ [1.5em]
			z_{3} &=& \sqrt[4]{2}\left(\cos\left(\dfrac{\dfrac{\pi}{4} + 2 \cdot 2 \cdot \pi}{4}\right) + i \sin\left(\dfrac{\dfrac{\pi}{4} + 2 \cdot 2 \cdot \pi}{4}\right)\right) \\ [2em]
			&=& \sqrt[4]{2}\left(\cos\left(\dfrac{17\pi}{16}\right) + i \sin\left(\dfrac{17\pi}{16}\right)\right) \\ [1.5em]
			z_{4} &=& \sqrt[4]{2}\left(\cos\left(\dfrac{\dfrac{\pi}{4} + 2 \cdot 3 \cdot \pi}{4}\right) + i \sin\left(\dfrac{\dfrac{\pi}{4} + 2 \cdot 3 \cdot \pi}{4}\right)\right) \\ [2em]
			&=& \sqrt[4]{2}\left(\cos\left(\dfrac{25\pi}{16}\right) + i \sin\left(\dfrac{25\pi}{16}\right)\right) \\ [1.5em]
		\end{array}
	\end{equation*}\newpage

	\subsubsection{Parte b}
	
	\textcolor{Green4}{\textbf{\emph{Considerare la seguente matrice:}}
	\begin{equation*}
		A = \begin{pmatrix}
			1	& -2	& 0		\\
			3k	& 8+2k	& k-1	\\
			0	& 8k+8k	& 0
		\end{pmatrix}
	\end{equation*}
	\textbf{\emph{Calcolare il rango $\mathrm{rk}\left(A\right)$ di $A$ e il determinante $\det\left(A\right)$ di $A$ al variare di $k \in \mathbb{R}$. Determinare per quali $k \in \mathbb{R}$ la matrice $A$ è invertibile.}}}\newline

	\noindent
	Si esegue l'eliminazione di Gauss per calcolare il rango e il determinante:
	\begin{equation*}
		\begin{pmatrix}
			1	& -2	& 0		\\
			3k	& 8+2k	& k-1	\\
			0	& 8k+8k	& 0
		\end{pmatrix}
		\xlongrightarrow[E_{2,3}\left(-\frac{16k}{8+8k}\right)]{E_{1,2}\left(-3k\right)}
		\begin{rowequmat}{ccc}
			1	& -2	& 0		\\ [.3em]
			0	& 8+8k	& k-1	\\ [.3em]
			0	& 0		& \frac{-2k^{2}+2k}{1+k}
		\end{rowequmat}
	\end{equation*}
	Il rango dipende dal valore di $k$, quindi:
	\begin{equation*}
		\mathrm{rk}\left(A\right) = \begin{cases}
			2 & k = -1 \lor k = 0 \lor k = 1 \\
			3 & \text{altrimenti}
		\end{cases}
	\end{equation*}
	Invece, il determinante corrisponde a:
	\begin{equation*}
		\det\left(A\right) = 1 \cdot \left(8+8k\right) \cdot \left(\dfrac{-2k^{2}+2k}{1+k}\right) = 8\left(1+k\right) \cdot \left(\dfrac{-2k^{2}+2k}{1+k}\right) = -16k^{2} + 16k
	\end{equation*}
	La matrice $A$ è invertibile se e solo se il determinante è diverso da zero e ha rango massimo, quindi essa è invertibile se e solo se $k \ne -1 \land k \ne 0 \land k \ne 1$.\newpage

	\subsection{Esercizio 2}

	(\textbf{8 punti}) Considerare la seguente matrice:
	\begin{equation*}
		B = \begin{pmatrix}
			1 & 0 & 0 \\
			0 & 1 & -3 \\
			-1 & -2 & 8
		\end{pmatrix}
	\end{equation*}

	\longline

	\subsubsection{Punto a}

	\textcolor{Green4}{\textbf{\emph{Trovare una forma ridotta e una decomposizione LU di B.}}}\newline

	\noindent
	Una forma ridotta si può trovare eseguendo l'eliminazione di Gauss:
	\begin{equation*}
		\begin{pmatrix}
			1 & 0 & 0 \\
			0 & 1 & -3 \\
			-1 & -2 & 8
		\end{pmatrix}
		\xlongrightarrow[E_{2,3}\left(2\right)]{E_{1,3}\left(1\right)}
		\begin{pmatrix}
			1 & 0 & 0 \\
			0 & 1 & -3 \\
			0 & 0 & 2
		\end{pmatrix}
	\end{equation*}
	La forma ridotta corrisponde alla matrice $U$, questo è possibile affermarlo poiché non sono stati effettuati scambi di righe. Quindi, si compone anche la matrice $L$ inserendo gli scalari invertiti di segno:
	\begin{equation*}
		L = \begin{pmatrix}
			1 & 0 & 0 \\
			0 & 1 & 0 \\
			-1 & -2 & 1
		\end{pmatrix}
	\end{equation*}
	Riscrivendo il risultato:
	\begin{equation*}
		B = LU \rightarrow \begin{pmatrix}
			1 & 0 & 0 \\
			0 & 1 & -3 \\
			-1 & -2 & 8
		\end{pmatrix} = \begin{pmatrix}
			1 & 0 & 0 \\
			0 & 1 & 0 \\
			-1 & -2 & 1
		\end{pmatrix} \begin{pmatrix}
			1 & 0 & 0 \\
			0 & 1 & -3 \\
			0 & 0 & 2
		\end{pmatrix}
	\end{equation*}\newpage

	\subsubsection{Punto b}
	
	\textcolor{Green4}{\textbf{\emph{Determinare se $B$ è invertibile e motivare la risposta. Se sì, calcolare l'inversa di $B$.}}}\newline

	\noindent
	Per determinare se $B$ è invertibile, è necessario verificare che il determinante sia diverso da zero e che il rango sia massimo. Data la forma ridotta della matrice, ovvero $U$, il determinante della matrice $B$ e il rango sono:
	\begin{gather*}
		\det\left(B\right) = 1 \cdot 1 \cdot 2 = 2 \\
		\mathrm{rk}\left(B\right) = 3
	\end{gather*}
	Quindi, la matrice $B$ è invertibile. Si procede con il calcolo di un'inversa. Si affianca la matrice identità a destra, si esegue l'eliminazione di Gauss e si ottiene una inversa:
	\begin{gather*}
		\begin{rowequmat}{ccc|ccc}
			1 & 0 & 0 	& 1 & 0 & 0 \\
			0 & 1 & -3 	& 0 & 1 & 0 \\
			-1 & -2 & 8 & 0 & 0 & 1
		\end{rowequmat}
		\xlongrightarrow[E_{2,3}\left(2\right)]{E_{1,3}\left(1\right)}
		\begin{rowequmat}{ccc|ccc}
			1 & 0 	& 0 	& 1 & 0 & 0 \\
			0 & 1 	& -3 	& 0 & 1 & 0 \\
			0 & 0 	& 2 	& 1 & 2 & 1
		\end{rowequmat}\\
		\\
		\xlongrightarrow[E_{3,2}\left(3\right)]{E_{3}\left(\frac{1}{2}\right)}
		\begin{rowequmat}{ccc|ccc}
			1 & 0 	& 0 	& 1 & 0 & 0 \\ [.3em]
			0 & 1 	& 0 	& \frac{3}{2} & 4 & \frac{3}{2} \\ [.3em]
			0 & 0 	& 1 	& \frac{1}{2} & 1 & \frac{1}{2}
		\end{rowequmat}
	\end{gather*}
	L'inversa di $B$ corrisponde a:
	\begin{equation*}
		B^{-1} = \begin{rowequmat}{ccc}
			1 & 0 & 0 \\ [.3em]
			\frac{3}{2} & 4 & \frac{3}{2} \\ [.3em]
			\frac{1}{2} & 1 & \frac{1}{2}
		\end{rowequmat}
	\end{equation*}\newpage
	
	\subsection{Esercizio 3}

	(\textbf{8 punti}) Considerare la seguente matrice:
	\begin{equation*}
		D = \begin{pmatrix}
			1 & 0 \\
			-1 & 1 \\
			0 & 1
		\end{pmatrix}
	\end{equation*}
	
	\longline

	\subsubsection{Punto a}

	\textcolor{Green4}{\textbf{\emph{Trovare una base del sottospazio $C\left(D\right)$ di $\mathbb{R}^{3}$ generato dalle colonne di $D$ e una base dello spazio nullo $N\left(D^{T}\right)$ della trasposta $D^{T}$ di $D$.}}}\newline

	\noindent
	Una base del sottospazio $C\left(D\right)$ è possibile trovarla applicando l'eliminazione di Gauss:
	\begin{equation*}
		\begin{pmatrix}
			1 & 0 \\
			-1 & 1 \\
			0 & 1
		\end{pmatrix}
		\xlongrightarrow[E_{2,3}\left(-1\right)]{E_{1,2}\left(1\right)}
		\begin{pmatrix}
			1 & 0 \\
			0 & 1 \\
			0 & 0
		\end{pmatrix}
	\end{equation*}
	Il rango della matrice, ovvero la dimensione del sottospazio $C\left(D\right)$ è pari a $2$, quindi una base è:
	\begin{equation*}
		\left\{
			\begin{pmatrix}
				1 \\ -1 \\ 0
			\end{pmatrix},
			\begin{pmatrix}
				0 \\ 1 \\ 1
			\end{pmatrix}
		\right\}
	\end{equation*}
	Per calcolare la base di uno spazio nullo, è necessario eseguire la trasposizione della matrice $D$:
	\begin{equation*}
		\begin{array}{rll}
			D &=& \begin{pmatrix}
				1 & 0 \\
				-1 & 1 \\
				0 & 1
			\end{pmatrix} \\ [2em]
			D^{T} &=& \begin{pmatrix}
				1 & -1 & 0 \\
				0 & 1  & 1
			\end{pmatrix}
		\end{array}
	\end{equation*}
	Per trovare una base dello spazio nullo, è necessario trovare una forma ridotta della matrice aumentata:
	\begin{equation*}
		\begin{rowequmat}{ccc|c}
			1 & -1 & 0 & 0 \\
			0 & 1  & 1 & 0
		\end{rowequmat}
	\end{equation*}
	In questo caso, la matrice aumentata è già nella sua forma ridotta, per cui, si procede a scrivere il sistema lineare equivalente:
	\begin{equation*}
		\begin{cases}
			x - y = 0 \\
			y + z = 0
		\end{cases} \longrightarrow
		\begin{cases}
			x - \left(-z\right) = 0 \\
			y = -z
		\end{cases} \longrightarrow
		\begin{cases}
			x = -z \\
			y = -z
		\end{cases}
	\end{equation*}
	La dimensione è dello spazio nullo corrisponde a:
	\begin{equation*}
		\dim\left(N\left(D^{T}\right)\right) = n - \mathrm{rk}\left\{D^{T}\right\} = 3 - 2 = 1
	\end{equation*}
	Quindi, in generale:
	\begin{equation*}
		\begin{cases}
			x = -z \\
			y = -z 
		\end{cases}
	\end{equation*}
	Quindi, una base è:
	\begin{equation*}
		N\left(D^{T}\right) = \left\{
			\left. \begin{pmatrix}
				-z \\
				-z
			\end{pmatrix} \: \right| \:
			z \in \mathbb{R}
		\right\} \longrightarrow
		\left\{
			z\begin{pmatrix}
				-1 \\
				-1
			\end{pmatrix}
		\right\} \longrightarrow
		\left\{
			\begin{pmatrix}
				-1 \\
				-1
			\end{pmatrix}
		\right\}
	\end{equation*}\newpage

	\subsubsection{Punto b}

	\textcolor{Green4}{\textbf{\emph{Mostrare che l'insieme $\mathcal{C} = \left\{
		\begin{pmatrix}
			1 \\ -1 \\ 0
		\end{pmatrix},
		\begin{pmatrix}
			0 \\ 1 \\ 1
		\end{pmatrix},
		\begin{pmatrix}
			-1 \\ -1 \\ 1
		\end{pmatrix}
	\right\}$ è una base di $\mathbb{R}^{3}$.}}}\newline

	\noindent
	Si costruisce la matrice associata:
	\begin{equation*}
		\begin{pmatrix}
			1 & 0 & -1 \\
			-1 & 1 & -1 \\
			0 & 1 & 1
		\end{pmatrix}
	\end{equation*}
	Si calcola il rango della matrice usando l'eliminazione di Gauss:
	\begin{equation*}
		\begin{pmatrix}
			1 & 0 & -1 \\
			-1 & 1 & -1 \\
			0 & 1 & 1
		\end{pmatrix}
		\xlongrightarrow[E_{2,3}\left(-1\right)]{E_{1,2}\left(1\right)}
		\begin{pmatrix}
			1 & 0 & -1 \\
			0 & 1 & -2 \\
			0 & 0 & 3
		\end{pmatrix}
	\end{equation*}
	Il rango della matrice è $3$, quindi la dimensione della base è corretta. Inoltre, i \emph{pivot} nelle colonne dominanti corrispondono ai rispettivi vettori nella base. Quindi, si conclude che $\mathcal{C}$ è una base di $\mathbb{R}^{3}$.\newline

	\longline

	\subsubsection{Punto c}

	\textcolor{Green4}{\textbf{\emph{Considerare la seguente base di $\mathbb{R}^{3} : \mathcal{B} = \left\{
		\begin{pmatrix}
			1 \\ 0 \\ 0
		\end{pmatrix},
		\begin{pmatrix}
			0 \\ 1 \\ 0
		\end{pmatrix},
		\begin{pmatrix}
			0 \\ 1 \\ 1
		\end{pmatrix}
	\right\}$. Calcolare la matrice $N=A_{\mathcal{C} \rightarrow \mathcal{B}}$, cioè l'unica matrice $N$ tale che $\mathcal{C}_{\mathcal{B}}\left(v\right) = Nc_{\mathcal{C}}\left(v\right)$ per ogni $v \in \mathbb{R}^{3}$.}}}\newline
	
	\noindent
	Per trovare la matrice del cambiamento di base da $\mathcal{C}\rightarrow \mathcal{B}$, è necessario esprimere i vettori di $\mathcal{C}$ come combinazioni lineari dei vettori di $\mathcal{B}$:
	\begin{itemize}
		\item La prima combinazione lineare:
		\begin{equation*}
			\begin{array}{rll}
				\mathcal{C}_{1} &=& a_{1}\mathcal{B}_{1} + b_{1}\mathcal{B}_{2} + c_{1}\mathcal{B}_{3} \\ [2em]
				\begin{pmatrix}
					1 \\ -1 \\ 0
				\end{pmatrix} &=& a_{1} \begin{pmatrix}
					1 \\ 0 \\ 0
				\end{pmatrix} + b_{1} \begin{pmatrix}
					0 \\ 1 \\ 0
				\end{pmatrix} + c_{1} \begin{pmatrix}
					0 \\ 1 \\ 1
				\end{pmatrix}
			\end{array}
		\end{equation*}
		Il relativo sistema:
		\begin{equation*}
			\begin{cases}
				a_{1} = 1 \\
				b_{1} + c_{1} = -1 \\
				c_{1} = 0
			\end{cases} \longrightarrow
			\begin{cases}
				a_{1} = 1 \\
				b_{1} = -1 \\
				c_{1} = 0
			\end{cases}
		\end{equation*}
		La soluzione dunque:
		\begin{equation*}
			\begin{pmatrix}
				1 \\ -1 \\ 0
			\end{pmatrix} = 1 \begin{pmatrix}
				1 \\ 0 \\ 0
			\end{pmatrix} + -1 \begin{pmatrix}
				0 \\ 1 \\ 0
			\end{pmatrix} + 0 \begin{pmatrix}
				0 \\ 1 \\ 1
			\end{pmatrix}
		\end{equation*}\newpage

		\item La seconda combinazione lineare:
		\begin{equation*}
			\begin{array}{rll}
				\mathcal{C}_{2} &=& a_{2}\mathcal{B}_{1} + b_{2}\mathcal{B}_{2} + c_{2}\mathcal{B}_{3} \\ [2em]
				\begin{pmatrix}
					0 \\ 1 \\ 1
				\end{pmatrix} &=& a_{2} \begin{pmatrix}
					1 \\ 0 \\ 0
				\end{pmatrix} + b_{2} \begin{pmatrix}
					0 \\ 1 \\ 0
				\end{pmatrix} + c_{2} \begin{pmatrix}
					0 \\ 1 \\ 1
				\end{pmatrix}
			\end{array}
		\end{equation*}
		Il relativo sistema:
		\begin{equation*}
			\begin{cases}
				a_{2} = 0 \\
				b_{2} + c_{1} = 1 \\
				c_{2} = 1
			\end{cases} \longrightarrow
			\begin{cases}
				a_{2} = 0 \\
				b_{2} = 0 \\
				c_{2} = 1
			\end{cases}
		\end{equation*}
		La soluzione dunque:
		\begin{equation*}
			\begin{pmatrix}
				0 \\ 1 \\ 1
			\end{pmatrix} = 0 \begin{pmatrix}
				1 \\ 0 \\ 0
			\end{pmatrix} + 0 \begin{pmatrix}
				0 \\ 1 \\ 0
			\end{pmatrix} + 1 \begin{pmatrix}
				0 \\ 1 \\ 1
			\end{pmatrix}
		\end{equation*}
		
		\item La terza combinazione lineare:
		\begin{equation*}
			\begin{array}{rll}
				\mathcal{C}_{3} &=& a_{3}\mathcal{B}_{1} + b_{3}\mathcal{B}_{2} + c_{3}\mathcal{B}_{3} \\ [2em]
				\begin{pmatrix}
					-1 \\ -1 \\ 1
				\end{pmatrix} &=& a_{3} \begin{pmatrix}
					1 \\ 0 \\ 0
				\end{pmatrix} + b_{3} \begin{pmatrix}
					0 \\ 1 \\ 0
				\end{pmatrix} + c_{3} \begin{pmatrix}
					0 \\ 1 \\ 1
				\end{pmatrix}
			\end{array}
		\end{equation*}
		Il relativo sistema:
		\begin{equation*}
			\begin{cases}
				a_{3} = -1 \\
				b_{3} + c_{1} = -1 \\
				c_{3} = 1
			\end{cases} \longrightarrow
			\begin{cases}
				a_{3} = -1 \\
				b_{3} = -2 \\
				c_{3} = 1
			\end{cases}
		\end{equation*}
		La soluzione dunque:
		\begin{equation*}
			\begin{pmatrix}
				-1 \\ -1 \\ 1
			\end{pmatrix} = -1 \begin{pmatrix}
				1 \\ 0 \\ 0
			\end{pmatrix} + -2 \begin{pmatrix}
				0 \\ 1 \\ 0
			\end{pmatrix} + 1 \begin{pmatrix}
				0 \\ 1 \\ 1
			\end{pmatrix}
		\end{equation*}
	\end{itemize}
	La matrice del cambiamento di base $\mathcal{C} \rightarrow \mathcal{B}$ è:
	\begin{equation*}
		N_{\mathcal{C} \rightarrow \mathcal{B}} = \begin{pmatrix}
			1	& 0	& -1 \\
			-1	& 0	& -2 \\
			0	& 1	& 1
		\end{pmatrix}
	\end{equation*}\newpage

	\subsection{Esercizio 4}

	(\textbf{6 punti}) Considerare la seguente matrice: $M = \begin{pmatrix}
		1 & -4 \\ -1 & 1
	\end{pmatrix}$. Vero o falso? Si motivi la risposta!
	
	\longline

	\subsubsection{Parte a}

	\textcolor{Green4}{\textbf{\emph{Gli autovalori di $M$ sono $\lambda_{1} = -1$ e $\lambda_{2} = 3$.}}}\newline

	\noindent
	Si scrive il polinomio caratteristico come il determinante di:
	\begin{equation*}
		\begin{array}{lll}
			p_{M}\left(\lambda\right) &=& \det\left(M - \lambda\mathrm{Id}_{2}\right) \\ [1.5em]
			&=& \det\left[
				\begin{pmatrix}
					1	& -4 \\
					-1	& 1	 
				\end{pmatrix} - \lambda
				\begin{pmatrix}
					1 & 0 \\
					0 & 1 
				\end{pmatrix}
			\right] \\ [2.5em]
			&=& \det\left[
				\begin{pmatrix}
					1	& -4 \\
					-1	& 1	
				\end{pmatrix} -
				\begin{pmatrix}
					\lambda & 0 \\
					0 & \lambda 
				\end{pmatrix}
			\right] \\ [2.5em]
			&=& \det\left[
				\begin{pmatrix}
					1 - \lambda	& -4 \\
					-1	& 1	- \lambda
				\end{pmatrix}
			\right]
		\end{array}
	\end{equation*}
	Si esegue l'eliminazione di Gauss:
	\begin{equation*}
		\begin{pmatrix}
			1 - \lambda	& -4 \\
			-1	& 1	- \lambda
		\end{pmatrix}
		\xlongrightarrow{E_{1,2}\left(\frac{1}{1-\lambda}\right)}
		\begin{rowequmat}{cc}
			1 - \lambda	& -4 \\ [.3em]
			0	& 1	- \lambda - \frac{4}{1-\lambda}
		\end{rowequmat}
	\end{equation*}
	E si calcola il determinante:
	\begin{equation*}
		\begin{array}{lll}
			\det\left(M - \lambda \mathrm{Id}_{3}\right) &=& \left(1-\lambda\right) \left(1	- \lambda - \dfrac{4}{1-\lambda}\right) \\ [1em]
			&=& 1 - \lambda -\lambda +\lambda^{2} -4 \\ [1em]
			&=& \lambda^{2} -2 \lambda -3
		\end{array}
	\end{equation*}
	Con gli autovalori forniti, il polinomio caratteristico si annulla, per cui essi sono gli autovalori di $M$. La risposta è: vero!\newpage

	\subsubsection{Parte b}

	\textcolor{Green4}{\textbf{\emph{La matrice $M$ è diagonalizzabile.}}}\newline

	\noindent
	La matrice $M$ è diagonalizzabile se la molteplicità algebrica corrisponde all'ordine della matrice e se la molteplicità geometrica corrisponde a quella algebrica.\newline

	\noindent
	La molteplicità algebrica di $\lambda_{1}$ è 1 e la molteplicità algebrica di $\lambda_{2}$ è 1. Adesso si calcolano le molteplicità geometriche:
	\begin{itemize}
		\item Con $\lambda_{1} = -1$:
		\begin{gather*}
			n - \mathrm{rk}\left(M - \left(-1\right)\mathrm{Id}_{2}\right) \\
			\\
			n - \mathrm{rk}\left(\begin{pmatrix}
				2 & -4 \\
				-1 & 2
			\end{pmatrix} \xrightarrow{E_{1,2}\left(\frac{1}{2}\right)}
			\begin{pmatrix}
				2 & -4 \\
				0 & 0
			\end{pmatrix}\right) \\
			\\
			2 - 1 = 1
		\end{gather*}

		\item  Con $\lambda_{2} = 3$:
		\begin{gather*}
			n - \mathrm{rk}\left(M - \left(3\right)\mathrm{Id}_{2}\right) \\
			\\
			n - \mathrm{rk}\left(\begin{pmatrix}
				-2 & -4 \\
				-1 & -2
			\end{pmatrix} \xrightarrow{E_{1,2}\left(-\frac{1}{2}\right)}
			\begin{pmatrix}
				-2 & -4 \\
				0 & 0
			\end{pmatrix}\right) \\
			\\
			2 - 1 = 1
		\end{gather*}
	\end{itemize}
	Le molteplicità algebriche corrispondo alle rispettive molteplicità geometriche, per cui $M$ è diagonalizzabile. La risposta è: vero!\newline

	\longline

	\subsubsection{Parte c}

	\textcolor{Green4}{\textbf{\emph{Le colonne di $M$ sono ortogonali.}}}\newline

	\noindent
	Una matrice è ortogonale se $A A^{T} = A^{T}A = \mathrm{Id}_{n}$. Quindi, si calcola la trasposta:
	\begin{equation*}
		\begin{array}{rll}
			M &=& \begin{pmatrix}
				1  & -4 \\
				-1 & 1
			\end{pmatrix} \\
			M^{T} &=& \begin{pmatrix}
				1 & -1 \\
				-4 & 1 \\
			\end{pmatrix}
		\end{array}
	\end{equation*}
	E si esegue la moltiplicazione:
	\begin{equation*}
		M \cdot M^{T} = \begin{pmatrix}
			1  & -4 \\
			-1 & 1
		\end{pmatrix}
		\begin{pmatrix}
			1 & -1 \\
			-4 & 1 \\
		\end{pmatrix} = \begin{pmatrix}
			17 & -5 \\
			-5 & 2
		\end{pmatrix}
	\end{equation*}
	La matrice risultante non è una matrice identità di ordine 2, quindi le colonne di $M$ non sono ortogonali. La risposta è: falso!\newpage

	\subsection{Esercizio 5}

	(\textbf{1 punto}) Sia $A \in M_{n \times n}\left(\mathbb{K}\right)$. Dimostrare la seguente affermazione: se $A$ possiede un'inversa destra $R$ e un'inversa sinistra $L$, allora $L = R$.\newline

	\noindent
	Dimostrazione lasciata al lettore.\newpage

	\section{Esame del 20/02/2023}

	\subsection{Esercizio 1}

	(\textbf{8 punti})\newline

	\longline

	\subsubsection{Punto a}

	\textcolor{Green4}{\textbf{\emph{Si consideri la seguente matrice:}}
	\begin{equation*}
		A = \begin{pmatrix}
			\alpha	& \alpha+3	& 2\alpha	\\
			\alpha	& 2\alpha+2	& 3\alpha	\\
			2\alpha	& \alpha+7	& 4\alpha	
		\end{pmatrix}
	\end{equation*}
	\textbf{\emph{Si studi $\det\left(A\right), \mathrm{rk}\left(A\right)$ e l'invertibilità di $A$ al variare di $\alpha \in \mathbb{R}$.}}}\newline

	\noindent
	Si esegue l'eliminazione di Gauss:
	\begin{equation*}
		\begin{pmatrix}
			\alpha	& \alpha+3	& 2\alpha	\\
			\alpha	& 2\alpha+2	& 3\alpha	\\
			2\alpha	& \alpha+7	& 4\alpha	
		\end{pmatrix}
		\xlongrightarrow[E_{1,3}\left(-2\right)]{E_{1,2}\left(-1\right)}
		\begin{pmatrix}
			\alpha	& \alpha+3	& 2\alpha	\\
			0		& \alpha-1	&  \alpha	\\
			0		& -\alpha+1	& 0	
		\end{pmatrix}
		\xlongrightarrow{E_{2,3}\left(1\right)}
		\begin{pmatrix}
			\alpha	& \alpha+3	& 2\alpha	\\
			0		& \alpha-1	& \alpha	\\
			0		& 0			& \alpha	
		\end{pmatrix}
	\end{equation*}
	Il determinante è:
	\begin{equation*}
		\begin{array}{rll}
			\det\left(A\right) &=& \alpha \cdot \left(\alpha -1\right) \cdot \alpha \\
			&=& \alpha^{2} \cdot \left(\alpha-1\right)
		\end{array}
	\end{equation*}
	Il rango è:
	\begin{equation*}
		\mathrm{rk}\left(A\right) = \begin{cases}
			1	& \alpha = 0 \\
			2	& \alpha = 1 \\
			3	& \text{altrimenti}
		\end{cases}
	\end{equation*}
	La matrice $A$ è invertibile se e solo se il determinante è diverso da zero, ovvero se $\alpha \ne 0 \land \alpha \ne 1$, e se il rango è massimo, quindi con la stessa condizione del determinante.\newpage

	\subsubsection{Parte b}

	\textcolor{Green4}{\textbf{\emph{Si calcoli $z^{6}$ dove $z = \frac{2}{\sqrt{3}-i}+\frac{1}{i}$.}}}\newline

	\noindent
	Si ottiene la forma trigonometrica:
	\begin{equation*}
		z = \dfrac{2}{\sqrt{3}-i} \cdot \dfrac{\sqrt{3}+i}{\sqrt{3}+i} - i = \dfrac{\sqrt{3}}{2} - \dfrac{1}{2}i = \cos\left(\dfrac{1}{6}\pi\right) + i \sin\left(\dfrac{1}{6}\pi\right)
	\end{equation*}
	E si eleva a potenza\footnote{\href{https://www.andreaminini.org/matematica/numeri-complessi/la-potenza-del-numero-complesso}{Tutorial online}}:
	\begin{equation*}
		\begin{array}{lll}
			z^{n} &=& d^{n} \cdot \left[\cos\left(n \cdot \alpha\right) + i \sin\left(n \cdot \alpha\right)\right] \\ [1em]
			z^{6} &=& 1^{6}\left(\cos\left(6 \cdot \dfrac{1}{6}\pi\right) + i \sin\left(6 \cdot \dfrac{1}{6}\pi\right)\right) \\[1em]
			&=& \cos\left(\pi\right) + i \sin\left(\pi\right) \\ [1em]
			&=& -1 + i \cdot 0 \\ [1em]
			&=& -1
		\end{array}
	\end{equation*}\newpage
	
	\subsection{Esercizio 2}
	
	(\textbf{8 punti}) Si consideri la seguente matrice:
	\begin{equation*}
		B = \begin{pmatrix}
			0 & 0 & 1 \\
			0 & 0 & -1 \\
			1 & -1 & 0
		\end{pmatrix}
	\end{equation*}
	
	\longline
	
	\subsubsection{Punto a}
	
	\textcolor{Green4}{\textbf{\emph{Si calcolino tutti gli autovalori di $B$ su $\mathbb{R}$ e si trovino delle basi dei loro autospazi.}}}\newline
	
	\noindent
	Per calcolare gli autovalori di $B$, è necessario innanzitutto calcolare il polinomio caratteristico, ovvero il determinante del risultato tra la differenza della matrice e la matrice identità per $\lambda$:
	\begin{equation*}
		\begin{array}{rcl}
			p_{B}\left(\lambda\right) &=& \det\left(B - \lambda \cdot \mathrm{Id}_{3}\right) \\ [.7em]
			&=& \left(
			\begin{pmatrix}
				0 & 0 & 1 \\
				0 & 0 & -1 \\
				1 & -1 & 0
			\end{pmatrix} -
			\begin{pmatrix}
				\lambda & 0 & 0 \\
				0 & \lambda & 0 \\
				0 & 0 & \lambda
			\end{pmatrix}
			\right) \\ [2.5em]
			&=& \left(
			\begin{pmatrix}
				-\lambda & 0 & 1 \\
				0 & -\lambda & -1 \\
				1 & -1 & -\lambda
			\end{pmatrix}
			\right)
		\end{array}
	\end{equation*}
	Si risolve rapidamente con EG per ottenere il determinante.
	\begin{gather*}
		\begin{rowequmat}{ccc}
			-\lambda & 0 & 1 \\
			0 & -\lambda & -1 \\
			1 & -1 & -\lambda
		\end{rowequmat}
		\xlongrightarrow[E_{2,3}\left(-\frac{1}{\lambda}\right)]{E_{1,3}\left(\frac{1}{\lambda}\right)}
		\begin{rowequmat}{ccc}
			-\lambda & 0 & 1 \\ [.3em]
			0 & -\lambda & -1 \\ [.3em]
			0 & 0 & \frac{2-\lambda^{2}}{\lambda}
		\end{rowequmat} \\
		\\
		\det\left(B - \lambda \cdot \mathrm{Id}_{3}\right) = \left(-\lambda\right) \cdot \left(-\lambda\right) \cdot \left(\dfrac{2-\lambda^{2}}{\lambda}\right) = 2\lambda - \lambda^{3}
	\end{gather*}
	Dunque, gli autovalori sono quei valori che pongono il polinomio caratteristico uguale a zero:
	\begin{gather*}
		2\lambda - \lambda^{3} = 0 \longrightarrow \lambda\left(2 - \lambda^{2}\right) = 0 \\
		\lambda_{1} = 0; \hspace{2em} \lambda_{2} = -\sqrt{2}; \hspace{2em} \lambda_{3} = \sqrt{2}
	\end{gather*}
	Adesso si procede con la ricerca degli autospazi.\newpage
	
	\noindent
	Si sostituisce l'autovalore $\lambda_{1} = 0$ all'interno della matrice:
	\begin{equation*}
		\begin{pmatrix}
			-\lambda & 0 & 1 \\
			0 & -\lambda & -1 \\
			1 & -1 & -\lambda
		\end{pmatrix}
		\longrightarrow
		\begin{pmatrix}
			0 & 0 & 1 \\
			0 & 0 & -1 \\
			1 & -1 & 0
		\end{pmatrix}
	\end{equation*}
	Si cerca la forma ridotta EG, si calcola il sistema generale e si cercano delle basi degli autospazi. In questo caso la matrice è quasi alla sua forma ridotta, per cui non si applicano ulteriori operazioni di riduzione. Si passa alla scrittura del sistema generale:
	\begin{equation*}
		\begin{cases}
			x_{3} = 0 \\
			-x_{3} = 0 \\
			x_{1} - x_{2} = 0
		\end{cases}
		\longrightarrow
		\begin{cases}
			x_{3} = 0 \\
			x_{3} = 0 \\
			x_{1} = x_{2}
		\end{cases}
	\end{equation*}
	Quindi, il sistema generale è:
	\begin{equation*}
		\begin{cases}
			x_{1} = x_{2} \\
			x_{2} = x_{2} \\
			x_{3} = 0
		\end{cases}
	\end{equation*}
	Se ne deduce che l'autospazio dell'autovalore $\lambda_{1} = 0$ è:
	\begin{equation*}
		\left\{\left.\begin{pmatrix}
			x_{2} \\ x_{2} \\ 0
		\end{pmatrix} \: \right| \: x_{2} \in \mathbb{R}\right\} = \left\{x_{2} \begin{pmatrix}
			1 \\ 1 \\ 0
		\end{pmatrix}\right\}
	\end{equation*}
	Quindi, una base dell'autospazio, con $x_{2} = 1$ è:
	\begin{equation*}
		\left\{\begin{pmatrix}
			1 \\ 1 \\ 0
		\end{pmatrix}\right\}
	\end{equation*}\newpage
	
	\noindent
	Si sostituisce l'autovalore $\lambda_{2} = -\sqrt{2}$ all'interno della matrice:
	\begin{equation*}
		\begin{pmatrix}
			-\lambda & 0 & 1 \\
			0 & -\lambda & -1 \\
			1 & -1 & -\lambda
		\end{pmatrix}
		\longrightarrow
		\begin{pmatrix}
			\sqrt{2} & 0 & 1 \\
			0 & \sqrt{2} & -1 \\
			1 & -1 & \sqrt{2}
		\end{pmatrix}
	\end{equation*}
	Si cerca una forma ridotta con EG:
	\begin{equation*}
		\begin{pmatrix}
			\sqrt{2} & 0 & 1 \\
			0 & \sqrt{2} & -1 \\
			1 & -1 & \sqrt{2}
		\end{pmatrix}
		\xlongrightarrow[E_{2,3}\left(\frac{1}{\sqrt{2}}\right)]{E_{1,3}\left(-\frac{1}{\sqrt{2}}\right)}
		\begin{rowequmat}{ccc}
			\sqrt{2} & 0 & 1 \\ 
			0 & \sqrt{2} & -1 \\ 
			0 & 0 & 0
		\end{rowequmat}
	\end{equation*}
	Si scrive il sistema e si risolve:
	\begin{gather*}
		\begin{cases}
			\sqrt{2}x_{1} + x_{3} = 0 \\
			\sqrt{2}x_{2} - x_{3} = 0
		\end{cases}
		\longrightarrow
		\begin{cases}
			x_{3} = -\sqrt{2}x_{1} \\
			\sqrt{2}x_{2} - \left(-\sqrt{2}x_{1}\right) = 0
		\end{cases} \\
		\longrightarrow
		\begin{cases}
			x_{3} = -\sqrt{2}x_{1} \\
			\frac{1}{\sqrt{2}} \left(\sqrt{2}x_{2} + \sqrt{2}x_{1}\right) = 0 \cdot \frac{1}{\sqrt{2}}
		\end{cases}
		\longrightarrow
		\begin{cases}
			x_{3} = -\sqrt{2}x_{1} \\
			x_{2} + x_{1} = 0
		\end{cases}
		\longrightarrow
		\begin{cases}
			x_{3} = -\sqrt{2}x_{1} \\
			x_{2} = -x_{1}
		\end{cases}
	\end{gather*}
	Il sistema generale:
	\begin{equation*}
		\begin{cases}
			x_{1} = x_{1} \\
			x_{2} = -x_{1} \\
			x_{3} = -\sqrt{2}x_{1}
		\end{cases}
	\end{equation*}
	Dunque, l'autospazio dell'autovalore $\lambda_{2} = -\sqrt{2}$ è:
	\begin{equation*}
		\left\{\left. \begin{pmatrix}
			x_{1} \\
			-x_{1} \\
			-\sqrt{2}x_{1}
		\end{pmatrix} \: \right| \: x_{1} \in \mathbb{R}\right\} =
		\left\{x_{1} \begin{pmatrix}
			1 \\ -1 \\ -\sqrt{2}
		\end{pmatrix}\right\}
	\end{equation*}
	Una base dell'autospazio con $x_{1} = 1$ è:
	\begin{equation*}
		\left\{\begin{pmatrix}
			1 \\ -1 \\ -\sqrt{2}
		\end{pmatrix}\right\}
	\end{equation*}\newpage
	
	\noindent
	Si sostituisce l'autovalore $\lambda_{3} = \sqrt{2}$ all'interno della matrice:
	\begin{equation*}
		\begin{pmatrix}
			-\lambda & 0 & 1 \\
			0 & -\lambda & -1 \\
			1 & -1 & -\lambda
		\end{pmatrix}
		\longrightarrow
		\begin{pmatrix}
			-\sqrt{2} & 0 & 1 \\
			0 & -\sqrt{2} & -1 \\
			1 & -1 & -\sqrt{2}
		\end{pmatrix}
	\end{equation*}
	Si cerca una forma ridotta con EG:
	\begin{equation*}
		\begin{pmatrix}
			-\sqrt{2} & 0 & 1 \\
			0 & -\sqrt{2} & -1 \\
			1 & -1 & -\sqrt{2}
		\end{pmatrix}
		\xlongrightarrow[E_{2,3}\left(-\frac{1}{\sqrt{2}}\right)]{E_{1,3}\left(\frac{1}{\sqrt{2}}\right)}
		\begin{rowequmat}{ccc}
			-\sqrt{2} & 0 & 1 \\ [.3em]
			0 & -\sqrt{2} & -1 \\ [.3em]
			0 & 0 & 0
		\end{rowequmat}
	\end{equation*}
	Si scrive e si risolve il sistema:
	\begin{gather*}
		\begin{cases}
			-\sqrt{2}x_{1} + x_{3} = 0 \\
			-\sqrt{2}x_{2} - x_{3} = 0
		\end{cases}
		\longrightarrow
		\begin{cases}
			x_{3} = \sqrt{2}x_{1} \\
			-\sqrt{2}x_{2} - \sqrt{2}x_{1} = 0 \\
		\end{cases} \\
		\longrightarrow
		\begin{cases}
			x_{3} = \sqrt{2}x_{1} \\
			x_{2} = -x_{1}
		\end{cases}
	\end{gather*}
	Il sistema generale dunque è:
	\begin{equation*}
		\begin{cases}
			x_{1} = x_{1} \\
			x_{2} = -x_{1} \\
			x_{3} = \sqrt{2}x_{1}
		\end{cases}
	\end{equation*}
	L'autospazio dell'autovalore $\lambda_{3} = \sqrt{2}$:
	\begin{equation*}
		\left\{\left. x_{1} \begin{pmatrix}
			x_{1} \\
			-x_{1} \\
			\sqrt{2}x_{1}
		\end{pmatrix} \: \right| \: x_{1} \in \mathbb{R}\right\}
	\end{equation*}
	Una base dell'autospazio con $x_{1} = \sqrt{2}$ è:
	\begin{equation*}
		\left\{\begin{pmatrix}
			1 \\ -1 \\ \sqrt{2}
		\end{pmatrix}\right\}
	\end{equation*}\newpage
	
	\subsubsection{Punto b}
	
	\textcolor{Green4}{\textbf{\emph{Si verifichi che la matrice $B$ è diagonalizzabile e si trovino la matrice diagonale $D$ e le matrici $S$, $S^{-1}$ tali che $B = SDS^{-1}$.}}}\newline
	
	\noindent
	Una matrice è diagonalizzabile se la somma della molteplicità algebrica degli autovalori è uguale all'ordine della matrice e se le molteplicità geometriche degli autovalori corrispondono alle rispettive molteplicità algebriche.\newline
	
	\noindent
	Il polinomio caratteristico, trovato al punto precedente, è:
	\begin{equation*}
		p_{B}\left(\lambda\right) = -\lambda^{3} + 2\lambda
	\end{equation*}
	Le soluzioni trovate, con le rispettive molteplicità algebriche, sono:
	\begin{equation*}
		\begin{array}{rclcl}
			\lambda_{1} &=& 0			&\longrightarrow& m_{1} = 1 \\
			\lambda_{2} &=& -\sqrt{2}	&\longrightarrow& m_{2} = 1 \\
			\lambda_{3} &=& \sqrt{2}	&\longrightarrow& m_{3} = 1 \\
		\end{array}
	\end{equation*}
	La somma delle molteplicità algebriche è uguale all'ordine della matrice. Quindi, si prosegue la verifica e si controlla le molteplicità geometriche:
	\begin{equation*}
		\begin{array}{rclcl}
			\lambda_{1} &=& 0			&\longrightarrow& m_{1} = n - \mathrm{rk}\left(B - 0 \cdot \mathrm{Id}_{3}\right) = 3 - 2 = 1 \\ [.3em]
			\lambda_{2} &=& -\sqrt{2}	&\longrightarrow& m_{2} = n - \mathrm{rk}\left(B - \left(-\sqrt{2}\right) \cdot \mathrm{Id}_{3}\right) = 3 - 2 = 1 \\ [.3em]
			\lambda_{3} &=& \sqrt{2}	&\longrightarrow& m_{3} = n - \mathrm{rk}\left(B - \sqrt{2} \cdot \mathrm{Id}_{3}\right) = 3 - 2 = 1
		\end{array}
	\end{equation*}
	Anche le molteplicità geometriche corrispondono alle rispettive molteplicità algebriche. Quindi, si prosegue con la scrittura della matrice $D$ che corrisponde alla matrice diagonale di $B$:
	\begin{equation*}
		D = \begin{pmatrix}
			0 & 0 & 0 \\
			0 & -\sqrt{2} & 0 \\
			0 & 0 & \sqrt{2}
		\end{pmatrix}
	\end{equation*}
	La matrice ha sulla diagonale principale gli autovalori di $B$. La matrice $S$, invece, è composta dalle relative basi trovate nel punto precedente:
	\begin{equation*}
		S = \begin{pmatrix}
			1 & 1			& 1			\\
			1 & -1			& -1		\\
			0 & -\sqrt{2}	& \sqrt{2}	\\
		\end{pmatrix}
	\end{equation*}
	Ovviamente, la matrice inversa è l'inversa di $S$, quindi si affianca una matrice identità a destra e si eseguono le operazioni di EG per ottenere una matrice identità a sinistra:
	\begin{equation*}
		\left(S|\mathrm{Id}_{3}\right) = \begin{rowequmat}{ccc|ccc}
			1 & 1			& 1			& 1 & 0 & 0 \\
			1 & -1			& -1		& 0 & 1 & 0 \\
			0 & -\sqrt{2}	& \sqrt{2}	& 0 & 0 & 1
		\end{rowequmat}
		\xlongrightarrow[E_{2,3}\left(-\frac{\sqrt{2}}{2}\right)]{E_{1,2}\left(-1\right)}
		\begin{rowequmat}{ccc|ccc}
			1 & 1			& 1			& 1 & 0 & 0 \\ [.3em]
			0 & -2			& -2		& -1 & 1 & 0 \\ [.3em]
			0 & 0	& 2\sqrt{2}	& \frac{\sqrt{2}}{2} & -\frac{\sqrt{2}}{2} & 1
		\end{rowequmat}
	\end{equation*}\newpage
	
	\begin{equation*}
		\xlongrightarrow[E_{3,2}\left(\frac{1}{\sqrt{2}}\right)]{E_{2,1}\left(\frac{1}{2}\right)}
		\begin{rowequmat}{ccc|ccc}
			1 & 0			& 0			& \frac{1}{2} & \frac{1}{2} & 0 \\ [.3em]
			0 & -2			& 0		& -\frac{1}{2} & \frac{1}{2} & \frac{1}{\sqrt{2}} \\ [.3em]
			0 & 0	& 2\sqrt{2}	& \frac{\sqrt{2}}{2} & -\frac{\sqrt{2}}{2} & 1
		\end{rowequmat}
		\xlongrightarrow[E_{3}\left(\frac{1}{2\sqrt{2}}\right)]{E_{2}\left(-\frac{1}{2}\right)}
		\begin{rowequmat}{ccc|ccc}
			1 & 0			& 0			& \frac{1}{2} & \frac{1}{2} & 0 \\ [.3em]
			0 & 1			& 0		& \frac{1}{4} & -\frac{1}{4} & -\frac{1}{2\sqrt{2}} \\ [.3em]
			0 & 0	& 1	& \frac{1}{4} & -\frac{1}{4} & \frac{1}{2\sqrt{2}}
		\end{rowequmat}
	\end{equation*}
	La matrice a destra risulta essere l'inversa di $S$:
	\begin{equation*}
		S^{-1} = \begin{rowequmat}{ccc}
			\frac{1}{2} & \frac{1}{2} & 0 \\ [.3em]
			\frac{1}{4} & -\frac{1}{4} & -\frac{1}{2\sqrt{2}} \\ [.3em]
			\frac{1}{4} & -\frac{1}{4} & \frac{1}{2\sqrt{2}}
		\end{rowequmat}
	\end{equation*}
	Si verifica che le matrici trovate siano corrette:
	\begin{equation*}
		\begin{array}{rcl}
			B &=& SDS^{-1} \\
			\begin{pmatrix}
				0 & 0 & 1 \\
				0 & 0 & -1 \\
				1 & -1 & 0
			\end{pmatrix} &=&
			\begin{pmatrix}
				1 & 1			& 1			\\
				1 & -1			& -1		\\
				0 & -\sqrt{2}	& \sqrt{2}	\\
			\end{pmatrix}
			\begin{pmatrix}
				0 & 0 & 0 \\
				0 & -\sqrt{2} & 0 \\
				0 & 0 & \sqrt{2}
			\end{pmatrix}
			\begin{rowequmat}{ccc}
				\frac{1}{2} & \frac{1}{2} & 0 \\ [.3em]
				\frac{1}{4} & -\frac{1}{4} & -\frac{1}{2\sqrt{2}} \\ [.3em]
				\frac{1}{4} & -\frac{1}{4} & \frac{1}{2\sqrt{2}}
			\end{rowequmat} \\
			\begin{pmatrix}
				0 & 0 & 1 \\
				0 & 0 & -1 \\
				1 & -1 & 0
			\end{pmatrix} &=&
			\begin{pmatrix}
				0	& -\sqrt{2} & \sqrt{2} \\
				0	& \sqrt{2} & -\sqrt{2} \\
				0	& 2 & 2
			\end{pmatrix}
			\begin{rowequmat}{ccc}
				\frac{1}{2} & \frac{1}{2} & 0 \\ [.3em]
				\frac{1}{4} & -\frac{1}{4} & -\frac{1}{2\sqrt{2}} \\ [.3em]
				\frac{1}{4} & -\frac{1}{4} & \frac{1}{2\sqrt{2}}
			\end{rowequmat} \\
			\begin{pmatrix}
				0 & 0 & 1 \\
				0 & 0 & -1 \\
				1 & -1 & 0
			\end{pmatrix} &=&
			\begin{pmatrix}
				0 & 0 & 1 \\
				0 & 0 & -1 \\
				1 & -1 & 0
			\end{pmatrix}
		\end{array}
	\end{equation*}\newpage
	
	\subsection{Esercizio 3}
	
	(\textbf{8 punti}) Si considerino le seguenti matrici:
	\begin{equation*}
		C = \begin{pmatrix}
			3 & 0 \\ 2 & 13 \\ 0 & 8
		\end{pmatrix}
		\hspace{2em}
		D = \begin{pmatrix}
			3 & 2 & 0 \\ 1 & -2 & 3
		\end{pmatrix}
	\end{equation*}
	
	\longline
	
	\subsubsection{Punto a}
	
	\textcolor{Green4}{\textbf{\emph{Si trova una base di ciascuno dei seguenti sottospazi di $\mathbb{R}^{3}$:}}}
	\begin{enumerate}
		\item[i.] \textcolor{Green4}{\textbf{\emph{Il sottospazio $C(C)$ generato dalle colonne di $C$.}}}
		\item[ii.] \textcolor{Green4}{\textbf{\emph{Lo spazio nullo $N(D)$ di $D$.}}}
		\item[iii.] \textcolor{Green4}{\textbf{\emph{La somma $C(C) + N(D)$ dei sottospazi $C(C)$ e $N(D)$.}}}
	\end{enumerate}
	Una base del sottospazio $C(C)$ è possibile trovarla eseguendo la EG sulla matrice $C$:
	\begin{equation*}
		\begin{pmatrix}
			3 & 0 \\
			2 & 13 \\
			0 & 8
		\end{pmatrix}
		\xlongrightarrow[E_{2,3}\left(-\frac{8}{13}\right)]{E_{1,2}\left(-\frac{2}{3}\right)}
		\begin{pmatrix}
			3 & 0 \\
			0 & 13 \\
			0 & 0
		\end{pmatrix}
	\end{equation*}
	E prendere in considerazione le colonne dei pivot. Quindi, le basi corrispondono alle colonne originarie rispetto ai pivot:
	\begin{equation*}
		\left\{\begin{pmatrix}
			3 \\ 2 \\ 0
		\end{pmatrix}, \begin{pmatrix}
			0 \\ 13 \\ 8 
		\end{pmatrix}\right\}
	\end{equation*}
	Una base dello spazio nullo $N(D)$ è possibile trovarlo eseguendo EG su D, ma considerando quest'ultima come la matrice aumentata, e calcolando le soluzioni del sistema lineare relativo:
	\begin{equation*}
		\begin{rowequmat}{ccc|c}
			3 & 2 & 0  & 0 \\
			1 & -2 & 3 & 0
		\end{rowequmat}
		\xlongrightarrow[E_{3}\left(-3\right)]{E_{1,2}\left(-\frac{1}{3}\right)}
		\begin{rowequmat}{ccc|c}
			3 & 2 & 0  & 0 \\ [.3em]
			0 & 8 & -9 & 0
		\end{rowequmat}
	\end{equation*}
	Si scrive e si risolve il relativo sistema:
	\begin{equation*}
		\begin{cases}
			3x_{1} + 2x_{2} = 0 \\
			8x_{2} - 9x_{3} = 0
		\end{cases}
		\longrightarrow
		\begin{cases}
			x_{2} = -\frac{3}{2}x_{1} \\
			8\left(-\frac{3}{2}x_{1}\right) - 9x_{3} = 0
		\end{cases}
		\longrightarrow
		\begin{cases}
			x_{2} = -\frac{3}{2}x_{1} \\
			x_{3} = -\frac{12}{9}x_{1} = -\frac{4}{3}x_{1}
		\end{cases}
	\end{equation*}
	Quindi, il sistema generale:
	\begin{equation*}
		\begin{cases}
			x_{1} = x_{1} \\
			x_{2} = -\frac{3}{2}x_{1} \\
			x_{3} = -\frac{4}{3}x_{1}
		\end{cases}
	\end{equation*}
	E una base:
	\begin{equation*}
		\left\{\left.\begin{rowequmat}{c}
			x_{1} \\ [.3em]
			-\frac{3}{2}x_{1} \\ [.3em]
			-\frac{4}{3}x_{1}
		\end{rowequmat} \: \right| \: x_{1} \in \mathbb{R}\right\} = \left\{x_{1}\begin{rowequmat}{c}
			1 \\ [.3em]
			-\frac{3}{2} \\ [.3em]
			-\frac{4}{3}
		\end{rowequmat}\right\}
		\xlongrightarrow{x_{1} = 1}
		\left\{\begin{rowequmat}{c}
			1 \\ [.3em]
			-\frac{3}{2} \\ [.3em]
			-\frac{4}{3}
		\end{rowequmat}\right\}
	\end{equation*}\newpage
	
	\noindent
	La somma dei sottospazi $C(C)$ e $N(D)$ si forma unendo le basi trovate in un'unica matrice e cercando le rispettive basi guardando i pivot della matrice:
	\begin{equation*}
		A = C(C) + N(D) = \begin{rowequmat}{ccc}
			3 & 0  & 1 \\ [.3em]
			2 & 13 & -\frac{3}{2} \\ [.3em]
			0 & 8  & -\frac{4}{3}
		\end{rowequmat}
	\end{equation*}
	Si utilizza EG per trovare una forma ridotta:
	\begin{equation*}
		\begin{rowequmat}{ccc}
			3 & 0  & 1 \\ [.3em]
			2 & 13 & -\frac{3}{2} \\ [.3em]
			0 & 8  & -\frac{4}{3}
		\end{rowequmat}
		\xlongrightarrow[E_{2,3}\left(-\frac{8}{13}\right)]{E_{1,2}\left(-\frac{2}{3}\right)}
		\begin{rowequmat}{ccc}
			3 & 0  & 1 \\ [.3em]
			0 & 13 & -\frac{13}{6} \\ [.3em]
			0 & 0  & 0
		\end{rowequmat}
	\end{equation*}
	I pivot sono 3 e 13, quindi una base di $C(C) + N(D)$:
	\begin{equation*}
		\left\{\begin{pmatrix}
			3 \\ 2 \\ 0
		\end{pmatrix}, \begin{pmatrix}
			0 \\ 13 \\ 8 
		\end{pmatrix}\right\}
	\end{equation*}\newpage
	
	\subsubsection{Punto b}
	
	\textcolor{Green4}{\textbf{\emph{Si calcoli la dimensione dell'intersezione $C(C) \cap N(D)$ dei sottospazi $C(C)$ e $N(D)$.}}}\newline
	
	\noindent
	Per calcolare la dimensione dell'intersezione è necessario ricordare la formula di Grassman\footnote{\href{https://www.youmath.it/domande-a-risposte/view/1090-chiarimenti-sul-teorema-di-grassman.html}{Link utile}}:
	\begin{equation*}
		\dim\left(C(C) + N(D)\right) = \dim\left(C(C)\right) + \dim\left(N(D)\right) - \dim\left(C(C) \cap N(D)\right)
	\end{equation*}
	La dimensione di $C(C)$ è pari a 2, infatti una sua base:
	\begin{equation*}
		\left\{\begin{pmatrix}
			3 \\ 2 \\ 0
		\end{pmatrix}, \begin{pmatrix}
			0 \\ 13 \\ 8 
		\end{pmatrix}\right\}
	\end{equation*}
	La dimensione di $N(D)$ è pari a 1, infatti una sua base:
	\begin{equation*}
		\left\{\begin{rowequmat}{c}
			1 \\ [.3em]
			-\frac{3}{2} \\ [.3em]
			-\frac{4}{3}
		\end{rowequmat}\right\}
	\end{equation*}
	La dimensione di $C(C) + N(D)$ è pari a 2, infatti una sua base:
	\begin{equation*}
		\left\{\begin{pmatrix}
			3 \\ 2 \\ 0
		\end{pmatrix}, \begin{pmatrix}
			0 \\ 13 \\ 8 
		\end{pmatrix}\right\}
	\end{equation*}
	Quindi:
	\begin{equation*}
		\begin{array}{rcl}
			2 &=& 2 + 1 - \dim\left(C(C) \cap N(D)\right) \\
			\dim\left(C(C) \cap N(D)\right) &=& 2 + 1 -2 \\
			&=& 1
		\end{array}
	\end{equation*}\newpage
	
	\subsection{Esercizio 4}
	
	(\textbf{6 punti}) Si considerino la matrice $P = \begin{pmatrix}
		3 & i \\
		-1 & 0
	\end{pmatrix}$ Vero o falso? Si giustifichi la risposta!
	
	\longline
	
	\subsubsection{Punto a}
	
	\textcolor{Green4}{\textbf{\emph{La matrice $P$ è hermitiana, ovvero $P = P^{H}$.}}}\newline
	
	\noindent
	La matrice hermitiana corrisponde alla matrice coniugata, ovvero è necessario cambiare i segni ai numeri immaginari:
	\begin{equation*}
		P^{H} = \begin{pmatrix}
			3 & -i \\ -1 & 0
		\end{pmatrix}
	\end{equation*}
	Dunque la risposta è falso, ovvero $P \ne P^{H}$.
	
	\longline
	
	\subsubsection{Punto b}
	
	\textcolor{Green4}{\textbf{\emph{La matrice $P$ è invertibile.}}}\newline
	
	\noindent
	La matrice $P$ è invertibile se e solo se ha $\det\left(P\right) \ne 0$ e $\mathrm{rk}\left(P\right) = 2$, ovvero rango massimo.
	\begin{gather*}
		\begin{pmatrix}
			3 & i \\
			-1 & 0
		\end{pmatrix}
		\xlongrightarrow{E_{1,2}\left(\frac{1}{3}\right)}
		\begin{rowequmat}{cc}
			3 & i \\ [.3em]
			0 & \frac{1}{3}i
		\end{rowequmat} \\
		\det\left(P\right) = 3 \cdot \dfrac{1}{3}i = i \\
		\mathrm{rk}\left(P\right) = 2
	\end{gather*}
	Vero, la matrice $P$ è invertibile.
	
	\longline
	
	\subsubsection{Punto c}
	
	\textcolor{Green4}{\textbf{\emph{Il vettore $c_{\mathcal{B}}\left(Pv\right)$ è uguale a $\begin{pmatrix}
		-1 \\ 4+i
	\end{pmatrix}$ dove $\mathcal{B} = \left\{\begin{pmatrix}
		1 \\ 1
	\end{pmatrix}, \begin{pmatrix}
		1 \\ 0
	\end{pmatrix}\right\}$ e $v = \begin{pmatrix}
		1 \\ 1
	\end{pmatrix}$.}}}\newline

	\noindent
	Non risolto.
	
	\longline
	
	\subsection{Esercizio 5}
	
	(\textbf{1 punti}) Sia $A \in M_{m\times n}\left(\mathbb{C}\right)$ e sia $b \in \mathbb{C}^{m}$. Si dimostri che, se $p \in \mathbb{C}^{n}$ è una soluzione particolare di $Ax = b$, allora ogni soluzione è della forma $p + u$ per qualche $u \in N\left(A\right)$.\newpage
	
	\section{Esame del 28/06/2023}
	
	\subsection{Esercizio 1}
	
	(\textbf{7 punti}) Si consideri il numero complesso $z = - \frac{\sqrt{2}}{2} - \frac{\sqrt{2}}{2}i$.
	
	\longline
	
	\subsubsection{Punto a}
	
	\textcolor{Green4}{\textbf{\emph{Si calcolino i seguenti numeri:}}}
	\begin{enumerate}
		\item[i.] \textcolor{Green4}{\textbf{\emph{Il modulo $|z|$ di $z$.}}}
		
		\item[ii.] \textcolor{Green4}{\textbf{\emph{Il coniugato $\overline{z}$ di $z$.}}}
		
		\item[iii.] \textcolor{Green4}{\textbf{\emph{Il numero complesso $\frac{1}{z}$.}}}
	\end{enumerate}
	Il modulo $|z|$ di $z$ si calcola ricordando la formula:
	\begin{equation*}
		|z| = \sqrt{a^{2} + b^{2}}
	\end{equation*}
	Quindi, sostituendo il numero complesso dato:
	\begin{equation*}
		|z| = \sqrt{\left(-\dfrac{\sqrt{2}}{2}\right)^{2} + \left(-\dfrac{\sqrt{2}}{2}\right)^{2}} = \sqrt{\dfrac{1}{2} + \dfrac{1}{2}} = 1
	\end{equation*}
	Il coniugato di $z$ corrisponde alla parte immaginaria $i$ cambiata di segno:
	\begin{equation*}
		\overline{z} = -\frac{\sqrt{2}}{2} + \frac{\sqrt{2}}{2}i
	\end{equation*}
	Il numero complesso $\frac{1}{z}$ corrisponde a $z^{-1}$, calcolabile con la formula:
	\begin{equation*}
		z^{-1} = \dfrac{\overline{z}}{|z|^{2}}
	\end{equation*}
	Quindi, sostituendo il numero complesso dato:
	\begin{equation*}
		\dfrac{1}{z} = \dfrac{-\dfrac{\sqrt{2}}{2} + \dfrac{\sqrt{2}}{2}i}{1^{2}} = -\dfrac{\sqrt{2}}{2} + \dfrac{\sqrt{2}}{2}i
	\end{equation*}\newpage
	
	\subsubsection{Punto b}
	
	\textcolor{Green4}{\textbf{\emph{Si calcoli il prodotto $zw$ dove $w = 5\left(\cos\left(\frac{3\pi}{4}\right) + i \sin\left(\frac{3\pi}{4}\right)\right)$.}}}\newline
	
	\noindent
	Si converte $w$ da forma trigonometrica a forma algebrica:
	\begin{equation*}
		\begin{cases}
			a = 5 \cdot \cos\left(\frac{3\pi}{4}\right) \\
			b = 5 \cdot \sin\left(\frac{3\pi}{4}\right)
		\end{cases}
		\longrightarrow
		\begin{cases}
			a = -\frac{5\sqrt{2}}{2} \\
			b = \frac{5\sqrt{2}}{2}
		\end{cases}
		\longrightarrow
		w = -\frac{5\sqrt{2}}{2} + \frac{5\sqrt{2}}{2} i
	\end{equation*}
	A questo punto si procede con la moltiplicazione tra due forme algebriche, seguendo la seguente regola:
	\begin{equation*}
		\left(a+bi\right) \cdot \left(c+di\right) = a\left(c+di\right) + bi\left(c+di\right)
	\end{equation*}
	Andando a sostituire i valori:
	\begin{gather*}
		\begin{array}{cl}
			=& -\dfrac{\sqrt{2}}{2}\left(-\dfrac{5\sqrt{2}}{2} + \dfrac{5\sqrt{2}}{2} i\right) -\dfrac{\sqrt{2}}{2}i\left(-\dfrac{5\sqrt{2}}{2} + \dfrac{5\sqrt{2}}{2} i\right) \\ [1.5em]
			=& \dfrac{5}{2} -\dfrac{5}{2}i + \dfrac{5}{2}i - \dfrac{5}{2}i^{2} \\ [1.5em]
			=& \dfrac{5}{2} + \dfrac{5}{2} \\ [1.5em]
			=& \dfrac{10}{2} \\ [1.5em]
			=& 5
		\end{array}
	\end{gather*}\newpage
	
	\subsubsection{Punto c}
	
	\textcolor{Green4}{\textbf{\emph{Si calcolino tutte le radici quadrate di $z$.}}}\newline
	
	\noindent
	Per calcolare le radici quadrate di un numero complesso, si segue la seguente formula:
	\begin{equation*}
		\sqrt[n]{z} = \sqrt[n]{d}\left[\cos\left(\dfrac{a+2\pi k}{n}\right) + i\sin\left(\dfrac{a + 2 \pi k}{n}\right)\right]
	\end{equation*}
	Prima di passare al calcolo, si trasforma il numero complesso da forma algebrica a forma trigonometrica:
	\begin{equation*}
		\begin{cases}
			d = |z| = 1 \\
			\theta = \begin{cases}
				\frac{\pi}{2}	& a=0, b>0 \\
				\frac{3\pi}{2}	& a=0, b<0 \\
				\text{non definito}	& a=0, b=0 \\
				\arctan\left(\frac{b}{a}\right)	& a>0, b\ge0 \\
				\arctan\left(\frac{b}{a}\right) + 2\pi	& a>0, b<0 \\
				\arctan\left(\frac{b}{a}\right) + \pi	& a<0, b \: \text{qualsiasi} \\
			\end{cases} = \arctan\left(1\right) + \pi = \dfrac{5\pi}{4}
		\end{cases}
	\end{equation*}
	Quindi, la forma trigonometrica corrispondente:
	\begin{equation*}
		z = \cos\left(\dfrac{5\pi}{4}\right) + i\sin\left(\dfrac{5\pi}{4}\right)
	\end{equation*}
	Si calcola $\sqrt[2]{z}$:
	\begin{equation*}
		\begin{array}{rcl}
			z_{0} &=& \sqrt[2]{1}\left[\cos\left(\dfrac{\dfrac{5\pi}{4} + 2 \cdot 0 \cdot \pi}{2}\right) + i \sin\left(\dfrac{\dfrac{5\pi}{4} + 2 \cdot 0 \cdot \pi}{2}\right)\right] \\ [2em]
			&=& \cos\left(\dfrac{5\pi}{8}\right) + i \sin\left(\dfrac{5\pi}{8}\right) \\ [2em]
			%
			z_{1} &=& \sqrt[2]{1}\left[\cos\left(\dfrac{\dfrac{5\pi}{4} + 2 \cdot 1 \cdot \pi}{2}\right) + i \sin\left(\dfrac{\dfrac{5\pi}{4} + 2 \cdot 1 \cdot \pi}{2}\right)\right] \\ [2em]
			&=& \cos\left(\dfrac{13\pi}{8}\right) + i \sin\left(\dfrac{13\pi}{8}\right) \\
		\end{array}
	\end{equation*}
\end{document}