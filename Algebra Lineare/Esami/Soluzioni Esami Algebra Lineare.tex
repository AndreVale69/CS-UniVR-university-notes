\documentclass[a4paper]{article}
\usepackage[T1]{fontenc}			% pacchetto per \chapter
\usepackage[italian]{babel}
\usepackage[italian]{isodate}  		% formato delle date in italiano
\usepackage{graphicx}				% gestione delle immagini
\usepackage{amsfonts}
\usepackage{booktabs}				% tabelle di qualità superiore
\usepackage{amsmath}				% pacchetto matematica
\usepackage{mathtools}				% per sottolineare sotto le equazioni
\usepackage{stmaryrd} 				% per '\llbracket' e '\rrbracket'
\usepackage{amsthm}					% teoremi migliorati
\usepackage{enumitem}				% gestione delle liste
\usepackage{pifont}					% pacchetto con elenchi carini
\usepackage{enumitem}				% pacchetto per elenchi con lettere dell'alfabeto
\usepackage{cancel}					% per cancellare delle espressioni matematiche



\usepackage[x11names]{xcolor}		% pacchetto colori RGB
% Link ipertestuali per l'indice
\usepackage{xcolor}
\usepackage[linkcolor=black, citecolor=blue, urlcolor=cyan]{hyperref}
\hypersetup{
	colorlinks=true
}

\usepackage{tikz}
\newcommand{\MyTikzmark}[2]{%
	\tikz[overlay,remember picture,baseline] \node [anchor=base] (#1) {#2};%
}
\newcommand{\DrawVLine}[3][]{%
	\begin{tikzpicture}[overlay,remember picture]
		\draw[shorten <=0.3ex, #1] (#2.north) -- (#3.south);
	\end{tikzpicture}
}
\newcommand{\DrawHLine}[3][]{%
	\begin{tikzpicture}[overlay,remember picture]
		\draw[shorten <=0.2em, #1] (#2.west) -- (#3.east);
	\end{tikzpicture}
}


%\usepackage{showframe}				% visualizzazione bordi
%\usepackage{showkeys}				% visualizzazione etichetta

\newtheorem{theorem}{\textcolor{Red3}{\underline{Teorema}}}
\newtheorem{lemma}{Lemma}
\renewcommand{\qedsymbol}{QED}
\newcommand{\exec}[1]{\llbracket #1\:\rrbracket}
\newcommand{\dquotes}[1]{``#1''}
\newcommand{\longline}{\noindent\rule{\textwidth}{0.4pt}}
\newcommand{\circledtext}[1]{\raisebox{.5pt}{\textcircled{\raisebox{-.9pt}{#1}}}}

\newenvironment{rowequmat}[1]{\left(\array{@{}#1@{}}}{\endarray\right)}
\newenvironment{rowequmatbra}[1]{\left[\array{@{}#1@{}}}{\endarray\right]}

\begin{document}
	\author{VR443470}
	\title{Soluzioni Esami di Algebra Lineare}
	\date{\printdayoff\today}
	\maketitle
	
	\newpage
	
	% indice
	\tableofcontents
	
	\newpage
	
	\section{Esame del 20/06/2022}
	
	\subsection{Esercizio 1}
	
	(\textbf{6 punti}) Si consideri la seguente matrice:
	\begin{equation*}
		A = \begin{pmatrix}
			1	& 0		& 0		& 2 \\
			-2	& -1	& 1		& -5 \\
			1	& -a	& 2+a	& 2-a \\
			1+a	& 0		& 2		& \left(1+a\right)\left(-1+a\right)
		\end{pmatrix}
	\end{equation*}
	
	\longline
	
	\subsubsection{Punto a}
	
	\textcolor{Green4}{\emph{\textbf{Si calcoli, al variare di $a \in \mathbb{R}$, il rango $\mathrm{rk} \: A$ di $A$.}}}\newline
	
	\noindent
	Si applica l'eliminazione di Gauss per ottenere il rango della matrice:
	\begin{gather*}
		\begin{pmatrix}
			1	& 0		& 0		& 2 \\
			-2	& -1	& 1		& -5 \\
			1	& -a	& 2+a	& 2-a \\
			1+a	& 0		& 2		& \left(1+a\right)\left(-1+a\right)
		\end{pmatrix}
		\xrightarrow{E_{1,2}\left(2\right)}
		\begin{pmatrix}
			1	& 0		& 0		& 2 \\
			0	& -1	& 1		& -1 \\
			1	& -a	& 2+a	& 2-a \\
			1+a	& 0		& 2		& \left(1+a\right)\left(-1+a\right)
		\end{pmatrix} \\
		\\
		\xrightarrow{E_{1,3}\left(-1\right)}
		\begin{pmatrix}
			1	& 0		& 0		& 2 \\
			0	& -1	& 1		& -1 \\
			0	& -a	& 2+a	& -a \\
			1+a	& 0		& 2		& \left(1+a\right)\left(-1+a\right)
		\end{pmatrix}
		\xrightarrow{E_{1,4}\left(-a-1\right)}
		\begin{pmatrix}
			1	& 0		& 0		& 2 \\
			0	& -1	& 1		& -1 \\
			0	& -a	& 2+a	& -a \\
			0	& 0		& 2		& a^{2} -2a -3
		\end{pmatrix} \\
		\\
		\xrightarrow{E_{2,3}\left(-a\right)}
		\begin{pmatrix}
			1	& 0		& 0		& 2 \\
			0	& -1	& 1		& -1 \\
			0	& 0		& 2		& 0 \\
			0	& 0		& 2		& a^{2} -2a -3
		\end{pmatrix}
		\xrightarrow{E_{3,4}\left(-1\right)}
		\begin{pmatrix}
			1	& 0		& 0		& 2 \\
			0	& -1	& 1		& -1 \\
			0	& 0		& 2		& 0 \\
			0	& 0		& 0		& a^{2} -2a -3
		\end{pmatrix}
	\end{gather*}
	Il rango della matrice è influenzato solo dall'espressione $a^{2} - 2a - 3$. Quindi, nel caso di:
	\begin{equation*}
		\mathrm{rk}\left(A\right) =
		\begin{cases}
			3	& a^{2} - 2a -3 = 0\\
			4	& a^{2} - 2a -3 \ne 0
		\end{cases}
	\end{equation*}\newpage
	
	\subsubsection{Punto b}
	
	\textcolor{Green4}{\emph{\textbf{Si calcoli il determinante $\det\left(A\right)$ di $A$.}}}\newline
	
	\noindent
	Per velocizzare i calcoli, si utilizza il metodo di Gauss Jordan\footnote{Approfondimento: \href{https://www.youmath.it/forum/algebra-lineare/52861-determinante-con-eliminazione-di-gauss.html}{YouMath}}. Per il calcolo del determinante, si ricordano le seguenti regole:
	\begin{itemize}
		\item Lo scambio di una riga cambia il segno del determinante (quindi lo moltiplica per $-1$);
		
		\item La moltiplicazione di una riga per uno scalare non nullo, provoca la moltiplicazione dell'inverso di esso al determinante della matrice. Quindi, data l'operazione $E_{i}\left(\alpha\right)$, il determinante viene moltiplicato per $\frac{1}{\alpha}$;
		
		\item La moltiplicazione di una riga per uno scalare e la successiva somma, non cambia il determinante.
	\end{itemize}
	Quindi, si ottiene il determinante della matrice ridotta $A'$ moltiplicando la diagonale:
	\begin{gather*}
		A' = \begin{pmatrix}
			1	& 0		& 0		& 2 \\
			0	& -1	& 1		& -1 \\
			0	& 0		& 2		& 0 \\
			0	& 0		& 0		& a^{2} -a -2
		\end{pmatrix} \\
		\\
		\det\left(A'\right) = 1 \cdot \left(-1\right) \cdot 2 \cdot \left(a^{2}-2a-3\right) = -2a^{2} + 4a +6
	\end{gather*}
	E controllando le operazioni eseguite al punto precedente, è possibile notare che non è stato effettuato nessuno scambio di righe e nessuna moltplicazione + somma. Quindi il determinante è lo stesso:
	\begin{equation*}
		\det\left(A\right) = \det\left(A'\right) = -2a^{2} + 4a +6
	\end{equation*}\newpage
	
	\subsubsection{Punto c}
	
	\textcolor{Green4}{\textbf{\emph{Si determino i valori di $a \in \mathbb{R}$ tali che $A$ possiede una inversa.}}}\newline
	
	\noindent
	La matrice $A$ possiede un'inversa se e solo se il suo determinante è diverso da zero. Quindi, risolvendo l'equazione del determinante, si può capire per quali valori di $a$, la matrice $A$ ammette inversa:
	\begin{equation*}
		-2a^{2} + 4a + 6 = 0\longrightarrow \dfrac{-b \pm \sqrt{b^{2} - 4ac}}{2a} \longrightarrow
		\dfrac{-4 \pm \sqrt{16 - 4 \cdot -2 \cdot 6}}{2 \cdot -2} = \dfrac{-4 \pm 8}{-4}
	\end{equation*}
	Le soluzioni che azzerano l'equazione sono:
	\begin{gather*}
		a_{0} = -1 \\
		a_{1} = 3
	\end{gather*}
	Si conclude dicendo che:
	\begin{equation*}
		\det\left(A\right) = \begin{cases}
			0										& a = -1 \lor a = 3 \\
			\mathbb{R} \setminus \left\{0\right\}	& \text{altrimenti}
		\end{cases}
	\end{equation*}
	La matrice $A$ possiede una inversa se e solo se il determinante è diverso da zero. Il determinante è diverso da zero se e solo se $a$ è diverso da $-1$ e da $3$. Quindi, $A$ possiede una inversa per i valori $a \in \mathbb{R} \setminus \left\{-1, 3\right\}$.\newpage

	\subsection{Esercizio 2}
	
	(\textbf{12 punti}) Si consideri la seguente matrice:
	\begin{equation*}
		B = \begin{pmatrix}
			-1	& 0		& 0		& 0 \\
			0	& 2		& -3	& 0 \\
			0	& 0		& -1	& 0 \\
			0	& 0		& 5		& 2
		\end{pmatrix}
	\end{equation*}
	
	\longline
	
	\subsubsection{Punto a}

	\textcolor{Green4}{\textbf{\emph{Si calcolino tutti gli autovalori di $B$ su $\mathbb{R}$ e si trovino delle basi dei loro autospazi.}}}\newline

	\noindent
	Si calcola il polinomio caratteristico associato alla matrice $B$:
	\begin{equation*}
		\begin{array}{lll}
			p_{B}\left(\lambda\right) &=& \det\left(B - \lambda \mathrm{Id}_{4}\right) \\ [.7em]
			&=& \det\left[
				\begin{pmatrix}
					-1	& 0		& 0		& 0 \\
					0	& 2		& -3	& 0 \\
					0	& 0		& -1	& 0 \\
					0	& 0		& 5		& 2
				\end{pmatrix} -
				\lambda \begin{pmatrix}
					1	& 0		& 0		& 0 \\
					0	& 1		& 0		& 0 \\
					0	& 0		& 1		& 0 \\
					0	& 0		& 0		& 1
				\end{pmatrix}
			\right] \\ [2.5em]
			&=& \det\left[
				\begin{pmatrix}
					-1	& 0		& 0		& 0 \\
					0	& 2		& -3	& 0 \\
					0	& 0		& -1	& 0 \\
					0	& 0		& 5		& 2
				\end{pmatrix} -
				\begin{pmatrix}
					\lambda	& 0			& 0			& 0 \\
					0		& \lambda	& 0			& 0 \\
					0		& 0			& \lambda	& 0 \\
					0		& 0			& 0			& \lambda
				\end{pmatrix}
			\right] \\ [2.5em]
			&=& \det \begin{pmatrix}
				-1-\lambda	& 0			& 0				& 0 \\
					0		& 2-\lambda	& -3			& 0 \\
					0		& 0			& -1-\lambda	& 0 \\
					0		& 0			& 5				& 2-\lambda
			\end{pmatrix} \\ [2.5em]
		\end{array}
	\end{equation*}
	Si utilizzano gli sviluppi di Laplace per calcolare il determinante della matrice. Si sceglie lo sviluppo per righe partendo dalla riga $4$ e rimando sulla colonna $4$:
	\begin{gather*}
		\begin{rowequmat}{cccc}
				-1-\lambda	& 0			& 0				& 0 \\
					0		& 2-\lambda	& -3			& 0 \\
					0		& 0			& -1-\lambda	& 0 \\
					0		& 0			& 5				& 2-\lambda
		\end{rowequmat} \\
		\\
		\begin{array}{lll}
			b_{4,4} = 2-\lambda &\longrightarrow& \left(-1\right)^{4+4} \cdot \left(2-\lambda\right) \cdot \det
			\begin{rowequmat}{ccc}
				-1-\lambda	& 0			& 0			 \\
					0		& 2-\lambda	& -3		 \\
					0		& 0			& -1-\lambda 
			\end{rowequmat} \\[2.5em]
			&\longrightarrow& 1 \cdot \left(2-\lambda\right) \cdot \left(-1-\lambda\right) \cdot \left(2-\lambda\right) \cdot \left(-1-\lambda\right) \\ [1em]
			b_{3,4} = 0 		&\longrightarrow& \left(-1\right)^{3+4} \cdot \left(0\right) \cdot \det
			\begin{rowequmat}{ccc}
				-1-\lambda	& 0			& 0			 \\
					0		& 2-\lambda	& -3		 \\
					0		& 0			& 5 		 
			\end{rowequmat} = 0 \\[2.5em]
			b_{2,4} = 0 		&\longrightarrow& \left(-1\right)^{2+4} \cdot \left(0\right) \cdot \det
			\begin{rowequmat}{ccc}
				-1-\lambda	& 0			& 0				\\
					0		& 0			& -1-\lambda	\\
					0		& 0			& 5				
			\end{rowequmat} = 0 \\[2.5em]
			b_{1,4} = 0 		&\longrightarrow& \left(-1\right)^{1+4} \cdot \left(0\right) \cdot \det
			\begin{rowequmat}{ccc}
					0		& 2-\lambda	& -3			\\
					0		& 0			& -1-\lambda	\\
					0		& 0			& 5				
			\end{rowequmat} = 0 \\[2.5em]
		\end{array}
	\end{gather*}
	Quindi, il determinante della matrice è:
	\begin{equation*}
		p_{B}\left(\lambda\right) = \det\left(B - \lambda \mathrm{Id}_{4}\right) = \left(-1-\lambda\right) \cdot \left(2-\lambda\right) \cdot \left(-1-\lambda\right) \cdot \left(2-\lambda\right)
	\end{equation*}
	Per continuare il calcolo del polinomio caratteristico, si cercano gli zeri, ovvero tutti quei valori tale che $p_{B}\left(\lambda\right) = 0$. Banalmente, i valori sono:
	\begin{equation*}
		\begin{array}{lll}
			\lambda_{1} = -1 &\longrightarrow& \left(-1-\left(-1\right)\right) \cdot \left(2-\left(-1\right)\right) \cdot \left(-1-\left(-1\right)\right) \cdot \left(2-\left(-1\right)\right) = 0 \\[.5em]
			\lambda_{2} = 2  &\longrightarrow& \left(-1-\left(2\right)\right) \cdot \left(2-\left(2\right)\right) \cdot \left(-1-\left(2\right)\right) \cdot \left(2-\left(2\right)\right) = 0
		\end{array}
	\end{equation*}
	Si conclude affermando che gli autovalori di $B$ sono $\lambda_{1} = -1$ e $\lambda_{2} = 2$.\newpage

	\noindent
	Per trovare delle basi dei loro autospazi, è necessario sostituire ogni $\lambda$ trovato, nella matrice calcolata precedentemente. Quindi, il primo autospazio con il primo autovalore $\lambda_{1} = -1$:
	\begin{equation*}
		B - \lambda_{1} \mathrm{Id}_{4} = B - \left(-1\right) \mathrm{Id}_{4} =
		\begin{rowequmat}{cccc}
				0	& 0		& 0		& 0 \\
				0	& 3		& -3	& 0 \\
				0	& 0		& 0		& 0 \\
				0	& 0		& 5		& 3
		\end{rowequmat}
	\end{equation*}
	Si cerca la forma ridotta della matrice, eseguendo l'eliminazione di Gauss:
	\begin{equation*}
		\begin{rowequmat}{cccc}
			0	& 0		& 0		& 0 \\
			0	& 3		& -3	& 0 \\
			0	& 0		& 0		& 0 \\
			0	& 0		& 5		& 3
		\end{rowequmat} \xlongrightarrow[E_{4,2}]{E_{2,1}}
		\begin{rowequmat}{cccc}
			0	& 3		& -3	& 0 \\
			0	& 0		& 5		& 3 \\
			0	& 0		& 0		& 0 \\
			0	& 0		& 0		& 0 
		\end{rowequmat}
	\end{equation*}
	Si ottiene il sistema lineare:
	\begin{equation*}
		\begin{cases}
			3x_{2} -3x_{3} = 0 \\
			5x_{3} +3x_{4} = 0	
		\end{cases}
		\longrightarrow
		\begin{cases}
			x_{3} = x_{2} \\
			x_{4} = -\frac{5}{3}x_{2}
		\end{cases}
	\end{equation*}
	Quindi, il sistema generale è:
	\begin{equation*}
		\begin{cases}
			x_{1} = x_{1} \\
			x_{2} = x_{2} \\
			x_{3} = x_{2} \\
			x_{4} = -\frac{5}{3}x_{2}
		\end{cases}
	\end{equation*}
	L'autospazio generale è:
	\begin{equation*}
		\left\{
			\left. \begin{pmatrix}
				x_{1} \\
				x_{2} \\
				x_{2} \\
				-\frac{5}{3}x_{2}
			\end{pmatrix} \: \right| \:
			x_{1}, x_{2} \in \mathbb{R}
		\right\} = \left\{x_{1}
		\begin{pmatrix}
			1 \\
			0 \\
			0 \\
			0
		\end{pmatrix} + x_{2}
		\begin{rowequmat}{c}
			0 \\ [.3em]
			1 \\ [.3em]
			1 \\ [.3em]
			-\frac{5}{3}
		\end{rowequmat}
		\right\}
	\end{equation*}
	Dove il seguente insieme è una base:
	\begin{equation*}
		\left\{
			\begin{pmatrix}
				1 \\
				0 \\
				0 \\
				0
			\end{pmatrix} ,
			\begin{rowequmat}{c}
				0 \\ [.3em]
				1 \\ [.3em]
				1 \\ [.3em]
				-\frac{5}{3}
			\end{rowequmat}
		\right\}
	\end{equation*}\newpage

	\noindent
	Il secondo autospazio con il secondo autovalore $\lambda = 2$:
	\begin{equation*}
		B - \lambda_{2} \mathrm{Id}_{4} = B - \left(2\right) \mathrm{Id}_{4} =
		\begin{rowequmat}{cccc}
				-3	& 0		& 0		& 0 \\
				0	& 0		& -3	& 0 \\
				0	& 0		& -3	& 0 \\
				0	& 0		& 5		& 0
		\end{rowequmat}
	\end{equation*}
	Si cerca la forma ridotta della matrice, eseguendo l'eliminazione di Gauss:
	\begin{equation*}
		\begin{rowequmat}{cccc}
			-3	& 0		& 0		& 0 \\
			0	& 0		& -3	& 0 \\
			0	& 0		& -3	& 0 \\
			0	& 0		& 5		& 0
		\end{rowequmat}
		\xlongrightarrow[E_{4,2}\left(\frac{3}{5}\right)]{E_{2,3}\left(-1\right)}
		\begin{rowequmat}{cccc}
			-3	& 0		& 0		& 0 \\
			0	& 0		& 0		& 0 \\
			0	& 0		& 0		& 0 \\
			0	& 0		& 5		& 0
		\end{rowequmat}
		\xlongrightarrow[E_{1}\left(-\frac{1}{3}\right)]{E_{4,2}, \: E_{2}\left(\frac{1}{5}\right)}
		\begin{rowequmat}{cccc}
			1	& 0		& 0		& 0 \\
			0	& 0		& 1		& 0 \\
			0	& 0		& 0		& 0 \\
			0	& 0		& 0		& 0
		\end{rowequmat}
	\end{equation*}
	Si ottiene il sistema lineare:
	\begin{equation*}
		\begin{cases}
			x_{1} = 0 \\
			x_{3} = 0
		\end{cases}
	\end{equation*}
	Quindi, il sistema generale è:
	\begin{equation*}
		\begin{cases}
			x_{1} = 0		\\
			x_{2} = x_{2} 	\\
			x_{3} = 0		\\
			x_{4} = x_{4}
		\end{cases}
	\end{equation*}
	L'autospazio generale è:
	\begin{equation*}
		\left\{
			\left. \begin{pmatrix}
				0		\\
				x_{2} 	\\
				0		\\
				x_{4}
			\end{pmatrix} \: \right| \:
			x_{2}, x_{4} \in \mathbb{R}
		\right\} =
		\left\{
			x_{2} \begin{pmatrix}
				0 \\
				1 \\
				0 \\
				0
			\end{pmatrix}
			+
			x_{4} \begin{pmatrix}
				0 \\
				0 \\
				0 \\
				1
			\end{pmatrix}
		\right\}
	\end{equation*}
	Dove il seguente insieme è una base:
	\begin{equation*}
		\left\{
			\begin{pmatrix}
				0 \\
				1 \\
				0 \\
				0
			\end{pmatrix},
			\begin{pmatrix}
				0 \\
				0 \\
				0 \\
				1
			\end{pmatrix}
		\right\}
	\end{equation*}\newpage

	\subsubsection{Punto b}

	\textcolor{Green4}{\textbf{\emph{Si verifichi che la matrice $B$ è diagonalizzabile e si scrivano la matrice diagonale $D$ e la matrice invertibile $S$ tali che $B = SDS^{-1}$.}}}\newline

	\noindent
	Una matrice $B$ è diagonalizzabile se rispetta due condizioni:
	\begin{enumerate}
		\item La somma delle molteplicità algebriche degli autovalori della matrice è uguale all'ordine della matrice;
		\item La molteplicità geometrica di ciascun autovalore coincide con la relativa molteplicità algebrica.
	\end{enumerate}
	Gli autovalori di $B$ sono $\lambda_{1}=-1$ e $\lambda_{2} = 2$. Ma dato che le soluzioni derivano da:
	\begin{equation*}
		p_{B}\left(\lambda\right) = \det\left(B - \lambda \mathrm{Id}_{4}\right) = \left(-1-\lambda\right) \cdot \left(2-\lambda\right) \cdot \left(-1-\lambda\right) \cdot \left(2-\lambda\right)
	\end{equation*}
	È immediato vedere come ogni soluzione trovata annulli due volte il polinomio caratteristico. Quindi le molteplicità algebriche sono:
	\begin{gather*}
		m_{1} = 2 \\
		m_{2} = 2
	\end{gather*}
	La prima condizione è soddisfatta. La seconda è la verifica della molteplicità geometrica. Quest'ultima è possibile verificarla con la seguente formula:
	\begin{equation*}
		m_{g}\left(\lambda\right) = n - \mathrm{rk}\left(A - \lambda\mathrm{Id}_{n}\right)
	\end{equation*}
	Dunque, si calcola il rango di tutte le matrici trovate sostituendo gli autovalori ottenuti:
	\begin{equation*}
		\begin{array}{lll}
			m_{1}\left(-1\right) &=& 4 - \mathrm{rk}\left(A - \left(-1\right)\mathrm{Id}_{4}\right) = 4 - 2 = 2 \\
			m_{2}\left( 2\right) &=& 4 - \mathrm{rk}\left(A - 2\mathrm{Id}_{4}\right) = 4 - 2 = 2
		\end{array}
	\end{equation*}
	Ciascuna molteplicità geometrica corrisponde con la relativa molteplicità algebrica. Questo conferma che la matrice $B$ è diagonalizzabile.\newline

	\noindent
	La matrice diagonale $D$ ha gli autovalori di $B$ nella diagonale principale:
	\begin{equation*}
		D = \begin{pmatrix}
			-1 & 0 & 0 & 0 \\
			0 & -1 & 0 & 0 \\
			0 & 0 & 2 & 0 \\
			0 & 0 & 0 & 2 
		\end{pmatrix}
	\end{equation*}
	La matrice $S$ invece è composta dalle basi trovate, ovvero:
	\begin{equation*}
		S = \begin{rowequmat}{cccc}
			1 & 0				& 0 & 0 \\ [.3em]
			0 & 1				& 1 & 0 \\ [.3em]
			0 & 1				& 0 & 0 \\ [.3em]
			0 & -\frac{5}{3}	& 0 & 1
		\end{rowequmat}
	\end{equation*}\newpage
	
	\noindent
	La matrice inversa si trova affiancando a destra la matrice identità ed eseguendo EG:
	\begin{gather*}
		\left(S \: | \: \mathrm{Id_{4}}\right) = 
		\begin{rowequmat}{cccc|cccc}
			1 & 0				& 0 & 0 & 1 & 0 & 0 & 0 \\ [.3em]
			0 & 1				& 1 & 0 & 0 & 1 & 0 & 0 \\ [.3em]
			0 & 1				& 0 & 0 & 0 & 0 & 1 & 0 \\ [.3em]
			0 & -\frac{5}{3}	& 0 & 1 & 0 & 0 & 0 & 1
		\end{rowequmat}
		\xlongrightarrow[E_{2,3}\left(-1\right)]{E_{3,2}}
		\begin{rowequmat}{cccc|cccc}
			1 & 0				& 0 & 0 & 1 & 0 & 0 & 0 \\ [.3em]
			0 & 1				& 0 & 0 & 0 & 0 & 1 & 0 \\ [.3em]
			0 & 0				& 1 & 0 & 0 & 1 & -1 & 0 \\ [.3em]
			0 & -\frac{5}{3}	& 0 & 1 & 0 & 0 & 0 & 1
		\end{rowequmat} \\
		\\
		\xlongrightarrow{E_{2,4}\left(\frac{5}{3}\right)}
		\begin{rowequmat}{cccc|cccc}
			1 & 0	& 0 & 0 & 1 & 0 & 0 & 0 \\ [.3em]
			0 & 1	& 0 & 0 & 0 & 0 & 1 & 0 \\ [.3em]
			0 & 0	& 1 & 0 & 0 & 1 & -1 & 0 \\ [.3em]
			0 & 0	& 0 & 1 & 0 & 0 & \frac{5}{3} & 1
		\end{rowequmat}\\
		\\
		S^{-1} = 
		\begin{rowequmat}{cccc}
			1 & 0 & 0 & 0 \\ [.3em]
			0 & 0 & 1 & 0 \\ [.3em]
			0 & 1 & -1 & 0 \\ [.3em]
			0 & 0 & \frac{5}{3} & 1
		\end{rowequmat}
	\end{gather*}
	Si verifica la correttezza:
	\begin{equation*}
		\begin{array}{lll}
			B = SDS^{-1} &=&
			\begin{rowequmat}{cccc}
				1 & 0				& 0 & 0 \\ [.3em]
				0 & 1				& 1 & 0 \\ [.3em]
				0 & 1				& 0 & 0 \\ [.3em]
				0 & -\frac{5}{3}	& 0 & 1
			\end{rowequmat}
			\begin{rowequmat}{cccc}
				-1 & 0 & 0 & 0 \\ [.3em]
				0 & -1 & 0 & 0 \\ [.3em]
				0 & 0 & 2 & 0 \\ [.3em]
				0 & 0 & 0 & 2 
			\end{rowequmat}
			\begin{rowequmat}{cccc}
				1 & 0 & 0 & 0 \\ [.3em]
				0 & 0 & 1 & 0 \\ [.3em]
				0 & 1 & -1 & 0 \\ [.3em]
				0 & 0 & \frac{5}{3} & 1
			\end{rowequmat} \\[3em]
			%
			&=&
			\begin{rowequmat}{cccc}
				1 & 0				& 0 & 0 \\ [.3em]
				0 & 1				& 1 & 0 \\ [.3em]
				0 & 1				& 0 & 0 \\ [.3em]
				0 & -\frac{5}{3}	& 0 & 1
			\end{rowequmat}
			\begin{rowequmat}{cccc}
				-1 & 0 & 0 & 0 \\ [.3em]
				0 & 0 & -1 & 0 \\ [.3em]
				0 & 2 & -2 & 0 \\ [.3em]
				0 & 0 & \frac{10}{3} & 2
			\end{rowequmat} \\[3em]
			%
			&=&
			\begin{rowequmat}{cccc}
				-1 & 0 & 0 & 0 \\ [.3em]
				0 & 2 & -3 & 0 \\ [.3em]
				0 & 0 & -1 & 0 \\ [.3em]
				0 & 0 & 5 & 2
			\end{rowequmat}
		\end{array}
	\end{equation*}\newpage

	\subsubsection{Punto c}

	\textcolor{Green4}{\textbf{\emph{Utilizzando la diagonalizzazione, si calcoli il prodotto $B^{5}$.}}}\newline

	\noindent
	Si calcola:
	\begin{equation*}
		B^{5} = \left(SDS^{-1}\right)^{5} = S D^{5} S^{-1}
	\end{equation*}
	Quindi, le matrici mutano in:
	\begin{equation*}
		\begin{array}{lll}
			B = \left(SDS^{-1}\right)^{5} &=&
			\begin{rowequmat}{cccc}
				1 & 0				& 0 & 0 \\ [.3em]
				0 & 1				& 1 & 0 \\ [.3em]
				0 & 1				& 0 & 0 \\ [.3em]
				0 & -\frac{5}{3}	& 0 & 1
			\end{rowequmat}
			\begin{rowequmat}{cccc}
				-1 & 0 & 0 & 0 \\ [.3em]
				0 & -1 & 0 & 0 \\ [.3em]
				0 & 0 & 32 & 0 \\ [.3em]
				0 & 0 & 0 & 32 
			\end{rowequmat}
			\begin{rowequmat}{cccc}
				1 & 0 & 0 & 0 \\ [.3em]
				0 & 0 & 1 & 0 \\ [.3em]
				0 & 1 & -1 & 0 \\ [.3em]
				0 & 0 & \frac{5}{3} & 1
			\end{rowequmat} \\[3em]
			%
			&=&
			\begin{rowequmat}{cccc}
				1 & 0				& 0 & 0 \\ [.3em]
				0 & 1				& 1 & 0 \\ [.3em]
				0 & 1				& 0 & 0 \\ [.3em]
				0 & -\frac{5}{3}	& 0 & 1
			\end{rowequmat}
			\begin{rowequmat}{cccc}
				-1 & 0 & 0 & 0 \\ [.3em]
				0 & 0 & -1 & 0 \\ [.3em]
				0 & 32 & -32 & 0 \\ [.3em]
				0 & 0 & \frac{160}{3} & 32
			\end{rowequmat} \\[3em]
			%
			&=&
			\begin{rowequmat}{cccc}
				-1 & 0 & 0 & 0 \\ [.3em]
				0 & 32 & -33 & 0 \\ [.3em]
				0 & 0 & -1 & 0 \\ [.3em]
				0 & 0 & 55 & 32
			\end{rowequmat}
		\end{array}
	\end{equation*}\newpage

	\subsection{Esercizio 3}

	(\textbf{8 punti}) Si consideri la seguente matrice:
	\begin{equation*}
		M = \begin{pmatrix}
			i	& 2i+1	\\
			-1	& 2		\\
			-i	& 2i
		\end{pmatrix}
	\end{equation*}
	
	\longline

	\subsubsection{Punto a}

	\textcolor{Green4}{\textbf{\emph{Si calcoli la $H$-trasposta $M^{H}$ di $M$.}}}\newline

	\noindent
	La matrice $H$-trasposta di $M$, si ottiene invertendo le righe con le colonne ed eseguendo l'operazione di coniugazione (cambiare di segno), la quale influisce solo sui numeri immaginari:
	\begin{equation*}
		\begin{array}{lll}
			M &=& \begin{pmatrix}
				i	& 2i+1	\\
				-1	& 2		\\
				-i	& 2i
			\end{pmatrix} \\ [2	em]
			M^{T} &=& \begin{pmatrix}
				i 		& -1	& -i \\
				2i+1	& 2		& 2i
			\end{pmatrix} \\ [2em]
			M^{H} &=& \begin{pmatrix}
				-i 		& -1	& i \\
				-2i+1	& 2		& -2i
			\end{pmatrix}
		\end{array}
	\end{equation*}

	\longline

	\subsubsection{Punto b}

	\textcolor{Green4}{\textbf{\emph{Si determinino una base di $C\left(M\right)$ e una base di $N\left(M^{H}\right)$ su $\mathbb{C}$.}}}\newline

	\noindent
	Si applica l'eliminazione di Gauss per ottenere una forma ridotta:
	\begin{equation*}
		M' =
		\begin{pmatrix}
			i	& 2i+1	\\
			-1	& 2		\\
			-i	& 2i
		\end{pmatrix}
		\xlongrightarrow[E_{1,2}\left(\frac{1}{i}\right)]{E_{1,3}\left(1\right)}
		\begin{rowequmat}{cc}
			i	& 2i+1 \\ [.3em]
			0	& 4+\frac{1}{i} \\ [.3em]
			0	& 4i+1
		\end{rowequmat}
		\xlongrightarrow{E_{2,3}\left(-i\right)}
		\begin{rowequmat}{cc}
			i	& 2i+1 \\ [.3em]
			0	& 4+\frac{1}{i} \\ [.3em]
			0	& 0
		\end{rowequmat}
	\end{equation*}
	Il numero di \emph{pivot}, in questo caso 2, rappresenta il rango della matrice ($\mathrm{rk}\left(M\right) = 2$), ovvero il numero di vettori colonna linearmente indipendenti. In altre parole, rappresenta la dimensione dello spazio generato dai vettori considerati inizialmente.

	Quindi, è possibile affermare che i vettori colonna della matrice non ridotta $M$, i quali corrispondono ai vettori colonna della matrice ridotta $M'$ (soprastante) che contengono i pivot, costituiscono una base dello spazio generato del sistema di generatori. Per cui, una base di $C\left(M\right)$:
	\begin{equation*}
		\dim\left(M\right) = \mathrm{rk}\left(M\right) \longrightarrow 2 = 2 \Longrightarrow
		C\left(M\right) = \left\{
			\begin{pmatrix}
				i  \\
				-1 \\
				-i
			\end{pmatrix},
			\begin{pmatrix}
				2i+1 \\
				2 	 \\
				2i
			\end{pmatrix}
		\right\}
	\end{equation*}\newpage

	\noindent
	Si prosegue l'esercizio calcolando una base della nullità di $M^{H}$. Quindi, si esegue l'eliminazione di Gauss:
	\begin{equation*}
		\begin{pmatrix}
			-i 		& -1	& i \\
			-2i+1	& 2		& -2i
		\end{pmatrix}
		\xlongrightarrow{E_{1,2}\left(-2-i\right)}
		\begin{pmatrix}
			-i 	& -1	& i \\
			0	& 4+i	& 1-4i
		\end{pmatrix}
	\end{equation*}
	Si ottiene il seguente sistema lineare:
	\begin{gather*}
		\begin{cases}
			-ix -y +iz = 0 \\
			\left(4+i\right)y + \left(1-4i\right) z = 0
		\end{cases}
		\longrightarrow
		\begin{cases}
			-ix -y +iz = 0 \\
			\\
			y = -\dfrac{\left(1-4i\right)}{\left(4+i\right)} z
		\end{cases}\\
		\\
		\dfrac{\left(1-4i\right)}{\left(4+i\right)} = \dfrac{\left(1-4i\right)\left(4-i\right)}{\left(4+i\right)\left(4-i\right)} = \dfrac{4-i-16i+4i^{2}}{16-4i+4i-i^2} = \dfrac{-17i}{17} = -i \\
		\\
		\begin{cases}
			-ix -y +iz = 0 \\
			y = -\left(-i\right) z
		\end{cases}
		\longrightarrow
		\begin{cases}
			-ix -iz +iz = 0 \\
			y = i z
		\end{cases}
		\longrightarrow
		\begin{cases}
			x = 0 \\
			y = i z
		\end{cases}\\
	\end{gather*}
	Dopo alcune semplificazioni, il risultato generale è:
	\begin{equation*}
		\begin{cases}
			x = 0 \\
			y = iz \\
			z = z
		\end{cases}
	\end{equation*}
	Dunque $N\left(M^{H}\right)$ è:
	\begin{equation*}
		N\left(M^{H}\right) = \left\{
			\left.
				z
				\begin{pmatrix}
					0 \\
					i \\
					1
				\end{pmatrix} \:
			\right| \: z \in \mathbb{C}
		\right\}
	\end{equation*}
	Per cui, si ha una base:
	\begin{equation*}
		\left\{
			\begin{pmatrix}
				0 \\
				i \\
				1
			\end{pmatrix}
		\right\}
	\end{equation*}

	\longline

	\subsubsection{Punto c}

	\textcolor{Green4}{\textbf{\emph{Si scriva una base di $\mathbb{C}^{3}$ che contiene le colonne di $M$.}}}\newline

	\noindent
	Dato che per definizione:
	\begin{equation*}
		\mathbb{C}^{3} = C\left(M\right) + N\left(M^{H}\right)
	\end{equation*}
	Allora una base di $\mathbb{C}^{3}$ è:
	\begin{equation*}
		\left\{
			\begin{pmatrix}
				i  \\
				-1 \\
				-i
			\end{pmatrix},
			\begin{pmatrix}
				2i+1 \\
				2 	 \\
				2i
			\end{pmatrix},
			\begin{pmatrix}
				0 \\
				i \\
				1
			\end{pmatrix}
		\right\}
	\end{equation*}\newpage

	\subsection{Esercizio 4}

	(\textbf{4 punti}) Vero o falso? Si motivi la risposta!

	\longline

	\subsubsection{Punto a}

	\textcolor{Green4}{\textbf{\emph{Il sistema omogeneo $Ax = 0$ ammette soltanto la soluzione banale}} $x = \begin{pmatrix}
		0 \\ 0 \\ 0
	\end{pmatrix}$ \textbf{\emph{dove}} $A = \begin{pmatrix}
		1 & 0 & 0 \\ 0 & 1 & 0
	\end{pmatrix}$}\newline

	\noindent
	Il sistema omogeneo $Ax = 0$ non ammette soltanto la soluzione banale. Per dimostrare ciò, si utilizza il teorema di Rouché-Capelli. Quindi, si calcola il rango delle matrici ridotte e aumentate:
	\begin{equation*}
		\begin{array}{rllll}
			A &=& \begin{pmatrix}
				1 & 0 & 0 \\ 0 & 1 & 0
			\end{pmatrix} &\longrightarrow& \mathrm{rk}\left(A\right) = 2 \\ [2em]
			\left(A \: | \: \mathbf{0}\right) &=& \begin{rowequmat}{ccc|c}
				1 & 0 & 0 & 0 \\
				0 & 1 & 0 & 0
			\end{rowequmat} &\longrightarrow& \mathrm{rk}\left(A \: | \: \mathbf{0}\right) = 2
		\end{array}
	\end{equation*}
	Il rango di entrambe le matrici è uguale a due, quindi, secondo il teorema, esistono una o infinite soluzioni. Precisamente:
	\begin{equation*}
		\mathrm{rk}\left(A\right) = \mathrm{\left(A \: | \: \mathbf{0}\right)} < n
	\end{equation*}
	Dove $n$ indica il numero di incognite, in questo caso 3 ($x,y,z$). Andando a sostituire i valori:
	\begin{equation*}
		2 = 2 < 3
	\end{equation*}
	La condizione è rispettata, quindi secondo il teorema di Rouché-Capelli, il sistema $Ax = 0$ ammette infinite soluzioni, precisamente $\infty^{n - \mathrm{rk}\left(A\right)} \rightarrow \infty^{3-2} = \infty$.
	
	\noindent
	In generale, la forma generale della soluzione deve essere:
	\begin{equation*}
		A x = 0 \Longrightarrow
		\begin{pmatrix}
			1 & 0 & 0 \\ 0 & 1 & 0
		\end{pmatrix}
		x\begin{pmatrix}
			0 \\ 0 \\ 1
		\end{pmatrix} = \mathbf{0}
	\end{equation*}\newpage

	\subsubsection{Punto b}

	\textcolor{Green4}{\textbf{\emph{L'insieme }} $\left\{\begin{pmatrix}
		-1 \\ 0
	\end{pmatrix}, \begin{pmatrix}
		2 \\ 1
	\end{pmatrix}, \begin{pmatrix}
		1 \\ 2
	\end{pmatrix}\right\}$ \textbf{\emph{è linearmente dipendente.}}}\newline

	\noindent
	Per verificarlo, si prendono tre generici scalari $a,b,c \in \mathbb{R}$ e si moltiplicano per i vettori:
	\begin{equation*}
		av_{1} + bv_{2} + cv_{3} = \mathbf{0}
	\end{equation*}
	Sostituendo i valori:
	\begin{equation*}
		a
		\begin{pmatrix}
			-1 \\ 0
		\end{pmatrix}
		+ b
		\begin{pmatrix}
			2 \\ 1
		\end{pmatrix}
		+ c
		\begin{pmatrix}
			1 \\ 2
		\end{pmatrix}
		=
		\begin{pmatrix}
			0 \\ 0
		\end{pmatrix}
	\end{equation*}
	Eseguendo i calcoli:
	\begin{equation*}
		\begin{pmatrix}
			-a + 2b + c \\
			b + 2c
		\end{pmatrix} \Longrightarrow
		\begin{cases}
			-a + 2b + c = 0 \\
			b + 2c = 0
		\end{cases} \longrightarrow
		\begin{cases}
			-a + 2\left(-2c\right) + c = 0 \\
			b = -2c
		\end{cases} \longrightarrow
		\begin{cases}
			c = -\frac{a}{3} \\
			b = -2c
		\end{cases}
	\end{equation*}
	È evidente che i tre vettori sono linearmente dipendenti. $b$ dipende da $c$ e $c$ dipende da $a$.

	\longline

	\subsection{Esercizio 5}

	(\textbf{1 punto}) Sia $V$ uno spazio vettoriale su $\mathbb{K}$. Si dimostri la seguente affermazione: se almeno uno dei vettori $v_{1}, \cdots, v_{2}$ è combinazione lineare dei rimanenti, allora $\left\{v_{1}, \cdots, v_{2}\right\}$ non è linearmente indipendente.\newline

	\noindent
	Dimostrazione lasciata al lettore.\newpage

	\section{Esame del 15/07/2022}
	
	\newpage
	\section{Esame del 02/09/2022}
	
	\newpage
	\section{Esame del 20/02/2023}
\end{document}