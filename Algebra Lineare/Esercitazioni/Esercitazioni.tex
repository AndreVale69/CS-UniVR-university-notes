\documentclass[a4paper]{article}
\usepackage[T1]{fontenc}			% pacchetto per \chapter
\usepackage[italian]{babel}
\usepackage[italian]{isodate}  		% formato delle date in italiano
\usepackage{graphicx}				% gestione delle immagini
\usepackage{amsfonts}
\usepackage{booktabs}				% tabelle di qualità superiore
\usepackage{amsmath}				% pacchetto matematica
\usepackage{mathtools}				% per sottolineare sotto le equazioni
\usepackage{stmaryrd} 				% per '\llbracket' e '\rrbracket'
\usepackage{amsthm}					% teoremi migliorati
\usepackage{enumitem}				% gestione delle liste
\usepackage{pifont}					% pacchetto con elenchi carini
\usepackage{enumitem}				% pacchetto per elenchi con lettere dell'alfabeto
\usepackage{cancel}					% per cancellare delle espressioni matematiche


\usepackage[x11names]{xcolor}		% pacchetto colori RGB
% Link ipertestuali per l'indice
\usepackage{xcolor}
\usepackage[linkcolor=black, citecolor=blue, urlcolor=cyan]{hyperref}
\hypersetup{
	colorlinks=true
}

%\usepackage{showframe}				% visualizzazione bordi
%\usepackage{showkeys}				% visualizzazione etichetta

\newtheorem{theorem}{\textcolor{Red3}{\underline{Teorema}}}
\newtheorem{lemma}{Lemma}
\renewcommand{\qedsymbol}{QED}
\newcommand{\exec}[1]{\llbracket #1\:\rrbracket}
\newcommand{\dquotes}[1]{``#1''}
\newcommand{\longline}{\noindent\rule{\textwidth}{0.4pt}}

\newenvironment{rowequmat}[1]{\left(\array{@{}#1@{}}}{\endarray\right)}
\newenvironment{rowequmatbra}[1]{\left[\array{@{}#1@{}}}{\endarray\right]}

\begin{document}
	\author{VR443470}
	\title{Esercitazioni Algebra Lineare}
	\date{\printdayoff\today}
	\maketitle
	
	\newpage
	
	% indice
	\tableofcontents
	
	\newpage
	
	\section{Basi}
	
	\subsection{Somma e trasposte}
	
	I classici esercizi di Algebra Lineare prevedono varie operazioni sulle matrici. Partendo dalle basi, si introducono le operazioni di somma e trasposizione.\newline
	
	\noindent
	Date $3$ matrici $A, B, C$:
	\begin{equation*}
		A = \begin{bmatrix}
			   1 & -i & 3 \\
			-2+i &  5 & i2
		\end{bmatrix} \hspace{2em}
		B = \begin{bmatrix}
			3i	& 2		\\
			4	& -i 	\\
			2-i	& -1
		\end{bmatrix} \hspace{2em}
		C = \begin{bmatrix}
			-2	& 3i	& 2-i	\\
			4i	& 1		& 0
		\end{bmatrix}
	\end{equation*}
	La \textcolor{Red3}{prima operazione} da eseguire è la classificazione delle matrici. In questo caso, le matrici hanno le seguenti dimensioni:
	\begin{equation*}
		A\in\mathbb{M}_{2\times3} \hspace{2em} B\in\mathbb{M}_{3\times2} \hspace{2em} C\in\mathbb{M}_{2\times3}
	\end{equation*}
	La \textcolor{Red3}{seconda operazione} da eseguire è controllare se è possibile eseguire l'operazione richiesta dall'esercizio. In questo caso, viene chiesta la somma. Per eseguire quest'ultima (vale lo stesso per la sottrazione), le dimensioni delle matrici devono essere tutte \textbf{identiche}. Dato che in questo caso la matrice $B \left(3\times2\right)$ differisce di dimensione rispetto alle due matrici $A,C \left(2\times3\right)$, è necessario fare qualcosa per eseguire l'operazione di somma.\newline
	
	\noindent
	Dato che è ancora l'inizio, non verranno effettuate manipolazioni complesse. Quindi, si supponga di eseguire questa operazione di somma/sottrazione:
	\begin{equation*}
		2A^{T} - 4\overline{B} + 3C^{T}
	\end{equation*}
	Prima di eseguire l'operazione, si ottengono le relative matrici coniugate e trasposte. L'operazione di \textbf{coniugazione} è eseguibile cambiando i segni ai valori complessi (quindi alle $i$). Invece, l'operazione di \textbf{trasposizione} ($T$) inverte le colonne e le righe di una matrice. I risultati sono:
	\begin{equation*}
		\begin{array}{lllllllll}
			A & = \begin{bmatrix}
				1 	& -i	& 3	\\
				-2+i&  5 	& i2
			\end{bmatrix} &
			A^{T} & = \begin{bmatrix}
				1	& -2+i	\\
				-i	& 5		\\
				3	& i2
			\end{bmatrix} &
			2A^{T} & = \begin{bmatrix}
				2	& -4+2i	\\
				-2i	& 10	\\
				6	& 4i
			\end{bmatrix} \\
			%%%%%%%%%%%%%%%%%%%%%%%%%%%%%%%%%%%%%%%%
			&&&&& \\
			%%%%%%%%%%%%%%%%%%%%%%%%%%%%%%%%%%%%%%%%
			B & = \begin{bmatrix}
				3i	& 2		\\
				4	& -i 	\\
				2-i	& -1
			\end{bmatrix} &
			\overline{B} & = \begin{bmatrix}
				-3i	& 2		\\
				4	& i		\\
				2+i	& -1
			\end{bmatrix} &
			4\overline{B} & = \begin{bmatrix}
				-12i	& 8		\\
				16		& 4i	\\
				8+4i	& -4
			\end{bmatrix} \\
			%%%%%%%%%%%%%%%%%%%%%%%%%%%%%%%%%%%%%%%%
			&&&&& \\
			%%%%%%%%%%%%%%%%%%%%%%%%%%%%%%%%%%%%%%%%
			C & = \begin{bmatrix}
				-2	& 3i	& 2-i	\\
				4i	& 1		& 0
			\end{bmatrix} &
			C^{T} & = \begin{bmatrix}
				-2	& 4i	\\
				3i	& 1		\\
				2-i	& 0
			\end{bmatrix} &
			3C^{T} & = \begin{bmatrix}
				-6	& 12i	\\
				9i	& 3		\\
				6-3i& 0
			\end{bmatrix}
		\end{array}
	\end{equation*}\newpage
	Adesso è possibile eseguire la sottrazione tra $\alpha = 2A^{T} - 4\overline{B}$ e successivamente la somma tra $\alpha + 3C^{T}$:
	\begin{equation*}
		\alpha = 2A^{T} - 4\overline{B} = \begin{bmatrix}
			2	& -4+2i	\\
			-2i	& 10	\\
			6	& 4i
		\end{bmatrix} - \begin{bmatrix}
			-12i	& 8		\\
			16		& 4i	\\
			8+4i	& -4
		\end{bmatrix} = \begin{bmatrix}
			2+12i	& 12+2i		\\
			-16-2i	& 10-4i		\\
			1-4i	& 4+4i
		\end{bmatrix}
	\end{equation*}
	Si esegue la somma:
	\begin{equation*}
		\alpha + 3C^{T} = \begin{bmatrix}
			2+12i	& 12+2i		\\
			-16-2i	& 10-4i		\\
			1-4i	& 4+4i
		\end{bmatrix} + \begin{bmatrix}
			-6	& 12i	\\
			9i	& 3		\\
			6-3i& 0
		\end{bmatrix} = \begin{bmatrix}
			-4+12i	& -12+14i	\\
			-16+7i	& 13-4i		\\
			7-7i	& 4+4i
		\end{bmatrix}
	\end{equation*}\newpage

	\subsection{(Anti-)Hermitiane e (anti-)simmetriche}
	
	Diamo alcune definizioni per capire come fare gli esercizi:
	\begin{itemize}
		\item È possibile abbreviare letteralmente le operazioni di trasposizione e coniugazione scrivendo \textbf{trasposta-coniugata};
		
		\item Una matrice viene detta \textcolor{Red3}{\textbf{hermitiana}} quando la matrice originaria è uguale alla sua trasposta-coniugata:
		\begin{equation*}
			\overline{\left(A^{T}\right)} = \left(\overline{A}\right)^{T} = A \Longrightarrow A^{H}
		\end{equation*}
		
		\item Una matrice viene detta \textcolor{Red3}{\textbf{anti-hermitiana}} quando la matrice trasposta-coniugata corrisponde alla matrice originaria ma cambiata di segno:
		\begin{equation*}
			\overline{\left(A^{T}\right)} = \left(\overline{A}\right)^{T} = -A \Longrightarrow \text{ anti-hermitiana}
		\end{equation*}
		
		\item Una matrice viene detta \textcolor{Red3}{\textbf{simmetrica}} quando la matrice originaria è uguale alla sua trasposta:
		\begin{equation*}
			A = A^{T} \Longrightarrow \text{ simmetrica}
		\end{equation*}
		
		\item Una matrice viene detta \textcolor{Red3}{\textbf{anti-simmetrica}} quando la sua trasposta corrisponde alla matrice originaria ma cambiata di segno:
		\begin{equation*}
			-A = A^{T} \Longrightarrow \text{ anti-simmetrica}
		\end{equation*}
	\end{itemize}
	Prendendo come \textcolor{Green4}{\textbf{esempio}} le tre matrici $A,B,C$:
	\begin{equation*}
		A = \begin{bmatrix}
			2i 	& 3		\\
			-3	& i
		\end{bmatrix} \hspace{2em}
		B = \begin{bmatrix}
			i 	& 2		\\
			2	& i
		\end{bmatrix} \hspace{2em}
		C = \begin{bmatrix}
			-1 	& i3	\\
			-i3	& 1
		\end{bmatrix}
	\end{equation*}
	Si eseguono le rispettive operazioni di trasposizione e coniugazione:
	\begin{equation*}
		\begin{array}{lllllllll}
			A & = \begin{bmatrix}
				2i 	& 3		\\
				-3	& i
			\end{bmatrix} &
			A^{T} & = \begin{bmatrix}
				2i	& -3	\\
				3	& i
			\end{bmatrix} &
			\overline{A^{T}} & = \begin{bmatrix}
				-2i	& -3	\\
				3	& -i
			\end{bmatrix} \\
			%%%%%%%%%%%%%%%%%%%%%%%%%%%%%%%%%%%%%%%%
			&&&&& \\
			%%%%%%%%%%%%%%%%%%%%%%%%%%%%%%%%%%%%%%%%
			B & = \begin{bmatrix}
				i 	& 2		\\
				2	& i
			\end{bmatrix} &
			B^{T} & = \begin{bmatrix}
				i	& 2		\\
				2	& i
			\end{bmatrix} &
			\overline{B^{T}} & = \begin{bmatrix}
				-i	& 2		\\
				2	& -i
			\end{bmatrix} \\
			%%%%%%%%%%%%%%%%%%%%%%%%%%%%%%%%%%%%%%%%
			&&&&& \\
			%%%%%%%%%%%%%%%%%%%%%%%%%%%%%%%%%%%%%%%%
			C & = \begin{bmatrix}
				-1 	& i3	\\
				-i3	& 1
			\end{bmatrix} &
			C^{T} & = \begin{bmatrix}
				-1	& -i3	\\
				i3	& 1
			\end{bmatrix} &
			\overline{C^{T}} & = \begin{bmatrix}
				-1	& i3	\\
				-i3	& 1
			\end{bmatrix}
		\end{array}
	\end{equation*}
	Da questi risultati è possibile notare come $A, B$ \underline{non} siano hermitiane, mentre $C$ lo sia. Inoltre, dalle trasposte è possibile osservare come $A, C$ \underline{non} siano simmetriche, mentre $B$ lo sia. Invece, per verificare l'anti-hermitiana e l'anti-simmetrica, è necessario negare le matrici originarie:
	\begin{equation*}
		-A = \begin{bmatrix}
			-2i	& -3	\\
			3	& -i
		\end{bmatrix} \hspace{2em}
		-B = \begin{bmatrix}
			-i 	& -2	\\
			-2	& -i
		\end{bmatrix} \hspace{2em}
		-C = \begin{bmatrix}
			1 	& -i3	\\
			i3	& -1
		\end{bmatrix}
	\end{equation*}
	Da questi risultati è possibile notare come $B,C$ \underline{non} siano anti-hermitiane, mentre $A$ lo sia. Inoltre, osservando nuovamente le trasposte, è possibile osservare come $A, B$ e $C$ \underline{non} siano anti-simmetriche.\newpage
	
	\subsection{Prodotto tra matrici righe per colonne}
	
	La \textcolor{Red3}{prima operazione} da eseguire per la moltiplicazione tra matrici righe per colonne è la verifica delle righe e delle colonne. Il prodotto tra matrici è ammesso solo se il numero delle colonne del primo operando è uguale al numero delle righe del secondo operando. Per esempio, la seguente operazione è ammessa:
	\begin{equation*}
		A_{m \times n} \cdot B_{n \times l}
	\end{equation*}
	Inoltre, la matrice risultante avrà come dimensione le righe del primo operando e le colonne del secondo. Quindi:
	\begin{equation*}
		C_{m \times l} = A_{m \times n} \cdot B_{n \times l}
	\end{equation*}
	La \textcolor{Red3}{seconda operazione} è la moltiplicazione vera e propria. Per farla, si prende ogni riga del primo operando e si moltiplica per ogni colonna del secondo operando. Dopo la moltiplicazione di una riga per una colonna, si sommano i risultati. Quindi:
	\begin{gather*}
		A_{m,n} = \begin{bmatrix}
			1,1 	& \cdots & 1,n 		\\
			\cdots 	& \cdots & \cdots	\\
			m,1		& \cdots & m,n
		\end{bmatrix} \times
		B_{n,l} = \begin{bmatrix}
			1,1 	& \cdots & 1,l 		\\
			\cdots 	& \cdots & \cdots	\\
			n,1		& \cdots & n,l
		\end{bmatrix} \\
		\\
		C_{1,1} = A_{1,1} \cdot B_{1,1} + \cdots + A_{1,n} \cdot B_{n,1} \\
		\cdots
	\end{gather*}
	Si presenta un \textcolor{Green4}{\textbf{esempio}}. Date due matrici $A,B$:
	\begin{equation*}
		A = \begin{bmatrix}
			1 & 0 &  2 \\
			0 & 3 & -1
		\end{bmatrix} \hspace{2em}
		B = \begin{bmatrix}
			4  	& 1 \\
			-2 	& 2 \\
			0 	& 3
		\end{bmatrix}
	\end{equation*}
	L'operazione di moltiplicazione di righe per colonne è ammessa poiché le righe di $A$ ($2$) sono lo stesso numero delle colonne di $B$ ($2$):
	\begin{equation*}
		C = A \cdot B = \begin{bmatrix}
			4  & 7 \\
			-6 & 3
		\end{bmatrix}
	\end{equation*}
	I calcoli sono banali. Si lasciano qua di seguito i passaggi:
	\begin{equation*}
		\begin{array}{lllllllll}
			C_{1,1} & = & 1 \cdot 4 & + & 0 \cdot -2 & + & 2 \cdot 0 & = & 4	\\
			C_{1,2} & = & 1 \cdot 1 & + & 0 \cdot 2 & + & 2 \cdot 3 & = & 7		\\
			C_{2,1} & = & 0 \cdot 4 & + & 3 \cdot -2 & + & -1 \cdot 0 & = & -6	\\
			C_{2,2} & = & 0 \cdot 1 & + & 3 \cdot 2 & + & -1 \cdot 3 & = & 3
		\end{array}
	\end{equation*}\newpage

	\noindent
	Si presenta un altro \textcolor{Green4}{\textbf{esempio}} ma con i numeri complessi. Date le due matrici $A, B$:
	\begin{equation*}
		A_{2,4} = \begin{bmatrix}
			1+i	& i					& 0		& \overline{3+2i} \\
			-i 	& \overline{-1-3i} 	& 7i	& 6i
		\end{bmatrix} \times
		B_{4,2} = \begin{bmatrix}
			2  				& 4i 	\\
			-3i				& 0 	\\
			\overline{1-i} 	& -2i	\\
			5-i				& \overline{2+i}
		\end{bmatrix}
	\end{equation*}
	Prima di eseguire la moltiplicazione tra righe e colonne si risolvono i coniugati, due nella matrice $A$ e due nella matrice $B$:
	\begin{equation*}
		A_{2,4} = \begin{bmatrix}
			1+i	& i			& 0		& 3-2i \\
			-i 	& -1+3i 	& 7i	& 6i
		\end{bmatrix} \times
		B_{4,2} = \begin{bmatrix}
			2  	& 4i 	\\
			-3i	& 0 	\\
			1+i	& -2i	\\
			5-i	& 2-i
		\end{bmatrix}
	\end{equation*}
	L'operazione di moltiplicazione di righe per colonne è ammessa. Quindi si presenta qui di seguito i calcoli eseguiti (attenzione alle parti immaginarie, si ricorda che $i^{2} = -1$):
	\begin{gather*}
		\begin{array}{llllllllllll}
			C_{1,1} & = & \left(1+i\right) \cdot 2 	& + & i \cdot \left(-3i\right) 		& + & 0 \cdot \left(1+i\right) 	& + & \left(3-2i\right) \cdot \left(5-i\right) 	& = & 18-11i \\
			C_{1,2} & = & \left(1+i\right) \cdot 4i 	& + & i \cdot 0 		& + & 0 \cdot -2i	& + & \left(3-2i\right) \cdot \left(2-i\right)	& = & -3i  \\
			C_{2,1} & = & -i \cdot 2	& + & \left(-1+3i\right) \cdot -3i 	& + & 7i \cdot \left(1+i\right) 	& + & 6i \cdot \left(5-i\right) 		& = & 8 + 38i \\
			C_{2,2} & = & -i \cdot 4i 	& + & \left(-1+3i\right) \cdot 0 	& + & 7i \cdot -2i 	& + & 6i \cdot \left(2-i\right) 		& = & 24 + 12i
		\end{array}\\
		\\
		C = \begin{bmatrix}
			18-11i & -3i \\
			8+38i & 24+12i
		\end{bmatrix}
	\end{gather*}\newpage

	\subsection{Prodotto tra matrici - Casi particolari}
	
	Esistono alcuni casi particolari quando vengono eseguite le moltiplicazioni tra matrici:
	\begin{enumerate}
		\item La \textbf{moltiplicazione tra un vettore riga e un vettore colonna} (\underline{non} viceversa) restituisce solamente un valore, chiamato \textbf{prodotto scalare}. Ovviamente, il numero delle colonne del vettore riga e il numero di righe del vettore colonna devono essere identici:
		\begin{equation*}
			A_{1,n} = \begin{bmatrix}
				\cdots & \cdots & \cdots
			\end{bmatrix} \times
			B_{n,1} = \begin{bmatrix}
				\cdots \\
				\cdots \\
				\cdots
			\end{bmatrix} \rightarrow
			C
		\end{equation*}
	
		\item La \textbf{moltiplicazione tra un vettore colonna e un vettore riga} restituisce una matrice avente il numero di righe pari al vettore colonna e il numero di colonne pari al vettore riga:
		\begin{equation*}
			B_{n,1} = \begin{bmatrix}
				\cdots \\
				\cdots \\
				\cdots
			\end{bmatrix} \times
			A_{1,n} = \begin{bmatrix}
				\cdots & \cdots & \cdots
			\end{bmatrix} \rightarrow
			C_{n, n}
		\end{equation*}
	
		\item La \textbf{moltiplicazione tra una matrice e un vettore colonna} restituisce un vettore colonna. Ovviamente, per applicare questa operazione è necessario che il numero delle colonne della matrice sia uguale al numero di righe del vettore:
		\begin{equation*}
			A_{m,n} = \begin{bmatrix}
				\cdots & \cdots & \cdots \\
				\cdots & \cdots & \cdots \\
				\cdots & \cdots & \cdots
			\end{bmatrix} \times
			B_{n, 1} = \begin{bmatrix}
				\cdots \\
				\cdots \\
				\cdots
			\end{bmatrix} \rightarrow
			C_{m, 1}
		\end{equation*}
	\end{enumerate}\newpage

	\section{Eliminazione di Gauss}
	
	\subsection{Le 3 operazioni}\label{EG: 3 operazioni}
	
	L'eliminazione di Gauss prevede 3 operazioni principali da applicare per ottenere la forma ridotta (forma finale):
	\begin{enumerate}
		\item Un'equazione può essere moltiplicata per uno scalare non nullo;
		
		\item Un'equazione viene sostituita con la somma tra lei e un'altra equazione, in cui quest'ultima è stata prima moltiplicata per uno scalare non nullo. Quindi, viene scelta un'equazione da moltiplicare per uno scalare non nullo e successivamente viene effettuata la somma tra il risultato della moltiplicazione e l'equazione interessata;
		
		\item Scambio di due equazioni.
	\end{enumerate}

	\subsection{Tipi di soluzione}
	
	Possono esistere tre tipi di soluzione:
	\begin{itemize}
		\item Tipo uno (una sola soluzione). Intuibile dalla forma ridotta (finale) di Gauss poiché l'ultima riga ha solo una variabile con valore positivo;
		
		\item Tipo zero (non esistono soluzioni per il sistema). Intuibile dalla forma ridotta (finale) di Gauss poiché l'ultima riga ha solo variabili nulle;
		
		\item Infinito (le soluzioni del sistema sono infinite). Intuibile dalla forma ridotta (finale) di Gauss poiché l'ultima riga presenta più di una variabile con valore positivo.
	\end{itemize}
	I prossimi paragrafi mostreranno tutte e tre le casistiche.\newpage

	\subsection{Esercizi}
	
	\subsubsection{Tipo uno - Una soluzione}
	
	Dato il seguente sistema:
	\begin{equation*}
		\begin{cases}
			2x + 4y + 4z = 4 \\
			x - z = 1 \\
			-x + 3y + 4z = 2
		\end{cases}
	\end{equation*}
	Si calcola la matrice risultante dopo l'eliminazione di Gauss.\newline
	
	\noindent
	Il \textcolor{Red3}{primo passo} è scrivere la matrice aumentata. Essa è banale da comporre, consiste nello scrivere i coefficienti di ogni variabile ($x,y,z$) in una matrice e aggiungere una colonna sulla destra in cui ci sono i valori risultanti. Si passa all'atto pratico:
	\begin{equation*}
		\begin{rowequmatbra}{ccc|c}
			2  & 4 &  4 & 4 \\
			1  & 0 & -1 & 1 \\
			-1 & 3 &  4 & 2
		\end{rowequmatbra}
	\end{equation*}
	Il \textcolor{Red3}{secondo passo} è eseguire alcune considerazioni sulla forma che si vuole ottenere e procedere con le varie operazioni. L'obbiettivo è quello di ottenere una matrice uni-triangolare superiore\footnote{Una matrice uni-triangolare superiore è una forma particolare in cui i valori sotto alla diagonale principale sono nulli, cioè uguale a zero}. In questo caso, si inizia con lo \textbf{scambio della prima riga con la seconda}:
	\begin{equation*}
		\begin{rowequmatbra}{ccc|c}
			2  & 4 &  4 & 4 \\
			1  & 0 & -1 & 1 \\
			-1 & 3 &  4 & 2
		\end{rowequmatbra} \xlongrightarrow{E_{1,2}}
		\begin{rowequmatbra}{ccc|c}
			1  & 0 & -1 & 1 \\
			2  & 4 &  4 & 4 \\
			-1 & 3 &  4 & 2
		\end{rowequmatbra}
	\end{equation*}
	Si \textbf{moltiplica la seconda riga per lo scalare} $\frac{1}{2}$:
	\begin{equation*}
		\begin{rowequmatbra}{ccc|c}
			1  & 0 & -1 & 1 \\
			2  & 4 &  4 & 4 \\
			-1 & 3 &  4 & 2
		\end{rowequmatbra} \xlongrightarrow{E_{2}\left(\frac{1}{2}\right)}
		\begin{rowequmatbra}{ccc|c}
			1  & 0 & -1 & 1 \\
			1  & 2 &  2 & 2 \\
			-1 & 3 &  4 & 2
		\end{rowequmatbra}
	\end{equation*}
	Si \textbf{moltiplica la prima riga per -1 e successivamente si somma la prima riga con la seconda}:
	\begin{equation*}
		\begin{rowequmatbra}{ccc|c}
			1  & 0 & -1 & 1 \\
			1  & 2 &  2 & 2 \\
			-1 & 3 &  4 & 2
		\end{rowequmatbra} \xlongrightarrow{E_{2,1}\left(-1\right)}
		\begin{rowequmatbra}{ccc|c}
			1  & 0 & -1 & 1 \\
			0  & 2 &  3 & 1 \\
			-1 & 3 &  4 & 2
		\end{rowequmatbra}
	\end{equation*}
	Si \textbf{moltiplica la prima riga per 1 e successivamente si somma la prima riga con la terza}:
	\begin{equation*}
		\begin{rowequmatbra}{ccc|c}
			1  & 0 & -1 & 1 \\
			0  & 2 &  3 & 1 \\
			-1 & 3 &  4 & 2
		\end{rowequmatbra} \xlongrightarrow{E_{3,1}\left(1\right)}
		\begin{rowequmatbra}{ccc|c}
			1  & 0 & -1 & 1 \\
			0  & 2 &  3 & 1 \\
			0  & 3 &  3 & 3
		\end{rowequmatbra}
	\end{equation*}
	Si \textbf{moltiplica la seconda riga per lo scalare} $\frac{1}{2}$:
	\begin{equation*}
		\begin{rowequmatbra}{ccc|c}
			1  & 0 & -1 & 1 \\
			0  & 2 &  3 & 1 \\
			0  & 3 &  3 & 3
		\end{rowequmatbra} \xlongrightarrow{E_{2}\left(\frac{1}{2}\right)}
		\begin{rowequmatbra}{ccc|c}
			1  & 0 & -1 & 1 \\
			0  & 1 &  \frac{3}{2} & \frac{1}{2} \\
			0  & 3 &  3 & 3
		\end{rowequmatbra}
	\end{equation*}
	Si \textbf{moltiplica la seconda riga per $-3$ e successivamente si somma la terza riga con la seconda}:
	\begin{equation*}
		\begin{rowequmatbra}{ccc|c}
			1  & 0 & -1 & 1 \\
			0  & 1 &  \frac{3}{2} & \frac{1}{2} \\
			0  & 3 &  3 & 3
		\end{rowequmatbra} \xlongrightarrow{E_{3,2}\left(-3\right)}
		\begin{rowequmatbra}{ccc|c}
			1  & 0 & -1 & 1 \\
			0  & 1 &  \frac{3}{2} & \frac{1}{2} \\
			0  & 0 &  -\frac{3}{2} & \frac{3}{2}
		\end{rowequmatbra}
	\end{equation*}
	Si \textbf{moltiplica la terza riga per lo scalare} $-\frac{2}{3}$:
	\begin{equation*}
		\begin{rowequmatbra}{ccc|c}
			1  & 0 & -1 & 1 \\
			0  & 1 &  \frac{3}{2} & \frac{1}{2} \\
			0  & 0 &  -\frac{3}{2} & \frac{3}{2}
		\end{rowequmatbra} \xlongrightarrow{E_{3,2}\left(-3\right)}
		\begin{rowequmatbra}{ccc|c}
			1  & 0 & -1 & 1 \\
			0  & 1 &  \frac{3}{2} & \frac{1}{2} \\
			0  & 0 &  1 & -1
		\end{rowequmatbra}
	\end{equation*}
	Si ottiene così la forma ridotta di Gauss.\newline
	
	\noindent
	Il \textcolor{Red3}{terzo passo} è classificare la forma ottenuta. Dalla forma ridotta è possibile dedurre che si è di fronte al tipo uno, ovvero esiste una sola soluzione per il sistema.
	
	Adesso è possibile ricostruire il vettore delle soluzioni andando al contrario. Quindi, si parte dall'ultima riga e sostituendo si va fino all'inizio:
	\begin{gather*}
		\begin{array}{rll}
			z & = & -1 \\
			\\
			y + \frac{3}{2}\left(-1\right) = \frac{1}{2} \rightarrow y & = & \phantom{-}2 \\
			\\
			x + -1\left(-1\right) = 1 \rightarrow x & = & \phantom{-}0
		\end{array}
	\end{gather*}\newpage

	\subsubsection{Tipo zero - Nessuna soluzione}
	
	Dato il seguente sistema:
	\begin{equation*}
		\begin{cases}
			x + 2y - z = 1 	\\
			-x - y + 2z = 1 \\
			x + 3y + z = 4 	\\
			2x + 4y -2z = -1
		\end{cases}
	\end{equation*}
	Si calcola la matrice risultante dopo l'eliminazione di Gauss.\newline
	
	\noindent
	Il \textcolor{Red3}{primo passo} è scrivere la matrice aumentata. Essa è banale da comporre, consiste nello scrivere i coefficienti di ogni variabile ($x,y,z$) in una matrice e aggiungere una colonna sulla destra in cui ci sono i valori risultanti. Si passa all'atto pratico:
	\begin{equation*}
		\begin{rowequmatbra}{ccc|c}
			 1  &  2 & -1 &  1 \\
			-1  & -1 &  2 &  1 \\
			 1  &  3 &  1 &  4 \\
			 2  &  4 & -2 & -1
		\end{rowequmatbra}
	\end{equation*}
	Il \textcolor{Red3}{secondo passo} è eseguire alcune considerazioni sulla forma che si vuole ottenere e procedere con le varie operazioni. L'obbiettivo è quello di ottenere una matrice uni-triangolare superiore\footnote{Una matrice uni-triangolare superiore è una forma particolare in cui i valori sotto alla diagonale principale sono nulli, cioè uguale a zero}. In questo caso, si inizia \textbf{moltiplicando la prima riga per 1 e successivamente si somma la prima riga con la seconda}:
	\begin{equation*}
		\begin{rowequmatbra}{ccc|c}
			1  &  2 & -1 &  1 \\
			-1  & -1 &  2 &  1 \\
			1  &  3 &  1 &  4 \\
			2  &  4 & -2 & -1
		\end{rowequmatbra} \xlongrightarrow{E_{2,1}\left(1\right)}
		\begin{rowequmatbra}{ccc|c}
			1  & 2 & -1 &  1 \\
			0  & 1 &  1 &  2 \\
			1  & 3 &  1 &  4 \\
			2  & 4 & -2 & -1
		\end{rowequmatbra}
	\end{equation*}
	Per rapidità, si eseguono in ordine le due operazioni seguenti. Si \textbf{moltiplica la prima riga per -1 e successivamente si somma la prima riga con la terza}, e poi si \textbf{moltiplica la prima riga per -2 e successivamente si somma la prima riga con la quarta}:
	\begin{equation*}
		\begin{rowequmatbra}{ccc|c}
			1  & 2 & -1 &  1 \\
			0  & 1 &  1 &  2 \\
			1  & 3 &  1 &  4 \\
			2  & 4 & -2 & -1
		\end{rowequmatbra} \xlongrightarrow[E_{4,1}\left(-2\right)]{E_{3,1}\left(-1\right)}
		\begin{rowequmatbra}{ccc|c}
			1  & 2 & -1 &  1 \\
			0  & 1 &  1 &  2 \\
			0  & 1 &  2 &  3 \\
			0  & 0 &  0 & -3
		\end{rowequmatbra}
	\end{equation*}
	Si ottiene così la forma ridotta di Gauss.\newline
	
	\noindent
	Il \textcolor{Red3}{terzo passo} è classificare la forma ottenuta. Dalla forma ridotta è possibile dedurre che si è di fronte al tipo zero, ovvero non esiste nessuna soluzione per il sistema. L'esercizio è concluso:
	\begin{equation*}
		\cancel{\exists} z : 0 \cdot z = -3
	\end{equation*}\newpage

	\subsubsection{Infinito}
	
	Dato il seguente sistema:
	\begin{equation*}
		\begin{cases}
			x + 2y + w = 0 	\\
			2x + 5y + 4z + 4w = 0 \\
			3x + 5y - 6z + 4w = 0
		\end{cases}
	\end{equation*}
	Si calcola la matrice risultante dopo l'eliminazione di Gauss.\newline
	
	\noindent
	Il \textcolor{Red3}{primo passo} è scrivere la matrice aumentata. Essa è banale da comporre, consiste nello scrivere i coefficienti di ogni variabile ($x,y,z,w$) in una matrice e aggiungere una colonna sulla destra in cui ci sono i valori risultanti. Si passa all'atto pratico:
	\begin{equation*}
		\begin{rowequmatbra}{cccc|c}
			1  &  2 &  0 &  1 & 0 	\\
			2  &  5 &  4 &  4 & 0	\\
			3  &  5 & -6 &  4 & 0
		\end{rowequmatbra}
	\end{equation*}
	Il \textcolor{Red3}{secondo passo} è eseguire alcune considerazioni sulla forma che si vuole ottenere e procedere con le varie operazioni. L'obbiettivo è quello di ottenere una matrice uni-triangolare superiore\footnote{Una matrice uni-triangolare superiore è una forma particolare in cui i valori sotto alla diagonale principale sono nulli, cioè uguale a zero}. In questo caso, si inizia con due operazioni per velocizzare i calcoli. Si \textbf{moltiplica la prima riga per -2 e successivamente si somma la prima riga con la seconda}, e poi si \textbf{moltiplica la prima riga per -3 e successivamente si somma la prima riga con la terza}:
	\begin{equation*}
		\begin{rowequmatbra}{cccc|c}
			1  &  2 &  0 &  1 & 0 	\\
			2  &  5 &  4 &  4 & 0	\\
			3  &  5 & -6 &  4 & 0
		\end{rowequmatbra} \xlongrightarrow[E_{3,1}\left(-3\right)]{E_{2,1}\left(-2\right)}
		\begin{rowequmatbra}{cccc|c}
			1  &  2 &  0 &  1 & 0 	\\
			0  &  1 &  4 &  2 & 0	\\
			0  & -1 & -6 &  1 & 0
		\end{rowequmatbra}
	\end{equation*}
	Si \textbf{moltiplica la seconda riga per 1 e successivamente si somma la seconda riga con la terza}:
	\begin{equation*}
		\begin{rowequmatbra}{cccc|c}
			1  &  2 &  0 &  1 & 0 	\\
			0  &  1 &  4 &  2 & 0	\\
			0  & -1 & -6 &  1 & 0
		\end{rowequmatbra} \xlongrightarrow{E_{3,2}\left(1\right)}
		\begin{rowequmatbra}{cccc|c}
			1  &  2 &  0 &  1 & 0 	\\
			0  &  1 &  4 &  2 & 0	\\
			0  &  0 & -2 &  3 & 0
		\end{rowequmatbra}
	\end{equation*}
	Si \textbf{moltiplica la terza riga per uno scalare $\frac{1}{2}$}:
	\begin{equation*}
		\begin{rowequmatbra}{cccc|c}
			1  &  2 &  0 &  1 & 0 	\\
			0  &  1 &  4 &  2 & 0	\\
			0  &  0 & -2 &  3 & 0
		\end{rowequmatbra} \xlongrightarrow{E_{3}\left(-\frac{1}{2}\right)}
		\begin{rowequmatbra}{cccc|c}
			1  &  2 &  0 &  1 & 0 	\\
			0  &  1 &  4 &  2 & 0	\\
			0  &  0 &  1 &  -\frac{3}{2} & 0
		\end{rowequmatbra}
	\end{equation*}
	Si ottiene così la forma ridotta di Gauss.\newpage
	
	\noindent
	Il \textcolor{Red3}{terzo passo} è classificare la forma ottenuta. Dalla forma ridotta è possibile dedurre che si è di fronte all'infinito, ovvero esistono un'infinità di soluzioni che dipendono da, in questo caso, un parametro:
	\begin{equation*}
		\begin{array}{rll}
			z - \dfrac{3}{2} w = 0 & \longrightarrow & z = \dfrac{3}{2} w \\
			\\
			y + 4\left(\dfrac{3}{2}w\right) + 2w = 0 & \longrightarrow & y = -8w \\
			\\
			x + 2\left(-8w\right) + 1w = 0 & \longrightarrow & x = 15w
		\end{array}
	\end{equation*}
	Quindi il vettore soluzione sarà composto in questo modo:
	\begin{equation*}
		soluzione = \begin{bmatrix}
			15w \\[0.3em]
			-8w \\[0.3em]
			\frac{3}{2}w \\[0.3em]
			w
		\end{bmatrix} = w \cdot \begin{bmatrix}
			15 \\[0.3em]
			-8 \\[0.3em]
			\frac{3}{2} \\[0.3em]
			1
		\end{bmatrix}
	\end{equation*}\newpage

	\subsubsection{Infinito - Caso particolare}
	
	Dato il seguente sistema:
	\begin{equation*}
		\begin{cases}
			x + y + z + w = -1 	 \\
			x + 2y + z + 2w = -1 \\
			2x + 3y + 2z + 3w = -2
		\end{cases}
	\end{equation*}
	Si calcola la matrice risultante dopo l'eliminazione di Gauss.\newline
	
	\noindent
	Il \textcolor{Red3}{primo passo} è scrivere la matrice aumentata. Essa è banale da comporre, consiste nello scrivere i coefficienti di ogni variabile ($x,y,z,w$) in una matrice e aggiungere una colonna sulla destra in cui ci sono i valori risultanti. Si passa all'atto pratico:
	\begin{equation*}
		\begin{rowequmatbra}{cccc|c}
			1  &  1 &  1 &  1 & -1 	\\
			1  &  2 &  1 &  2 & -1	\\
			2  &  3 &  2 &  3 & -2
		\end{rowequmatbra}
	\end{equation*}
	Il \textcolor{Red3}{secondo passo} è eseguire alcune considerazioni sulla forma che si vuole ottenere e procedere con le varie operazioni. L'obbiettivo è quello di ottenere una matrice uni-triangolare superiore\footnote{Una matrice uni-triangolare superiore è una forma particolare in cui i valori sotto alla diagonale principale sono nulli, cioè uguale a zero}. In questo caso, si inizia con due operazioni per velocizzare i calcoli. Si \textbf{moltiplica la prima riga per -1 e successivamente si somma la prima riga con la seconda}, e poi si \textbf{moltiplica la prima riga per -2 e successivamente si somma la prima riga con la terza}:
	\begin{equation*}
		\begin{rowequmatbra}{cccc|c}
			1  &  1 &  1 &  1 & -1 	\\
			1  &  2 &  1 &  2 & -1	\\
			2  &  3 &  2 &  3 & -2
		\end{rowequmatbra} \xlongrightarrow[E_{2,1}\left(-1\right)]{E_{3,1}\left(-2\right)}
		\begin{rowequmatbra}{cccc|c}
			1  &  1 &  1 &  1 & -1 	\\
			0  &  1 &  0 &  1 &  0	\\
			0  &  1 &  0 &  1 &  0
		\end{rowequmatbra}
	\end{equation*}
	Si \textbf{moltiplica la seconda riga per -1 e successivamente si somma la seconda riga con la terza}:
	\begin{equation*}
		\begin{rowequmatbra}{cccc|c}
			1  &  1 &  1 &  1 & -1 	\\
			0  &  1 &  0 &  1 &  0	\\
			0  &  1 &  0 &  1 &  0
		\end{rowequmatbra} \xlongrightarrow{E_{3,2}\left(-1\right)}
		\begin{rowequmatbra}{cccc|c}
			1  &  1 &  1 &  1 & -1 	\\
			0  &  1 &  0 &  1 &  0	\\
			0  &  0 &  0 &  0 &  0
		\end{rowequmatbra}
	\end{equation*}
	Si ottiene così la forma ridotta di Gauss.\newline
	
	\noindent
	Il \textcolor{Red3}{terzo passo} è classificare la forma ottenuta. Dalla forma ridotta è possibile dedurre che si è di fronte all'infinito, ovvero esistono un'infinità di soluzioni che dipendono da, in questo caso, due parametri. Le variabili $z$ e $w$ sono libere:
	\begin{equation*}
		\begin{array}{rll}
			1y + 1w = 0 & \longrightarrow & y = -w \\
			\\
			x - w = -1 & \longrightarrow & x = w - 1
		\end{array}
	\end{equation*}
	Quindi il vettore soluzione sarà composto in questo modo:
	\begin{equation*}
		soluzione = \begin{bmatrix}
			w-1 \\
			-w 	\\
			w	\\
			z
		\end{bmatrix}
	\end{equation*}\newpage

	\subsection{Sistemi con parametri}\label{Sistemi con parametri}
	
	Può accadere che l'eliminazione di Gauss si complichi tramite l'inserimento di un parametro. Niente panico, l'esercizio rimane identico, si esegue una classica eliminazione di Gauss con le operazioni elementari. L'unica differenza è un parametro che deve essere considerato come un'incognita. Infine, si esplicita quali valori può (o non può) assumere il parametro.
	
	\subsubsection{Esercizio}
	
	Dato il sistema:
	\begin{equation*}
		\begin{cases}
			ty = 1 \\
			x + y + tz = 2 \\
			2x + ty + z = 0
		\end{cases}
	\end{equation*}
	Sapendo che il parametro $t \in \mathbb{R}$, si risolva l'esercizio con l'eliminazione di Gauss, determinando per quali parametri $t$ non esiste soluzione.\newline
	
	\noindent
	Il \textcolor{Red3}{primo passo} è scrivere la matrice aumentata. Essa è banale da comporre, consiste nello scrivere i coefficienti di ogni variabile ($x,y,z$) in una matrice e aggiungere una colonna sulla destra in cui ci sono i valori risultanti. Si passa all'atto pratico:
	\begin{equation*}
		\begin{rowequmatbra}{ccc|c}
			0  &  t &  0 &  1 \\
			1  &  1 &  t &  2 \\
			2  &  t &  1 &  0
		\end{rowequmatbra}
	\end{equation*}
	Il \textcolor{Red3}{secondo passo} è eseguire alcune considerazioni sulla forma che si vuole ottenere e procedere con le varie operazioni. L'obbiettivo è quello di ottenere una matrice uni-triangolare superiore\footnote{Una matrice uni-triangolare superiore è una forma particolare in cui i valori sotto alla diagonale principale sono nulli, cioè uguale a zero}. L'obbiettivo di questo paragrafo è evidenziare le differenze con gli altri tipi di esercizi, quindi non si mostrano i passaggi specifici. Le operazioni eseguite sono: la \textbf{moltiplicazione della seconda riga per $\frac{2-t}{t}$ e somma della seconda riga con la terza}; la \textbf{moltiplicazione della seconda riga per $\frac{1}{t}$}; la \textbf{moltiplicazione della terza riga per $\frac{1}{1-2t}$}.
	\begin{equation*}
		\begin{rowequmatbra}{ccc|c}
			0  &  t &  0 &  1 \\
			1  &  1 &  t &  2 \\
			2  &  t &  1 &  0
		\end{rowequmatbra} \xlongrightarrow{E_{3,2}\left(\frac{2-t}{t}\right)} \left[...\right] \xlongrightarrow{E_{2}\left(\frac{1}{t}\right)} \left[...\right] \xlongrightarrow{E_{3}\left(\frac{1}{1-2t}\right)}
		\begin{rowequmatbra}{ccc|c}
			1  &  1 &  t &  2 			\\ [0.3em]
			0  &  1 &  0 &  \frac{1}{t} \\ [0.3em]
			0  &  0 &  1 &  \frac{2-5t}{t\left(1-2t\right)} 
		\end{rowequmatbra}
	\end{equation*}
	Il \textcolor{Red3}{terzo passo} è capire quali soluzioni non sono ammesse. Per farlo, basta guardare le operazioni elementari applicate e, considerando le frazioni, cercare di escludere quei valori di $t$ che renderebbero impossibile la risoluzione. In questo caso, la $t$ non può essere zero poiché le frazioni sarebbero impossibili da risolvere e non può essere $\frac{1}{2}$ perché la frazione $\frac{2-5t}{t\left(1-2t\right)}$ sarebbe impossibile da risolvere. Guardando le operazioni elementari, non può essere zero a causa della prima e seconda operazione elementare, e non può essere $\frac{1}{2}$ a causa della terza operazione elementare. Più formalmente, è possibile scrivere:
	\begin{equation*}
		t \in \mathbb{R} \setminus \left\{0, \dfrac{1}{2}\right\}
	\end{equation*}
	
	\newpage

	\section{Eliminazione di Gauss-Jordan}\label{Eliminazione di Gauss-Jordan}
	
	\subsection{Algoritmo matrici quadrate}
	
	L'eliminazione di Gauss-Jordan mira ad \textcolor{Red3}{\textbf{ottenere}}, tramite le classiche tre operazioni elementari dell'EG (paragrafo~\ref{EG: 3 operazioni}), la corrispondente \textcolor{Red3}{\textbf{matrice inversa}}.\newline
	
	\noindent
	Per ottenerla, si affianca a destra una matrice identità\footnote{Matrice identità: matrice con valori pari ad 1 sulla diagonale principale e valori nulli nelle altre posizioni}, si applicano le operazioni elementare dell'EG così da ottenere una forma ridotta a sinistra e infine si esegue un'eliminazione all'indietro così da ottenere a sinistra una matrice identità e a destra la matrice inversa dell'originaria.
	
	\subsection{Algoritmo matrici con numero di righe e colonne diverso}
	
	Nel caso in cui la matrice non sia quadrata, al fianco della matrice originaria si affianca a destra sempre una matrice identità ma in questo caso il numero di colonne sarà diverso da quella originaria. A questo punto si procede con la classica EG. Una volta ottenuta la forma ridotta di Gauss, si utilizzano le matrici ridotte per ottenere $n$ sistemi da risolvere con l'eliminazione di Gauss. In cui $n$ indica il numero di colonne nella (ex) matrice identità a destra. Si veda l'esercizio per comprendere meglio.
	
	\subsection{Esercizi}
	
	\subsubsection{Matrice quadrata}
	
	Data la matrice quadrata:
	\begin{equation*}
		A = \begin{bmatrix}
			1 & -1 & 3 \\
			1 &  1 & 2 \\
			2 &  0 & 7
		\end{bmatrix}
	\end{equation*}
	La \textcolor{Red3}{prima operazione} è aumentare la matrice scrivendo l'identità a destra:
	\begin{equation*}
		\left[A | I_{3}\right] = \begin{rowequmatbra}{ccc|ccc}
			1 & -1 & 3 & 1 & 0 & 0 \\
			1 &  1 & 2 & 0 & 1 & 0 \\
			2 &  0 & 7 & 0 & 0 & 1
		\end{rowequmatbra}
	\end{equation*}
	La \textcolor{Red3}{seconda operazione} è applicare il classico algoritmo dell'eliminazione di Gauss, ma considerando tre colonne a destra, invece di una. Quindi, effettivamente vi è il calcolo di tre sistemi contemporaneamente! La prima operazione che viene eseguita è la \textbf{moltiplicazione della prima riga per -1 e la somma della prima riga con la seconda}. Successivamente, viene \textbf{moltiplicata la prima riga per -2 e sommata la prima riga con la seconda}:
	\begin{equation*}
		\begin{rowequmatbra}{ccc|ccc}
			1 & -1 & 3 & 1 & 0 & 0 \\
			1 &  1 & 2 & 0 & 1 & 0 \\
			2 &  0 & 7 & 0 & 0 & 1
		\end{rowequmatbra} \xlongrightarrow[E_{2,1}\left(-1\right)]{E_{3,1}\left(-2\right)}
		\begin{rowequmatbra}{ccc|ccc}
			1 & -1 &  3 &  1 & 0 & 0 \\
			0 &  2 & -1 & -1 & 1 & 0 \\
			0 &  2 &  1 & -2 & 0 & 1
		\end{rowequmatbra}
	\end{equation*}
	Adesso si \textbf{moltiplica la seconda riga per $\frac{1}{2}$}:
	\begin{equation*}
		\begin{rowequmatbra}{ccc|ccc}
			1 & -1 &  3 &  1 & 0 & 0 \\[0.3em]
			0 &  2 & -1 & -1 & 1 & 0 \\[0.3em]
			0 &  2 &  1 & -2 & 0 & 1
		\end{rowequmatbra} \xlongrightarrow{E_{2}\left(-\frac{1}{2}\right)}
		\begin{rowequmatbra}{ccc|ccc}
			1 & -1 &  3 &  1 & 0 & 0 \\[0.3em]
			0 &  1 & -\frac{1}{2} & -\frac{1}{2} & \frac{1}{2} & 0 \\[0.3em]
			0 &  2 &  1 & -2 & 0 & 1
		\end{rowequmatbra}
	\end{equation*}
	Poi si \textbf{moltiplica la seconda riga per -2 e si somma la seconda riga con la terza} e successivamente si \textbf{moltiplica la terza riga per $\frac{1}{2}$}:
	\begin{equation*}
		\begin{rowequmatbra}{ccc|ccc}
			1 & -1 &  3 &  1 & 0 & 0 \\[0.3em]
			0 &  1 & -\frac{1}{2} & -\frac{1}{2} & \frac{1}{2} & 0 \\[0.3em]
			0 &  2 &  1 & -2 & 0 & 1
		\end{rowequmatbra} \xlongrightarrow[E_{3}\left(\frac{1}{2}\right)]{E_{3,2}\left(-2\right)}
		\begin{rowequmatbra}{ccc|ccc}
			1 & -1 &  3 &  1 & 0 & 0 \\[0.3em]
			0 &  1 & -\frac{1}{2} & -\frac{1}{2} & \frac{1}{2} & 0 \\[0.3em]
			0 &  0 &  1 & -\frac{1}{2} & -\frac{1}{2} & \frac{1}{2}
		\end{rowequmatbra}
	\end{equation*}
	La \textcolor{Red3}{terza operazione} si raggiunge quando l'eliminazione di Gauss porta ad avere la matrice ridotta a sinistra. A questo punto, si cerca di avere una matrice identità a sinistra. Quindi, si inizia ad azzerare i valori sopra la diagonale applicando sempre l'EG.\newline
	\textbf{Attenzione!} Si capisce che una matrice è invertibile quando nella forma ridotta di Gauss, i pivot\footnote{I \textbf{pivot} sono i primi elementi diversi da zero a partire da sinistra che hanno tutti zeri sottostanti.} corrispondono al rango massimo di una matrice.
	
	Si \textbf{moltiplica la terza riga per $\frac{1}{2}$ e successivamente si somma la terza riga con la seconda}. Dopodiché si \textbf{moltiplica la terza riga per -3 e poi si somma la terza riga con la prima}:
	\begin{equation*}
		\begin{rowequmatbra}{ccc|ccc}
			1 & -1 &  3 &  1 & 0 & 0 \\[0.3em]
			0 &  1 & -\frac{1}{2} & -\frac{1}{2} & \frac{1}{2} & 0 \\[0.3em]
			0 &  0 &  1 & -\frac{1}{2} & -\frac{1}{2} & \frac{1}{2}
		\end{rowequmatbra} \xlongrightarrow[E_{1,3}\left(-3\right)]{E_{2,3}\left(\frac{1}{2}\right)}
		\begin{rowequmatbra}{ccc|ccc}
			1 & -1 &  0 &  \frac{5}{2} &  \frac{3}{2} & -\frac{3}{2} \\[0.3em]
			0 &  1 &  0 & -\frac{3}{4} &  \frac{1}{4} &  \frac{1}{4} \\[0.3em]
			0 &  0 &  1 & -\frac{1}{2} & -\frac{1}{2} &  \frac{1}{2}
		\end{rowequmatbra}
	\end{equation*}
	Si vuole togliere l'ultimo -1 e quindi si \textbf{moltiplica la seconda riga per 1 e si somma la seconda riga con la prima}:
	\begin{equation*}
		\begin{rowequmatbra}{ccc|ccc}
			1 & -1 &  0 &  \frac{5}{2} &  \frac{3}{2} & -\frac{3}{2} \\[0.3em]
			0 &  1 &  0 & -\frac{3}{4} &  \frac{1}{4} &  \frac{1}{4} \\[0.3em]
			0 &  0 &  1 & -\frac{1}{2} & -\frac{1}{2} &  \frac{1}{2}
		\end{rowequmatbra} \xlongrightarrow[E_{1,3}\left(-3\right)]{E_{2,3}\left(\frac{1}{2}\right)}
		\begin{rowequmatbra}{ccc|ccc}
			1 &  0 &  0 &  \frac{7}{4} &  \frac{7}{4} & -\frac{5}{4} \\[0.3em]
			0 &  1 &  0 & -\frac{3}{4} &  \frac{1}{4} &  \frac{1}{4} \\[0.3em]
			0 &  0 &  1 & -\frac{1}{2} & -\frac{1}{2} &  \frac{1}{2}
		\end{rowequmatbra}
	\end{equation*}
	L'esercizio termina qui poiché a sinistra si ha la matrice identità e a destra la matrice inversa di $A$. Inoltre, in questo caso, la matrice inversa è bilatera, ovvero che se viene moltiplicata per $A$, il risultato è una matrice identità. Quindi, supponendo che la matrice inversa sia identificata con la lettera $B$:
	\begin{equation*}
		A \cdot B = B \cdot A = I_{3}
	\end{equation*}\newpage

	\subsubsection{Matrice con numero di righe e colonne diverso}
	
	Nelle matrici non quadrate, l'inversa può essere soltanto destra o sinistra. Nel caso in cui la matrice abbia il \textbf{numero di colonne maggiore al numero di righe}, allora si cerca l'\textbf{inversa destra}. Al contrario, se la matrice ha il numero di \textbf{righe maggiore al numero di colonne}, allora si cerca l'\textbf{inversa sinistra}.\newline
	
	\noindent
	Data la matrice:
	\begin{equation*}
		B = \begin{bmatrix}
			3 & 6 & -3 \\
			2 & 5 &  1
		\end{bmatrix}
	\end{equation*}
	Il \textcolor{Red3}{primo passo} è affiancare la matrice identità:
	\begin{equation*}
		\left[B | I_{2}\right] = \begin{rowequmatbra}{ccc|cc}
			3 & 6 & -3 & 1 & 0 \\
			2 & 5 &  1 & 0 & 1 
		\end{rowequmatbra}
	\end{equation*}
	Il \textcolor{Red3}{secondo passo} è procedere con l'eliminazione di Gauss. Si \textbf{moltiplica la prima riga per $\frac{1}{3}$}:
	\begin{equation*}
		\begin{rowequmatbra}{ccc|cc}
			3 & 6 & -3 & 1 & 0 \\ [0.3em]
			2 & 5 &  1 & 0 & 1
		\end{rowequmatbra} \xlongrightarrow{E_{1}\left(\frac{1}{3}\right)}
		\begin{rowequmatbra}{ccc|cc}
			1 & 2 & -1 & \frac{1}{3} & 0 \\ [0.3em]
			2 & 5 &  1 &           0 & 1
		\end{rowequmatbra}
	\end{equation*}
	Si \textbf{moltiplica la prima riga per -2 e si somma la prima riga con la seconda}:
	\begin{equation*}
		\begin{rowequmatbra}{ccc|cc}
			1 & 2 & -1 & \frac{1}{3} & 0 \\ [0.3em]
			2 & 5 &  1 &           0 & 1
		\end{rowequmatbra} \xlongrightarrow{E_{2,1}\left(-2\right)}
		\begin{rowequmatbra}{ccc|cc}
			1 & 2 & -1 & \frac{1}{3} & 0 \\ [0.3em]
			0 & 1 &  3 &           0 & 1
		\end{rowequmatbra}
	\end{equation*}
	Il \textcolor{Red3}{terzo passo} è utilizzare le forme ridotta per scrivere i due sistemi e risolverli. Quindi:
	\begin{gather*}
		\begin{rowequmatbra}{ccc|cc}
			1 & 2 & -1 & \frac{1}{3} & 0 \\ [0.3em]
			0 & 1 &  3 &           0 & 1
		\end{rowequmatbra} \\
		\\
		\begin{array}{lllll}
			\text{Primo sistema}	& \rightarrow & \begin{rowequmatbra}{ccc}
				1 & 2 & -1 \\ [0.3em]
				0 & 1 &  3
			\end{rowequmatbra} \begin{rowequmatbra}{c}
				\frac{1}{3}  \\ [0.3em]
				-\frac{2}{3}
			\end{rowequmatbra} & \rightarrow & \begin{cases}
				a_{1} + 2b_{1} - c_{1} = \frac{1}{3} \\
				b_{1} + 3c_{1} = -\frac{2}{3}
			\end{cases} \\
			&&&& \\
			\text{Secondo sistema}	& \rightarrow & \begin{rowequmatbra}{ccc}
				1 & 2 & -1 \\
				0 & 1 &  3
			\end{rowequmatbra} \begin{rowequmatbra}{c}
				0 \\
				1
			\end{rowequmatbra} & \rightarrow & \begin{cases}
				a_{2} + 2b_{2} - c_{2} = 0 \\
				b_{2} + 3c_{2} = 1
			\end{cases}
		\end{array}
	\end{gather*}
	Adesso si risolvono i sistemi, notando subito che sia $c_{1}$ che $c_{2}$ sono uguali a zero. Infatti, applicando Gauss ci si accorgerebbe che entrambi i sistemi hanno infinite soluzioni, ovvero dipendono dal parametro $c$. Quindi:
	\begin{equation*}
		\begin{array}{lllll}
			\text{Primo sistema}	& \rightarrow & \begin{cases}
				a_{1} = \frac{5}{3}  \\
				b_{1} = -\frac{2}{3} \\
				c_{1} = 0
			\end{cases} & \rightarrow & \begin{rowequmatbra}{c}
				\frac{5}{3}  \\ [0.3em]
				-\frac{2}{3} \\ [0.3em]
				0
			\end{rowequmatbra} \\
			\\
			\text{Secondo sistema}	& \rightarrow & \begin{cases}
				a_{2} = -2 \\
				b_{2} =  1  \\
				c_{2} =  0
			\end{cases} & \rightarrow & \begin{rowequmatbra}{c}
				-2 \\ [0.3em]
				 1 \\ [0.3em]
				 0
			\end{rowequmatbra}
		\end{array}
	\end{equation*}
	Il \textcolor{Red3}{quarto passo} è scrivere la matrice inversa, in questo caso la matrice inversa destra:
	\begin{equation*}
		R = \begin{rowequmatbra}{cc}
			\frac{5}{3}	 & -2 \\ [0.3em]
			-\frac{2}{3} &  1 \\ [0.3em]
			0			 &  0
		\end{rowequmatbra}
	\end{equation*}
	Quindi, la soluzione è:
	\begin{equation*}
		B \cdot R = I_{2}
	\end{equation*}\newpage

	\subsection{Sistemi con parametri}
	
	Lo svolgimento di questa tipologia di esercizi è identica al paragrafo~\ref{Sistemi con parametri}. L'unica differenza è alla fine, quando è necessario verificare se la matrice è invertibile:
	\begin{itemize}
		\item Se il rango massimo della matrice corrisponde al numero di colonne dominanti (primo esercizio), allora è invertibile;
		\item Grazie al determinante, ma verrà introdotto più avanti;	
		\item Altri modi, ma sono troppo laboriosi.
	\end{itemize}	

	\subsubsection{Esercizio}
	
	Data la matrice:
	\begin{equation*}
		A = \begin{bmatrix}
			k &  k-1 &   k \\
			0 & 2k-2 &   0 \\
			1 &  k-1 & 2-k
		\end{bmatrix}
	\end{equation*}
	Per quali valori di $k$ la matrice $A$ ammette l'inversa.\newline
	
	\noindent
	Il \textcolor{Red3}{primo passo} è l'esecuzione dell'eliminazione di Gauss. Quindi, si \textbf{moltiplica la prima riga per $-\frac{1}{k}$ e poi si somma la prima riga con la terza}:
	\begin{equation*}
		\begin{rowequmatbra}{ccc}
			k &  k-1 &   k \\ [0.3em]
			0 & 2k-2 &   0 \\ [0.3em]
			1 &  k-1 & 2-k
		\end{rowequmatbra} \xlongrightarrow{E_{3,1}\left(-\frac{1}{k}\right)}
		\begin{rowequmatbra}{ccc}
			k &  k-1 &   k \\ [0.3em]
			0 & 2k-2 &   0 \\ [0.3em]
			0 &  \frac{\left(k-1\right)^{2}}{k} & 1-k
		\end{rowequmatbra}
	\end{equation*}
	Poi si \textbf{moltiplica la seconda riga per $\frac{\left(k-1\right)^{2}}{k\left(2-2k\right)}$ e poi si moltiplica la seconda riga per la terza}:
	\begin{equation*}
		\begin{rowequmatbra}{ccc}
			k &  k-1 &   k \\ [0.3em]
			0 & 2k-2 &   0 \\ [0.3em]
			1 &  \frac{\left(k-1\right)^{2}}{k} & 1-k
		\end{rowequmatbra} \xlongrightarrow{E_{3,2}\left(\frac{\left(k-1\right)^{2}}{k\left(2-2k\right)}\right)}
		\begin{rowequmatbra}{ccc}
			k &  k-1 &   k \\ [0.3em]
			0 & 2k-2 &   0 \\ [0.3em]
			0 &    0 & 1-k
		\end{rowequmatbra}
	\end{equation*}
	Poi si \textbf{moltiplica la prima riga per $\frac{1}{k}$}, si \textbf{moltiplica la seconda riga per $\frac{1}{2k-2}$} e infine si \textbf{moltiplica la terza riga per $\frac{1}{1-k}$}:
	\begin{equation*}
		\begin{rowequmatbra}{ccc}
			k &  k-1 &   k \\ [0.3em]
			0 & 2k-2 &   0 \\ [0.3em]
			0 &    0 & 1-k
		\end{rowequmatbra} \xlongrightarrow{E_{1}\left(\frac{1}{k}\right)} \left[...\right]
		\xlongrightarrow{E_{2}\left(\frac{1}{2k-2}\right)} \left[...\right]
		\xlongrightarrow{E_{3}\left(\frac{1}{1-k}\right)}
		\begin{rowequmatbra}{ccc}
			1 &  \frac{k-1}{k} 	&   1 \\ [0.3em]
			0 &    			1	&   0 \\ [0.3em]
			0 &    			0	&   1
		\end{rowequmatbra}
	\end{equation*}
	Il \textcolor{Red3}{secondo passo} è verificare per quali valori la matrice non è risolvibile. In questo caso per $0$ e per $1$ perché nelle operazioni elementari (la seconda) si avrebbe una frazione impossibile da risolvere (zero al denominatore):
	\begin{equation*}
		k \in \mathbb{R} \setminus \left\{0,1\right\}
	\end{equation*}
	Dato che il rango della matrice è uguale a $3$, allora la matrice è invertibile per ogni valore di $k$ escluso lo zero e l'uno. Infatti, grazie all'algoritmo di Gauss-Jordan è possibile ottenere la matrice originaria.\newpage
	
	\section{Decomposizione LU}
	
	\subsection{Algoritmo}
	
	Una matrice $A$, in generale $n \times m$, è possibile riscriverla come il prodotto tra due matrici:
	\begin{equation*}
		A = L U
	\end{equation*}
	In cui $L$ è una matrice triangolare inferiore invertibile e $U$ la matrice in forma ridotta (grazie a EG) di $A$.\newline
	
	\noindent
	La decomposizione LU classica lavora \textbf{senza scambi di righe}. In questo caso, data una matrice, si inizia con la classica eliminazione di Gauss. La forma ridotta corrisponderà ad $U$. A quel punto, si moltiplicheranno tutte le operazioni elementari eseguite mettendole in ordine decrescente, quindi dalla più recente alla più vecchia, e così facendo si otterrà una matrice. Quest'ultima, dovrà essere invertita così da ottenere la matrice $L$. Per farlo si utilizza ovviamente Gauss-Jordan.
	
	\subsection{Parametrizzazione}
	
	Nel caso della parametrizzazione di una matrice, l'esercizio non si discosta molto da quelli classici. Viene fornita una matrice che presenta al suo interno un parametro. L'esercizio si svolge come una decomposizione LU normale ma contando il parametro come un numero reale.
	
	\subsection{Esercizi}
	
	\subsubsection{Decomposizione LU}\label{Decomposizione LU ex}
	
	Data la matrice:
	\begin{equation*}
		A = \begin{bmatrix}
			1 & -4 & 1 \\
			2 & -6 & 5 \\
			1 & -2 & 5
		\end{bmatrix}
	\end{equation*}
	Il \textcolor{Red3}{primo passo} è eseguire l'eliminazione di Gauss. Quindi, la prima operazione $OP_{1}$ è la \textbf{moltiplicazione della prima riga per -2 e si somma la prima riga con la seconda}:
	\begin{equation*}
		\begin{rowequmatbra}{ccc}
			1 & -4 & 1 \\
			2 & -6 & 5 \\
			1 & -2 & 5
		\end{rowequmatbra} \xlongrightarrow{E_{2,1}\left(-2\right)}
		\begin{rowequmatbra}{ccc}
			1 & -4 & 1 \\
			0 &  2 & 3 \\
			1 & -2 & 5
		\end{rowequmatbra}
	\end{equation*}
	Poi la seconda operazione $OP_{2}$ è la \textbf{moltiplicazione della prima riga per -1 e la somma della rima riga con la terza}:
	\begin{equation*}
		\begin{rowequmatbra}{ccc}
			1 & -4 & 1 \\
			0 &  2 & 3 \\
			1 & -2 & 5
		\end{rowequmatbra} \xlongrightarrow{E_{3,1}\left(-1\right)}
		\begin{rowequmatbra}{ccc}
			1 & -4 & 1 \\
			0 &  2 & 3 \\
			0 &  2 & 4
		\end{rowequmatbra}
	\end{equation*}
	Poi la terza operazione $OP_{3}$ è la \textbf{moltiplicazione della seconda riga per $\frac{1}{2}$ e la somma della rima riga con la terza}:
	\begin{equation*}
		\begin{rowequmatbra}{ccc}
			1 & -4 & 1 \\
			0 &  2 & 3 \\
			0 &  2 & 4
		\end{rowequmatbra} \xlongrightarrow{E_{2}\left(\frac{1}{2}\right)}
		\begin{rowequmatbra}{ccc}
			1 & -4 & 1 \\ [0.3em]
			0 &  1 & \frac{3}{2} \\ [0.3em]
			0 &  2 & 4
		\end{rowequmatbra}
	\end{equation*}
	Poi la quarta operazione $OP_{4}$ è la \textbf{moltiplicazione della seconda riga per $-2$ e la somma della seconda riga con la terza}:
	\begin{equation*}
		\begin{rowequmatbra}{ccc}
			1 & -4 & 1 \\
			0 &  1 & \frac{3}{2} \\
			0 &  2 & 4
		\end{rowequmatbra} \xlongrightarrow{E_{2}\left(\frac{1}{2}\right)}
		\begin{rowequmatbra}{ccc}
			1 & -4 & 1 \\ [0.3em]
			0 &  1 & \frac{3}{2} \\ [0.3em]
			0 &  0 & 1
		\end{rowequmatbra} = U
	\end{equation*}
	La forma ridotta di Gauss corrisponde alla matrice $U$. Il \textcolor{Red3}{secondo passo} è moltiplicare \textbf{in ordine decrescente d'esecuzione} (dalla più recente alla più vecchia) tutte le matrici ottenute con le operazioni elementari. Quindi si riprendono le matrici $OP$:
	\begin{equation*}
		\begin{array}{lllll}
			OP_{1} & = & \begin{rowequmatbra}{ccc}
				1 & -4 & 1 \\
				0 &  2 & 3 \\
				1 & -2 & 5
			\end{rowequmatbra} \hspace{2em} OP_{2} & = & \begin{rowequmatbra}{ccc}
				1 & -4 & 1 \\
				0 &  2 & 3 \\
				0 &  2 & 4
			\end{rowequmatbra} \\
			\\
			OP_{3} & = & \begin{rowequmatbra}{ccc}
				1 & -4 & 1 \\
				0 &  1 & \frac{3}{2} \\
				0 &  2 & 4
			\end{rowequmatbra} \hspace{2em}	OP_{4} & = & \begin{rowequmatbra}{ccc}
				1 & -4 & 1 \\
				0 &  1 & \frac{3}{2} \\
				0 &  0 & 1
			\end{rowequmatbra}
		\end{array}
	\end{equation*}
	Adesso si moltiplicano le matrici e si ottiene la matrice risultante $B$ che corrisponde alla matrice $L^{-1}$, cioè invertita:
	\begin{equation*}
		L^{-1} = B = OP_{4} \times OP_{3} \times OP_{2} \times OP_{1} =
		\begin{rowequmatbra}{ccc}
			 1 &  		   0  & 0 \\ [0.3em]
			-1 &  \frac{1}{2} & 0 \\ [0.3em]
			 1 & 		  -1  & 1
		\end{rowequmatbra}
	\end{equation*}
	Il \textcolor{Red3}{terzo passo} è applicare l'algoritmo di Gauss-Jordan per ottenere la matrice inversa e avere la corrispondente matrice $L$. Si saltano i passaggi poiché concettualmente identici al paragrafo~\ref{Eliminazione di Gauss-Jordan}:
	\begin{equation*}
		L^{-1} = B \xlongrightarrow{\text{Gauss-Jordan}} B^{-1} = L = \begin{bmatrix}
			1 & 0 & 0 \\
			2 & 2 & 0 \\
			1 & 2 & 0
		\end{bmatrix}
	\end{equation*}
	Quindi, il risultato è il seguente:
	\begin{equation*}
		A = LU \longrightarrow \begin{rowequmatbra}{ccc}
			1 & -4 & 1 \\ [0.3em]
			2 & -6 & 5 \\ [0.3em]
			1 & -2 & 5
		\end{rowequmatbra} = \begin{rowequmatbra}{ccc}
			1 & 0 & 0 \\ [0.3em]
			2 & 2 & 0 \\ [0.3em]
			1 & 2 & 0
		\end{rowequmatbra} \times \begin{rowequmatbra}{ccc}
			1 & -4 & 1 \\ [0.3em]
			0 &  1 & \frac{3}{2} \\ [0.3em]
			0 &  0 & 1
		\end{rowequmatbra}
	\end{equation*}
	N.B. c'è un possibile errore di calcolo poiché verificando la moltiplicazione, nella posizione $\left(3,3\right)$ si ottiene il valore $4$ invece del valore $5$.\newpage
	
	\subsubsection{Decomposizione LU con parametrizzazione}
	
	Data la matrice:
	\begin{equation*}
		A_{\alpha} = \begin{bmatrix}
			-1		& 0 	& 1			& -\alpha		& 0				\\
			\alpha 	& 2 	& 4-\alpha	& \alpha^{2}-2	& 0				\\
			0 		& -1	& -2		& \alpha+1		& -\alpha^{2}	\\
			0		& 0		& 0			& 1				& -\alpha
		\end{bmatrix}
	\end{equation*}
	Il \textcolor{Red3}{primo passo} è calcolare la forma ridotta di Gauss. Quindi, la prima operazione $OP_{1}$ è la \textbf{moltiplicazione della prima riga per -1}:
	\begin{equation*}
		\begin{bmatrix}
			-1		& 0 	& 1			& -\alpha		& 0				\\
			\alpha 	& 2 	& 4-\alpha	& \alpha^{2}-2	& 0				\\
			0 		& -1	& -2		& \alpha+1		& -\alpha^{2}	\\
			0		& 0		& 0			& 1				& -\alpha
		\end{bmatrix} \xlongrightarrow{E_{1}\left(-1\right)}
		\begin{rowequmatbra}{ccccc}
			1		& 0 	& -1		& \alpha		& 0				\\
			\alpha 	& 2 	& 4-\alpha	& \alpha^{2}-2	& 0				\\
			0 		& -1	& -2		& \alpha+1		& -\alpha^{2}	\\
			0		& 0		& 0			& 1				& -\alpha
		\end{rowequmatbra}
	\end{equation*}
	La seconda operazione $OP_{2}$ è la \textbf{moltiplicazione della prima riga per $-\alpha$ e la somma della prima riga con la seconda}:
	\begin{equation*}
		\begin{rowequmatbra}{ccccc}
			1		& 0 	& -1		& \alpha		& 0				\\
			\alpha 	& 2 	& 4-\alpha	& \alpha^{2}-2	& 0				\\
			0 		& -1	& -2		& \alpha+1		& -\alpha^{2}	\\
			0		& 0		& 0			& 1				& -\alpha
		\end{rowequmatbra} \xlongrightarrow{E_{1}\left(-1\right)}
		\begin{rowequmatbra}{ccccc}
			1		& 0 	& -1		& \alpha		& 0				\\
			0		& 2 	& 4			& -2			& 0				\\
			0 		& -1	& -2		& \alpha+1		& -\alpha^{2}	\\
			0		& 0		& 0			& 1				& -\alpha
		\end{rowequmatbra}
	\end{equation*}
	La terza operazione $OP_{3}$ è la \textbf{moltiplicazione della seconda riga per $\frac{1}{2}$}:
	\begin{equation*}
		\begin{rowequmatbra}{ccccc}
			1		& 0 	& -1		& \alpha		& 0				\\
			0		& 2 	& 4			& -2			& 0				\\
			0 		& -1	& -2		& \alpha+1		& -\alpha^{2}	\\
			0		& 0		& 0			& 1				& -\alpha
		\end{rowequmatbra} \xlongrightarrow{E_{1}\left(-1\right)}
		\begin{rowequmatbra}{ccccc}
			1		& 0 	& -1		& \alpha		& 0				\\
			0		& 1 	& 2			& -1			& 0				\\
			0 		& -1	& -2		& \alpha+1		& -\alpha^{2}	\\
			0		& 0		& 0			& 1				& -\alpha
		\end{rowequmatbra}
	\end{equation*}
	La quarta operazione $OP_{4}$ è la \textbf{moltiplicazione della seconda riga per $1$ e la somma della seconda riga con la terza}:
	\begin{equation*}
		\begin{rowequmatbra}{ccccc}
			1		& 0 	& -1		& \alpha		& 0				\\
			0		& 1 	& 2			& -1			& 0				\\
			0 		& -1	& -2		& \alpha+1		& -\alpha^{2}	\\
			0		& 0		& 0			& 1				& -\alpha
		\end{rowequmatbra} \xlongrightarrow{E_{1}\left(-1\right)}
		\begin{rowequmatbra}{ccccc}
			1		& 0 	& -1		& \alpha		& 0				\\
			0		& 1 	& 2			& -1			& 0				\\
			0 		& 0		& 0			& \alpha		& -\alpha^{2}	\\
			0		& 0		& 0			& 1				& -\alpha
		\end{rowequmatbra}
	\end{equation*}
	La quinta operazione $OP_{5}$ è la \textbf{moltiplicazione della terza riga per $\frac{1}{\alpha}$}. La sesta operazione $OP_{6}$ è la \textbf{moltiplicazione della terza riga per $-1$ e la somma della terza riga con la quarta}:
	\begin{equation*}
		\begin{rowequmatbra}{ccccc}
			1		& 0 	& -1		& \alpha		& 0				\\
			0		& 1 	& 2			& -1			& 0				\\
			0 		& 0		& 0			& \alpha		& -\alpha^{2}	\\
			0		& 0		& 0			& 1				& -\alpha
		\end{rowequmatbra} \xlongrightarrow[E_{4,3}\left(-1\right)]{E_{3}\left(\frac{1}{\alpha}\right)}
		\begin{rowequmatbra}{ccccc}
			1		& 0 	& -1		& \alpha		& 0				\\
			0		& 1 	& 2			& -1			& 0				\\
			0 		& 0		& 0			& 1				& -\alpha		\\
			0		& 0		& 0			& 0				& 0
		\end{rowequmatbra}
	\end{equation*}
	Si ottiene così la forma ridotta di Gauss. Guardando le operazioni, si può affermare che la matrice può essere risolta solamente se $\alpha$ non è uguale a zero:
	\begin{equation*}
		\alpha \in \mathbb{R} \setminus \left\{0\right\}
	\end{equation*}
	Da adesso in poi l'esercizio diventa come nel precedente paragrafo (\ref{Decomposizione LU ex}).
\end{document}