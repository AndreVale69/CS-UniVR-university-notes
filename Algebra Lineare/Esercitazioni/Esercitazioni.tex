\documentclass[a4paper]{article}
\usepackage[T1]{fontenc}			% pacchetto per \chapter
\usepackage[italian]{babel}
\usepackage[italian]{isodate}  		% formato delle date in italiano
\usepackage{graphicx}				% gestione delle immagini
\usepackage{amsfonts}
\usepackage{booktabs}				% tabelle di qualità superiore
\usepackage{amsmath}				% pacchetto matematica
\usepackage{mathtools}				% per sottolineare sotto le equazioni
\usepackage{stmaryrd} 				% per '\llbracket' e '\rrbracket'
\usepackage{amsthm}					% teoremi migliorati
\usepackage{enumitem}				% gestione delle liste
\usepackage{pifont}					% pacchetto con elenchi carini
\usepackage{enumitem}				% pacchetto per elenchi con lettere dell'alfabeto
\usepackage{cancel}					% per cancellare delle espressioni matematiche


\usepackage[x11names]{xcolor}		% pacchetto colori RGB
% Link ipertestuali per l'indice
\usepackage{xcolor}
\usepackage[linkcolor=black, citecolor=blue, urlcolor=cyan]{hyperref}
\hypersetup{
	colorlinks=true
}

%\usepackage{showframe}				% visualizzazione bordi
%\usepackage{showkeys}				% visualizzazione etichetta

\newtheorem{theorem}{\textcolor{Red3}{\underline{Teorema}}}
\newtheorem{lemma}{Lemma}
\renewcommand{\qedsymbol}{QED}
\newcommand{\exec}[1]{\llbracket #1\:\rrbracket}
\newcommand{\dquotes}[1]{``#1''}
\newcommand{\longline}{\noindent\rule{\textwidth}{0.4pt}}

\newenvironment{rowequmat}[1]{\left(\array{@{}#1@{}}}{\endarray\right)}
\newenvironment{rowequmatbra}[1]{\left[\array{@{}#1@{}}}{\endarray\right]}

\begin{document}
	\author{VR443470}
	\title{Esercitazioni Algebra Lineare}
	\date{\printdayoff\today}
	\maketitle
	
	\newpage
	
	% indice
	\tableofcontents
	
	\newpage
	
	\section{Basi}
	
	\subsection{Somma e trasposte}
	
	I classici esercizi di Algebra Lineare prevedono varie operazioni sulle matrici. Partendo dalle basi, si introducono le operazioni di somma e trasposizione.\newline
	
	\noindent
	Date $3$ matrici $A, B, C$:
	\begin{equation*}
		A = \begin{bmatrix}
			   1 & -i & 3 \\
			-2+i &  5 & i2
		\end{bmatrix} \hspace{2em}
		B = \begin{bmatrix}
			3i	& 2		\\
			4	& -i 	\\
			2-i	& -1
		\end{bmatrix} \hspace{2em}
		C = \begin{bmatrix}
			-2	& 3i	& 2-i	\\
			4i	& 1		& 0
		\end{bmatrix}
	\end{equation*}
	La \textcolor{Red3}{prima operazione} da eseguire è la classificazione delle matrici. In questo caso, le matrici hanno le seguenti dimensioni:
	\begin{equation*}
		A\in\mathbb{M}_{2\times3} \hspace{2em} B\in\mathbb{M}_{3\times2} \hspace{2em} C\in\mathbb{M}_{2\times3}
	\end{equation*}
	La \textcolor{Red3}{seconda operazione} da eseguire è controllare se è possibile eseguire l'operazione richiesta dall'esercizio. In questo caso, viene chiesta la somma. Per eseguire quest'ultima (vale lo stesso per la sottrazione), le dimensioni delle matrici devono essere tutte \textbf{identiche}. Dato che in questo caso la matrice $B \left(3\times2\right)$ differisce di dimensione rispetto alle due matrici $A,C \left(2\times3\right)$, è necessario fare qualcosa per eseguire l'operazione di somma.\newline
	
	\noindent
	Dato che è ancora l'inizio, non verranno effettuate manipolazioni complesse. Quindi, si supponga di eseguire questa operazione di somma/sottrazione:
	\begin{equation*}
		2A^{T} - 4\overline{B} + 3C^{T}
	\end{equation*}
	Prima di eseguire l'operazione, si ottengono le relative matrici coniugate e trasposte. L'operazione di \textbf{coniugazione} è eseguibile cambiando i segni ai valori complessi (quindi alle $i$). Invece, l'operazione di \textbf{trasposizione} ($T$) inverte le colonne e le righe di una matrice. I risultati sono:
	\begin{equation*}
		\begin{array}{lllllllll}
			A & = \begin{bmatrix}
				1 	& -i	& 3	\\
				-2+i&  5 	& i2
			\end{bmatrix} &
			A^{T} & = \begin{bmatrix}
				1	& -2+i	\\
				-i	& 5		\\
				3	& i2
			\end{bmatrix} &
			2A^{T} & = \begin{bmatrix}
				2	& -4+2i	\\
				-2i	& 10	\\
				6	& 4i
			\end{bmatrix} \\
			%%%%%%%%%%%%%%%%%%%%%%%%%%%%%%%%%%%%%%%%
			&&&&& \\
			%%%%%%%%%%%%%%%%%%%%%%%%%%%%%%%%%%%%%%%%
			B & = \begin{bmatrix}
				3i	& 2		\\
				4	& -i 	\\
				2-i	& -1
			\end{bmatrix} &
			\overline{B} & = \begin{bmatrix}
				-3i	& 2		\\
				4	& i		\\
				2+i	& -1
			\end{bmatrix} &
			4\overline{B} & = \begin{bmatrix}
				-12i	& 8		\\
				16		& 4i	\\
				8+4i	& -4
			\end{bmatrix} \\
			%%%%%%%%%%%%%%%%%%%%%%%%%%%%%%%%%%%%%%%%
			&&&&& \\
			%%%%%%%%%%%%%%%%%%%%%%%%%%%%%%%%%%%%%%%%
			C & = \begin{bmatrix}
				-2	& 3i	& 2-i	\\
				4i	& 1		& 0
			\end{bmatrix} &
			C^{T} & = \begin{bmatrix}
				-2	& 4i	\\
				3i	& 1		\\
				2-i	& 0
			\end{bmatrix} &
			3C^{T} & = \begin{bmatrix}
				-6	& 12i	\\
				9i	& 3		\\
				6-3i& 0
			\end{bmatrix}
		\end{array}
	\end{equation*}\newpage
	Adesso è possibile eseguire la sottrazione tra $\alpha = 2A^{T} - 4\overline{B}$ e successivamente la somma tra $\alpha + 3C^{T}$:
	\begin{equation*}
		\alpha = 2A^{T} - 4\overline{B} = \begin{bmatrix}
			2	& -4+2i	\\
			-2i	& 10	\\
			6	& 4i
		\end{bmatrix} - \begin{bmatrix}
			-12i	& 8		\\
			16		& 4i	\\
			8+4i	& -4
		\end{bmatrix} = \begin{bmatrix}
			2+12i	& 12+2i		\\
			-16-2i	& 10-4i		\\
			1-4i	& 4+4i
		\end{bmatrix}
	\end{equation*}
	Si esegue la somma:
	\begin{equation*}
		\alpha + 3C^{T} = \begin{bmatrix}
			2+12i	& 12+2i		\\
			-16-2i	& 10-4i		\\
			1-4i	& 4+4i
		\end{bmatrix} + \begin{bmatrix}
			-6	& 12i	\\
			9i	& 3		\\
			6-3i& 0
		\end{bmatrix} = \begin{bmatrix}
			-4+12i	& -12+14i	\\
			-16+7i	& 13-4i		\\
			7-7i	& 4+4i
		\end{bmatrix}
	\end{equation*}\newpage

	\subsection{(Anti-)Hermitiane e (anti-)simmetriche}
	
	Diamo alcune definizioni per capire come fare gli esercizi:
	\begin{itemize}
		\item È possibile abbreviare letteralmente le operazioni di trasposizione e coniugazione scrivendo \textbf{trasposta-coniugata};
		
		\item Una matrice viene detta \textcolor{Red3}{\textbf{hermitiana}} quando la matrice originaria è uguale alla sua trasposta-coniugata:
		\begin{equation*}
			\overline{\left(A^{T}\right)} = \left(\overline{A}\right)^{T} = A \Longrightarrow A^{H}
		\end{equation*}
		
		\item Una matrice viene detta \textcolor{Red3}{\textbf{anti-hermitiana}} quando la matrice trasposta-coniugata corrisponde alla matrice originaria ma cambiata di segno:
		\begin{equation*}
			\overline{\left(A^{T}\right)} = \left(\overline{A}\right)^{T} = -A \Longrightarrow \text{ anti-hermitiana}
		\end{equation*}
		
		\item Una matrice viene detta \textcolor{Red3}{\textbf{simmetrica}} quando la matrice originaria è uguale alla sua trasposta:
		\begin{equation*}
			A = A^{T} \Longrightarrow \text{ simmetrica}
		\end{equation*}
		
		\item Una matrice viene detta \textcolor{Red3}{\textbf{anti-simmetrica}} quando la sua trasposta corrisponde alla matrice originaria ma cambiata di segno:
		\begin{equation*}
			-A = A^{T} \Longrightarrow \text{ anti-simmetrica}
		\end{equation*}
	\end{itemize}
	Prendendo come \textcolor{Green4}{\textbf{esempio}} le tre matrici $A,B,C$:
	\begin{equation*}
		A = \begin{bmatrix}
			2i 	& 3		\\
			-3	& i
		\end{bmatrix} \hspace{2em}
		B = \begin{bmatrix}
			i 	& 2		\\
			2	& i
		\end{bmatrix} \hspace{2em}
		C = \begin{bmatrix}
			-1 	& i3	\\
			-i3	& 1
		\end{bmatrix}
	\end{equation*}
	Si eseguono le rispettive operazioni di trasposizione e coniugazione:
	\begin{equation*}
		\begin{array}{lllllllll}
			A & = \begin{bmatrix}
				2i 	& 3		\\
				-3	& i
			\end{bmatrix} &
			A^{T} & = \begin{bmatrix}
				2i	& -3	\\
				3	& i
			\end{bmatrix} &
			\overline{A^{T}} & = \begin{bmatrix}
				-2i	& -3	\\
				3	& -i
			\end{bmatrix} \\
			%%%%%%%%%%%%%%%%%%%%%%%%%%%%%%%%%%%%%%%%
			&&&&& \\
			%%%%%%%%%%%%%%%%%%%%%%%%%%%%%%%%%%%%%%%%
			B & = \begin{bmatrix}
				i 	& 2		\\
				2	& i
			\end{bmatrix} &
			B^{T} & = \begin{bmatrix}
				i	& 2		\\
				2	& i
			\end{bmatrix} &
			\overline{B^{T}} & = \begin{bmatrix}
				-i	& 2		\\
				2	& -i
			\end{bmatrix} \\
			%%%%%%%%%%%%%%%%%%%%%%%%%%%%%%%%%%%%%%%%
			&&&&& \\
			%%%%%%%%%%%%%%%%%%%%%%%%%%%%%%%%%%%%%%%%
			C & = \begin{bmatrix}
				-1 	& i3	\\
				-i3	& 1
			\end{bmatrix} &
			C^{T} & = \begin{bmatrix}
				-1	& -i3	\\
				i3	& 1
			\end{bmatrix} &
			\overline{C^{T}} & = \begin{bmatrix}
				-1	& i3	\\
				-i3	& 1
			\end{bmatrix}
		\end{array}
	\end{equation*}
	Da questi risultati è possibile notare come $A, B$ \underline{non} siano hermitiane, mentre $C$ lo sia. Inoltre, dalle trasposte è possibile osservare come $A, C$ \underline{non} siano simmetriche, mentre $B$ lo sia. Invece, per verificare l'anti-hermitiana e l'anti-simmetrica, è necessario negare le matrici originarie:
	\begin{equation*}
		-A = \begin{bmatrix}
			-2i	& -3	\\
			3	& -i
		\end{bmatrix} \hspace{2em}
		-B = \begin{bmatrix}
			-i 	& -2	\\
			-2	& -i
		\end{bmatrix} \hspace{2em}
		-C = \begin{bmatrix}
			1 	& -i3	\\
			i3	& -1
		\end{bmatrix}
	\end{equation*}
	Da questi risultati è possibile notare come $B,C$ \underline{non} siano anti-hermitiane, mentre $A$ lo sia. Inoltre, osservando nuovamente le trasposte, è possibile osservare come $A, B$ e $C$ \underline{non} siano anti-simmetriche.\newpage
	
	\subsection{Prodotto tra matrici righe per colonne}
	
	La \textcolor{Red3}{prima operazione} da eseguire per la moltiplicazione tra matrici righe per colonne è la verifica delle righe e delle colonne. Il prodotto tra matrici è ammesso solo se il numero delle colonne del primo operando è uguale al numero delle righe del secondo operando. Per esempio, la seguente operazione è ammessa:
	\begin{equation*}
		A_{m \times n} \cdot B_{n \times l}
	\end{equation*}
	Inoltre, la matrice risultante avrà come dimensione le righe del primo operando e le colonne del secondo. Quindi:
	\begin{equation*}
		C_{m \times l} = A_{m \times n} \cdot B_{n \times l}
	\end{equation*}
	La \textcolor{Red3}{seconda operazione} è la moltiplicazione vera e propria. Per farla, si prende ogni riga del primo operando e si moltiplica per ogni colonna del secondo operando. Dopo la moltiplicazione di una riga per una colonna, si sommano i risultati. Quindi:
	\begin{gather*}
		A_{m,n} = \begin{bmatrix}
			1,1 	& \cdots & 1,n 		\\
			\cdots 	& \cdots & \cdots	\\
			m,1		& \cdots & m,n
		\end{bmatrix} \times
		B_{n,l} = \begin{bmatrix}
			1,1 	& \cdots & 1,l 		\\
			\cdots 	& \cdots & \cdots	\\
			n,1		& \cdots & n,l
		\end{bmatrix} \\
		\\
		C_{1,1} = A_{1,1} \cdot B_{1,1} + \cdots + A_{1,n} \cdot B_{n,1} \\
		\cdots
	\end{gather*}
	Si presenta un \textcolor{Green4}{\textbf{esempio}}. Date due matrici $A,B$:
	\begin{equation*}
		A = \begin{bmatrix}
			1 & 0 &  2 \\
			0 & 3 & -1
		\end{bmatrix} \hspace{2em}
		B = \begin{bmatrix}
			4  	& 1 \\
			-2 	& 2 \\
			0 	& 3
		\end{bmatrix}
	\end{equation*}
	L'operazione di moltiplicazione di righe per colonne è ammessa poiché le righe di $A$ ($2$) sono lo stesso numero delle colonne di $B$ ($2$):
	\begin{equation*}
		C = A \cdot B = \begin{bmatrix}
			4  & 7 \\
			-6 & 3
		\end{bmatrix}
	\end{equation*}
	I calcoli sono banali. Si lasciano qua di seguito i passaggi:
	\begin{equation*}
		\begin{array}{lllllllll}
			C_{1,1} & = & 1 \cdot 4 & + & 0 \cdot -2 & + & 2 \cdot 0 & = & 4	\\
			C_{1,2} & = & 1 \cdot 1 & + & 0 \cdot 2 & + & 2 \cdot 3 & = & 7		\\
			C_{2,1} & = & 0 \cdot 4 & + & 3 \cdot -2 & + & -1 \cdot 0 & = & -6	\\
			C_{2,2} & = & 0 \cdot 1 & + & 3 \cdot 2 & + & -1 \cdot 3 & = & 3
		\end{array}
	\end{equation*}\newpage

	\noindent
	Si presenta un altro \textcolor{Green4}{\textbf{esempio}} ma con i numeri complessi. Date le due matrici $A, B$:
	\begin{equation*}
		A_{2,4} = \begin{bmatrix}
			1+i	& i					& 0		& \overline{3+2i} \\
			-i 	& \overline{-1-3i} 	& 7i	& 6i
		\end{bmatrix} \times
		B_{4,2} = \begin{bmatrix}
			2  				& 4i 	\\
			-3i				& 0 	\\
			\overline{1-i} 	& -2i	\\
			5-i				& \overline{2+i}
		\end{bmatrix}
	\end{equation*}
	Prima di eseguire la moltiplicazione tra righe e colonne si risolvono i coniugati, due nella matrice $A$ e due nella matrice $B$:
	\begin{equation*}
		A_{2,4} = \begin{bmatrix}
			1+i	& i			& 0		& 3-2i \\
			-i 	& -1+3i 	& 7i	& 6i
		\end{bmatrix} \times
		B_{4,2} = \begin{bmatrix}
			2  	& 4i 	\\
			-3i	& 0 	\\
			1+i	& -2i	\\
			5-i	& 2-i
		\end{bmatrix}
	\end{equation*}
	L'operazione di moltiplicazione di righe per colonne è ammessa. Quindi si presenta qui di seguito i calcoli eseguiti (attenzione alle parti immaginarie, si ricorda che $i^{2} = -1$):
	\begin{gather*}
		\begin{array}{llllllllllll}
			C_{1,1} & = & \left(1+i\right) \cdot 2 	& + & i \cdot \left(-3i\right) 		& + & 0 \cdot \left(1+i\right) 	& + & \left(3-2i\right) \cdot \left(5-i\right) 	& = & 18-11i \\
			C_{1,2} & = & \left(1+i\right) \cdot 4i 	& + & i \cdot 0 		& + & 0 \cdot -2i	& + & \left(3-2i\right) \cdot \left(2-i\right)	& = & -3i  \\
			C_{2,1} & = & -i \cdot 2	& + & \left(-1+3i\right) \cdot -3i 	& + & 7i \cdot \left(1+i\right) 	& + & 6i \cdot \left(5-i\right) 		& = & 8 + 38i \\
			C_{2,2} & = & -i \cdot 4i 	& + & \left(-1+3i\right) \cdot 0 	& + & 7i \cdot -2i 	& + & 6i \cdot \left(2-i\right) 		& = & 24 + 12i
		\end{array}\\
		\\
		C = \begin{bmatrix}
			18-11i & -3i \\
			8+38i & 24+12i
		\end{bmatrix}
	\end{gather*}\newpage

	\subsection{Prodotto tra matrici - Casi particolari}
	
	Esistono alcuni casi particolari quando vengono eseguite le moltiplicazioni tra matrici:
	\begin{enumerate}
		\item La \textbf{moltiplicazione tra un vettore riga e un vettore colonna} (\underline{non} viceversa) restituisce solamente un valore, chiamato \textbf{prodotto scalare}. Ovviamente, il numero delle colonne del vettore riga e il numero di righe del vettore colonna devono essere identici:
		\begin{equation*}
			A_{1,n} = \begin{bmatrix}
				\cdots & \cdots & \cdots
			\end{bmatrix} \times
			B_{n,1} = \begin{bmatrix}
				\cdots \\
				\cdots \\
				\cdots
			\end{bmatrix} \rightarrow
			C
		\end{equation*}
	
		\item La \textbf{moltiplicazione tra un vettore colonna e un vettore riga} restituisce una matrice avente il numero di righe pari al vettore colonna e il numero di colonne pari al vettore riga:
		\begin{equation*}
			B_{n,1} = \begin{bmatrix}
				\cdots \\
				\cdots \\
				\cdots
			\end{bmatrix} \times
			A_{1,n} = \begin{bmatrix}
				\cdots & \cdots & \cdots
			\end{bmatrix} \rightarrow
			C_{n, n}
		\end{equation*}
	
		\item La \textbf{moltiplicazione tra una matrice e un vettore colonna} restituisce un vettore colonna. Ovviamente, per applicare questa operazione è necessario che il numero delle colonne della matrice sia uguale al numero di righe del vettore:
		\begin{equation*}
			A_{m,n} = \begin{bmatrix}
				\cdots & \cdots & \cdots \\
				\cdots & \cdots & \cdots \\
				\cdots & \cdots & \cdots
			\end{bmatrix} \times
			B_{n, 1} = \begin{bmatrix}
				\cdots \\
				\cdots \\
				\cdots
			\end{bmatrix} \rightarrow
			C_{m, 1}
		\end{equation*}
	\end{enumerate}\newpage

	\section{Eliminazione di Gauss}
	
	\subsection{Le 3 operazioni}
	
	L'eliminazione di Gauss prevede 3 operazioni principali da applicare per ottenere la forma ridotta (forma finale):
	\begin{enumerate}
		\item Un'equazione può essere moltiplicata per uno scalare non nullo;
		
		\item Un'equazione viene sostituita con la somma tra lei e un'altra equazione, in cui quest'ultima è stata prima moltiplicata per uno scalare non nullo. Quindi, viene scelta un'equazione da moltiplicare per uno scalare non nullo e successivamente viene effettuata la somma tra il risultato della moltiplicazione e l'equazione interessata;
		
		\item Scambio di due equazioni.
	\end{enumerate}

	\subsection{Tipi di soluzione}
	
	Possono esistere tre tipi di soluzione:
	\begin{itemize}
		\item Tipo uno (una sola soluzione). Intuibile dalla forma ridotta (finale) di Gauss poiché l'ultima riga ha solo una variabile con valore positivo;
		
		\item Tipo zero (non esistono soluzioni per il sistema). Intuibile dalla forma ridotta (finale) di Gauss poiché l'ultima riga ha solo variabili nulle;
		
		\item Infinito (le soluzioni del sistema sono infinite). Intuibile dalla forma ridotta (finale) di Gauss poiché l'ultima riga presenta più di una variabile con valore positivo.
	\end{itemize}
	I prossimi paragrafi mostreranno tutte e tre le casistiche.\newpage

	\subsection{Esercizi}
	
	\subsubsection{Tipo uno - Una soluzione}
	
	Dato il seguente sistema:
	\begin{equation*}
		\begin{cases}
			2x + 4y + 4z = 4 \\
			x - z = 1 \\
			-x + 3y + 4z = 2
		\end{cases}
	\end{equation*}
	Si calcola la matrice risultante dopo l'eliminazione di Gauss.\newline
	
	\noindent
	Il \textcolor{Red3}{primo passo} è scrivere la matrice aumentata. Essa è banale da comporre, consiste nello scrivere i coefficienti di ogni variabile ($x,y,z$) in una matrice e aggiungere una colonna sulla destra in cui ci sono i valori risultanti. Si passa all'atto pratico:
	\begin{equation*}
		\begin{rowequmatbra}{ccc|c}
			2  & 4 &  4 & 4 \\
			1  & 0 & -1 & 1 \\
			-1 & 3 &  4 & 2
		\end{rowequmatbra}
	\end{equation*}
	Il \textcolor{Red3}{secondo passo} è eseguire alcune considerazioni sulla forma che si vuole ottenere e procedere con le varie operazioni. L'obbiettivo è quello di ottenere una matrice uni-triangolare superiore\footnote{Una matrice uni-triangolare superiore è una forma particolare in cui i valori sotto alla diagonale principale sono nulli, cioè uguale a zero}. In questo caso, si inizia con lo \textbf{scambio della prima riga con la seconda}:
	\begin{equation*}
		\begin{rowequmatbra}{ccc|c}
			2  & 4 &  4 & 4 \\
			1  & 0 & -1 & 1 \\
			-1 & 3 &  4 & 2
		\end{rowequmatbra} \xlongrightarrow{E_{1,2}}
		\begin{rowequmatbra}{ccc|c}
			1  & 0 & -1 & 1 \\
			2  & 4 &  4 & 4 \\
			-1 & 3 &  4 & 2
		\end{rowequmatbra}
	\end{equation*}
	Si \textbf{moltiplica la seconda riga per lo scalare} $\frac{1}{2}$:
	\begin{equation*}
		\begin{rowequmatbra}{ccc|c}
			1  & 0 & -1 & 1 \\
			2  & 4 &  4 & 4 \\
			-1 & 3 &  4 & 2
		\end{rowequmatbra} \xlongrightarrow{E_{2}\left(\frac{1}{2}\right)}
		\begin{rowequmatbra}{ccc|c}
			1  & 0 & -1 & 1 \\
			1  & 2 &  2 & 2 \\
			-1 & 3 &  4 & 2
		\end{rowequmatbra}
	\end{equation*}
	Si \textbf{moltiplica la prima riga per -1 e successivamente si somma la prima riga con la seconda}:
	\begin{equation*}
		\begin{rowequmatbra}{ccc|c}
			1  & 0 & -1 & 1 \\
			1  & 2 &  2 & 2 \\
			-1 & 3 &  4 & 2
		\end{rowequmatbra} \xlongrightarrow{E_{2,1}\left(-1\right)}
		\begin{rowequmatbra}{ccc|c}
			1  & 0 & -1 & 1 \\
			0  & 2 &  3 & 1 \\
			-1 & 3 &  4 & 2
		\end{rowequmatbra}
	\end{equation*}
	Si \textbf{moltiplica la prima riga per 1 e successivamente si somma la prima riga con la terza}:
	\begin{equation*}
		\begin{rowequmatbra}{ccc|c}
			1  & 0 & -1 & 1 \\
			0  & 2 &  3 & 1 \\
			-1 & 3 &  4 & 2
		\end{rowequmatbra} \xlongrightarrow{E_{3,1}\left(1\right)}
		\begin{rowequmatbra}{ccc|c}
			1  & 0 & -1 & 1 \\
			0  & 2 &  3 & 1 \\
			0  & 3 &  3 & 3
		\end{rowequmatbra}
	\end{equation*}
	Si \textbf{moltiplica la seconda riga per lo scalare} $\frac{1}{2}$:
	\begin{equation*}
		\begin{rowequmatbra}{ccc|c}
			1  & 0 & -1 & 1 \\
			0  & 2 &  3 & 1 \\
			0  & 3 &  3 & 3
		\end{rowequmatbra} \xlongrightarrow{E_{2}\left(\frac{1}{2}\right)}
		\begin{rowequmatbra}{ccc|c}
			1  & 0 & -1 & 1 \\
			0  & 1 &  \frac{3}{2} & \frac{1}{2} \\
			0  & 3 &  3 & 3
		\end{rowequmatbra}
	\end{equation*}
	Si \textbf{moltiplica la seconda riga per $-3$ e successivamente si somma la terza riga con la seconda}:
	\begin{equation*}
		\begin{rowequmatbra}{ccc|c}
			1  & 0 & -1 & 1 \\
			0  & 1 &  \frac{3}{2} & \frac{1}{2} \\
			0  & 3 &  3 & 3
		\end{rowequmatbra} \xlongrightarrow{E_{3,2}\left(-3\right)}
		\begin{rowequmatbra}{ccc|c}
			1  & 0 & -1 & 1 \\
			0  & 1 &  \frac{3}{2} & \frac{1}{2} \\
			0  & 0 &  -\frac{3}{2} & \frac{3}{2}
		\end{rowequmatbra}
	\end{equation*}
	Si \textbf{moltiplica la terza riga per lo scalare} $-\frac{2}{3}$:
	\begin{equation*}
		\begin{rowequmatbra}{ccc|c}
			1  & 0 & -1 & 1 \\
			0  & 1 &  \frac{3}{2} & \frac{1}{2} \\
			0  & 0 &  -\frac{3}{2} & \frac{3}{2}
		\end{rowequmatbra} \xlongrightarrow{E_{3,2}\left(-3\right)}
		\begin{rowequmatbra}{ccc|c}
			1  & 0 & -1 & 1 \\
			0  & 1 &  \frac{3}{2} & \frac{1}{2} \\
			0  & 0 &  1 & -1
		\end{rowequmatbra}
	\end{equation*}
	Si ottiene così la forma ridotta di Gauss.\newline
	
	\noindent
	Il \textcolor{Red3}{terzo passo} è classificare la forma ottenuta. Dalla forma ridotta è possibile dedurre che si è di fronte al tipo uno, ovvero esiste una sola soluzione per il sistema.
	
	Adesso è possibile ricostruire il vettore delle soluzioni andando al contrario. Quindi, si parte dall'ultima riga e sostituendo si va fino all'inizio:
	\begin{gather*}
		\begin{array}{rll}
			z & = & -1 \\
			\\
			y + \frac{3}{2}\left(-1\right) = \frac{1}{2} \rightarrow y & = & \phantom{-}2 \\
			\\
			x + -1\left(-1\right) = 1 \rightarrow x & = & \phantom{-}0
		\end{array}
	\end{gather*}\newpage

	\subsubsection{Tipo zero}
	
	Dato il seguente sistema:
	\begin{equation*}
		\begin{cases}
			x + 2y - z = 1 	\\
			-x - y + 2z = 1 \\
			x + 3y + z = 4 	\\
			2x + 4y -2z = -1
		\end{cases}
	\end{equation*}
	Si calcola la matrice risultante dopo l'eliminazione di Gauss.\newline
	
	\noindent
	Il \textcolor{Red3}{primo passo} è scrivere la matrice aumentata. Essa è banale da comporre, consiste nello scrivere i coefficienti di ogni variabile ($x,y,z$) in una matrice e aggiungere una colonna sulla destra in cui ci sono i valori risultanti. Si passa all'atto pratico:
	\begin{equation*}
		\begin{rowequmatbra}{ccc|c}
			 1  &  2 & -1 &  1 \\
			-1  & -1 &  2 &  1 \\
			 1  &  3 &  1 &  4 \\
			 2  &  4 & -2 & -1
		\end{rowequmatbra}
	\end{equation*}
	Il \textcolor{Red3}{secondo passo} è eseguire alcune considerazioni sulla forma che si vuole ottenere e procedere con le varie operazioni. L'obbiettivo è quello di ottenere una matrice uni-triangolare superiore\footnote{Una matrice uni-triangolare superiore è una forma particolare in cui i valori sotto alla diagonale principale sono nulli, cioè uguale a zero}. In questo caso, si inizia \textbf{moltiplicando la prima riga per 1 e successivamente si somma la prima riga con la seconda}:
	\begin{equation*}
		\begin{rowequmatbra}{ccc|c}
			1  &  2 & -1 &  1 \\
			-1  & -1 &  2 &  1 \\
			1  &  3 &  1 &  4 \\
			2  &  4 & -2 & -1
		\end{rowequmatbra} \xlongrightarrow{E_{2,1}\left(1\right)}
		\begin{rowequmatbra}{ccc|c}
			1  & 2 & -1 &  1 \\
			0  & 1 &  1 &  2 \\
			1  & 3 &  1 &  4 \\
			2  & 4 & -2 & -1
		\end{rowequmatbra}
	\end{equation*}
	Per rapidità, si eseguono in ordine le due operazioni seguenti. Si \textbf{moltiplica la prima riga per -1 e successivamente si somma la prima riga con la terza}, e poi si \textbf{moltiplica la prima riga per -2 e successivamente si somma la prima riga con la quarta}:
	\begin{equation*}
		\begin{rowequmatbra}{ccc|c}
			1  & 2 & -1 &  1 \\
			0  & 1 &  1 &  2 \\
			1  & 3 &  1 &  4 \\
			2  & 4 & -2 & -1
		\end{rowequmatbra} \xlongrightarrow[E_{4,1}\left(-2\right)]{E_{3,1}\left(-1\right)}
		\begin{rowequmatbra}{ccc|c}
			1  & 2 & -1 &  1 \\
			0  & 1 &  1 &  2 \\
			0  & 1 &  2 &  3 \\
			0  & 0 &  0 & -3
		\end{rowequmatbra}
	\end{equation*}
	Si ottiene così la forma ridotta di Gauss.\newline
	
	\noindent
	Il \textcolor{Red3}{terzo passo} è classificare la forma ottenuta. Dalla forma ridotta è possibile dedurre che si è di fronte al tipo zero, ovvero non esiste nessuna soluzione per il sistema. L'esercizio è concluso:
	\begin{equation*}
		\cancel{\exists} z : 0 \cdot z = -3
	\end{equation*}\newpage

	\subsubsection{Infinito}
	
	Dato il seguente sistema:
	\begin{equation*}
		\begin{cases}
			x + 2y + w = 0 	\\
			2x + 5y + 4z + 4w = 0 \\
			3x + 5y - 6z + 4w = 0
		\end{cases}
	\end{equation*}
	Si calcola la matrice risultante dopo l'eliminazione di Gauss.\newline
	
	\noindent
	Il \textcolor{Red3}{primo passo} è scrivere la matrice aumentata. Essa è banale da comporre, consiste nello scrivere i coefficienti di ogni variabile ($x,y,z,w$) in una matrice e aggiungere una colonna sulla destra in cui ci sono i valori risultanti. Si passa all'atto pratico:
	\begin{equation*}
		\begin{rowequmatbra}{cccc|c}
			1  &  2 &  0 &  1 & 0 	\\
			2  &  5 &  4 &  4 & 0	\\
			3  &  5 & -6 &  4 & 0
		\end{rowequmatbra}
	\end{equation*}
	Il \textcolor{Red3}{secondo passo} è eseguire alcune considerazioni sulla forma che si vuole ottenere e procedere con le varie operazioni. L'obbiettivo è quello di ottenere una matrice uni-triangolare superiore\footnote{Una matrice uni-triangolare superiore è una forma particolare in cui i valori sotto alla diagonale principale sono nulli, cioè uguale a zero}. In questo caso, si inizia con due operazioni per velocizzare i calcoli. Si \textbf{moltiplica la prima riga per -2 e successivamente si somma la prima riga con la seconda}, e poi si \textbf{moltiplica la prima riga per -3 e successivamente si somma la prima riga con la terza}:
	\begin{equation*}
		\begin{rowequmatbra}{cccc|c}
			1  &  2 &  0 &  1 & 0 	\\
			2  &  5 &  4 &  4 & 0	\\
			3  &  5 & -6 &  4 & 0
		\end{rowequmatbra} \xlongrightarrow[E_{3,1}\left(-3\right)]{E_{2,1}\left(-2\right)}
		\begin{rowequmatbra}{cccc|c}
			1  &  2 &  0 &  1 & 0 	\\
			0  &  1 &  4 &  2 & 0	\\
			0  & -1 & -6 &  1 & 0
		\end{rowequmatbra}
	\end{equation*}
	Si \textbf{moltiplica la seconda riga per 1 e successivamente si somma la seconda riga con la terza}:
	\begin{equation*}
		\begin{rowequmatbra}{cccc|c}
			1  &  2 &  0 &  1 & 0 	\\
			0  &  1 &  4 &  2 & 0	\\
			0  & -1 & -6 &  1 & 0
		\end{rowequmatbra} \xlongrightarrow{E_{3,2}\left(1\right)}
		\begin{rowequmatbra}{cccc|c}
			1  &  2 &  0 &  1 & 0 	\\
			0  &  1 &  4 &  2 & 0	\\
			0  &  0 & -2 &  3 & 0
		\end{rowequmatbra}
	\end{equation*}
	Si \textbf{moltiplica la terza riga per uno scalare $\frac{1}{2}$}:
	\begin{equation*}
		\begin{rowequmatbra}{cccc|c}
			1  &  2 &  0 &  1 & 0 	\\
			0  &  1 &  4 &  2 & 0	\\
			0  &  0 & -2 &  3 & 0
		\end{rowequmatbra} \xlongrightarrow{E_{3}\left(-\frac{1}{2}\right)}
		\begin{rowequmatbra}{cccc|c}
			1  &  2 &  0 &  1 & 0 	\\
			0  &  1 &  4 &  2 & 0	\\
			0  &  0 &  1 &  -\frac{3}{2} & 0
		\end{rowequmatbra}
	\end{equation*}
	Si ottiene così la forma ridotta di Gauss.\newpage
	
	\noindent
	Il \textcolor{Red3}{terzo passo} è classificare la forma ottenuta. Dalla forma ridotta è possibile dedurre che si è di fronte all'infinito, ovvero esistono un'infinità di soluzioni che dipendono da, in questo caso, un parametro:
	\begin{equation*}
		\begin{array}{rll}
			z - \dfrac{3}{2} w = 0 & \longrightarrow & z = \dfrac{3}{2} w \\
			\\
			y + 4\left(\dfrac{3}{2}w\right) + 2w = 0 & \longrightarrow & y = -8w \\
			\\
			x + 2\left(-8w\right) + 1w = 0 & \longrightarrow & x = 15w
		\end{array}
	\end{equation*}
	Quindi il vettore soluzione sarà composto in questo modo:
	\begin{equation*}
		soluzione = \begin{bmatrix}
			15w \\
			\\
			-8w \\
			\\
			\dfrac{3}{2}w \\
			\\
			w
		\end{bmatrix} = w \cdot \begin{bmatrix}
			15 \\
			\\
			-8 \\
			\\
			\dfrac{3}{2} \\
			\\
			1
		\end{bmatrix}
	\end{equation*}\newpage

	\subsubsection{Infinito - Caso particolare}
	
	Dato il seguente sistema:
	\begin{equation*}
		\begin{cases}
			x + y + z + w = -1 	 \\
			x + 2y + z + 2w = -1 \\
			2x + 3y + 2z + 3w = -2
		\end{cases}
	\end{equation*}
	Si calcola la matrice risultante dopo l'eliminazione di Gauss.\newline
	
	\noindent
	Il \textcolor{Red3}{primo passo} è scrivere la matrice aumentata. Essa è banale da comporre, consiste nello scrivere i coefficienti di ogni variabile ($x,y,z,w$) in una matrice e aggiungere una colonna sulla destra in cui ci sono i valori risultanti. Si passa all'atto pratico:
	\begin{equation*}
		\begin{rowequmatbra}{cccc|c}
			1  &  1 &  1 &  1 & -1 	\\
			1  &  2 &  1 &  2 & -1	\\
			2  &  3 &  2 &  3 & -2
		\end{rowequmatbra}
	\end{equation*}
	Il \textcolor{Red3}{secondo passo} è eseguire alcune considerazioni sulla forma che si vuole ottenere e procedere con le varie operazioni. L'obbiettivo è quello di ottenere una matrice uni-triangolare superiore\footnote{Una matrice uni-triangolare superiore è una forma particolare in cui i valori sotto alla diagonale principale sono nulli, cioè uguale a zero}. In questo caso, si inizia con due operazioni per velocizzare i calcoli. Si \textbf{moltiplica la prima riga per -1 e successivamente si somma la prima riga con la seconda}, e poi si \textbf{moltiplica la prima riga per -2 e successivamente si somma la prima riga con la terza}:
	\begin{equation*}
		\begin{rowequmatbra}{cccc|c}
			1  &  1 &  1 &  1 & -1 	\\
			1  &  2 &  1 &  2 & -1	\\
			2  &  3 &  2 &  3 & -2
		\end{rowequmatbra} \xlongrightarrow[E_{2,1}\left(-1\right)]{E_{3,1}\left(-2\right)}
		\begin{rowequmatbra}{cccc|c}
			1  &  1 &  1 &  1 & -1 	\\
			0  &  1 &  0 &  1 &  0	\\
			0  &  1 &  0 &  1 &  0
		\end{rowequmatbra}
	\end{equation*}
	Si \textbf{moltiplica la seconda riga per -1 e successivamente si somma la seconda riga con la terza}:
	\begin{equation*}
		\begin{rowequmatbra}{cccc|c}
			1  &  1 &  1 &  1 & -1 	\\
			0  &  1 &  0 &  1 &  0	\\
			0  &  1 &  0 &  1 &  0
		\end{rowequmatbra} \xlongrightarrow{E_{3,2}\left(-1\right)}
		\begin{rowequmatbra}{cccc|c}
			1  &  1 &  1 &  1 & -1 	\\
			0  &  1 &  0 &  1 &  0	\\
			0  &  0 &  0 &  0 &  0
		\end{rowequmatbra}
	\end{equation*}
	Si ottiene così la forma ridotta di Gauss.\newline
	
	\noindent
	Il \textcolor{Red3}{terzo passo} è classificare la forma ottenuta. Dalla forma ridotta è possibile dedurre che si è di fronte all'infinito, ovvero esistono un'infinità di soluzioni che dipendono da, in questo caso, due parametri. Le variabili $z$ e $w$ sono libere:
	\begin{equation*}
		\begin{array}{rll}
			1y + 1w = 0 & \longrightarrow & y = -w \\
			\\
			x - w = -1 & \longrightarrow & x = w - 1
		\end{array}
	\end{equation*}
	Quindi il vettore soluzione sarà composto in questo modo:
	\begin{equation*}
		soluzione = \begin{bmatrix}
			w-1 \\
			-w 	\\
			w	\\
			z
		\end{bmatrix}
	\end{equation*}
\end{document}