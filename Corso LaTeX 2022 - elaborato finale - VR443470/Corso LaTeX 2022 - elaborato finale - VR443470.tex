\documentclass[a4paper]{article}
\usepackage[T1]{fontenc}			% pacchetto per \chapter
\usepackage[italian]{babel}
\usepackage[italian]{isodate}  		% formato delle date in italiano
\usepackage{graphicx}				% gestione delle immagini
\usepackage{amsfonts}
\usepackage{booktabs}				% tabelle di qualità superiore
\usepackage{amsmath}				% pacchetto matematica
\usepackage{enumitem}				% gestione delle liste
\usepackage{pifont}					% pacchetto con elenchi carini
% Link ipertestuali per l'indice
\usepackage{xcolor}
\usepackage[linkcolor=black, citecolor=blue, urlcolor=cyan]{hyperref}
\hypersetup{
	colorlinks=true
}

%\usepackage{showframe}				% visualizzazione bordi
%\usepackage{showkeys}				% visualizzazione etichetta

%\newtheorem{theorem}{Teorema}[section]	% definizione teorema

% Definizione nuovi comandi
\newcommand{\Z}{\mathcal{Z}}
\newcommand{\F}{\mathcal{F}}
\newcommand{\customimg}[2]{\includegraphics[width=#1\textwidth]{#2}}

% lista personalizzata di primo livello
\setlist[itemize, 1]{label=\ding{217}}

\begin{document}
	% logo università di Verona
	\begin{figure}
		\centering
		\customimg{0.3}{img/logo-univr.jpg}
		\label{img-univr}
	\end{figure}

	\author{VR443470}
	\title{Corso \LaTeX\:2022 - Elaborato finale}
	\date{\printdayoff\today}
	\maketitle
	
	\newpage
	
	% indice
	\tableofcontents
	
	\newpage
	
	\section{Campionamento e quantizzazione}
	
	\subsection{Campionamento}
	
	Il \textbf{campionamento} è la tecnica che consente di effettuare la \underline{trasformazione} da analogico (\emph{segnale continuo}) a digitale (\emph{segnale discreto}). In particolare, è il dominio che viene trasformato da continuo a discreto.
	La trasformazione avviene senza perdita di informazioni, se rispettate alcune condizioni (\textbf{teorema del campionamento ideale}), e il segnale sarà ricostruito perfettamente (\textbf{formula di interpolazione ideale di Shannon}).
	Graficamente, per eseguire il campionamento di un segnale continuo, sarà necessario campionare ad un certo intervallo $T$ il segnale, come nel seguente esempio, ottenendo valori reali:
	
	\begin{figure}[!htp]
		\centering
		\customimg{0.7}{img/campionamento.pdf}
		\caption{In verde il segnale originale, in grigio il segnale campionato nel tempo.}
		\label{img-camp_esempio}
	\end{figure}
	
	\subsection{Quantizzazione}
	
	La la \textbf{quantizzazione} consente di effettuare la trasformazione del codominio entro certi intervalli. Per capire meglio, si prenda in considerazione il seguente grafico:
	
	\begin{figure}[!htp]
		\centering
		\customimg{0.7}{img/quantizzazione.pdf}
		\caption{Segnale sinusoidale quantizzato nelle ampiezze e il suo segnale originale.}
		\label{img-quant_esempio}
	\end{figure}

	Gli scalini che sono visibili sulla linea del segnale, sono i vari campionamenti effettuati durante il tempo, ovvero lungo l’asse delle ascisse $t$. A differenza del campionamento in cui i valori campionati potevano essere acquisiti liberamente, nella quantizzazione i valori devono essere uguali a determinati valori presenti sull’asse delle ordinate $v(t)$. Quindi, ipotizzando che sull’asse $v(t)$ ci siano numeri interi $(0, 1, 2, 3, ...)$ la punta del primo scalino indicherà il valore $1$, la seconda punta dello scalino, spostandosi lungo l'asse del tempo $t$, indicherà due e così via.
	
	\section{Replicazione e campionamento}\label{sec-rep_e_camp}
	
	In questo capitolo vengono introdotti i concetti di replicazione e campionamento.
	
	\subsection{Replicazione}\label{subsec-rep}

	Si definisce \textbf{treno campionatore} ideale di periodo$T$ la funzione:
	\begin{equation}\label{eq-treno_camp}
		\tilde{\delta}_T(t) = \sum_{k=-\infty}^{\infty} \delta(t-kT)
	\end{equation}

	Il treno campionatore è sostanzialmente una sequenza infinita di impulsi ideali, come in figura, di supporto $\{t = kT \in \mathbb{R}\:|\:k \in \mathbb{Z}\}$.
	
	\begin{figure}[!htp]
		\centering
		\customimg{0.5}{img/treno-campionatore.pdf}
		\caption{Treno campionatore $\tilde{\delta}_T(t)$.}
		\label{img-treno_camp}
	\end{figure}

	Inoltre, si ha che:
	\begin{equation*}
		\tilde{\delta}_T(t) \underset{\longleftrightarrow}{\F} \dfrac{1}{T} \sum_{k=-\infty}^{\infty} \delta\left(f-\dfrac{k}{T}\right)
	\end{equation*}
	
	Che equivale a:
	\begin{equation*}
		\tilde{\delta}_T(t) \underset{\longleftrightarrow}{\F} \dfrac{1}{T} \tilde{\delta}_{\frac{1}{T}}(f)
	\end{equation*}

	Infine, la trasformata di Fourier del treno campionatore ideale è un treno campionatore ideale (in frequenza) in cui gli impulsi hanno l’area $\dfrac{1}{T}$ e sono equispaziati alla frequenza $\dfrac{1}{T}$ (come in figura~\ref*{img-tdf_treno_camp}).
	
	\begin{figure}[!htp]
		\centering
		\customimg{0.5}{img/tdf-treno-campionatore.pdf}
		\caption{La Trasformata di Fourier del treno campionatore~(\ref{eq-treno_camp}) con periodo $T=2$.}
		\label{img-tdf_treno_camp}
	\end{figure}
	
	Dato un segnale $v(t), t\in\mathbb{R}, 0<T\in\mathbb{R}$, si definisce \textbf{campionamento} del segnale $v$ come:	
	\begin{equation}\label{eq-camp}
		[\mathrm{samp}_T\:v](t) := \sum_{k=-\infty}^{\infty} v(kT)
	\end{equation}
	
	Allora, per la proprietà di campionamento dell’impulso, si ha:
	\begin{equation}\label{eq-camp_imp}
		[\mathrm{samp}_T\:v](t) = \sum_{k=-\infty}^{\infty} v(t)\delta(t-kT) = v(t)\cdot\tilde{\delta}_T(t)
	\end{equation}
	
	\begin{figure}[!htp]
		\centering
		\customimg{0.7}{img/camp-impulso.pdf}
		\caption{Campionamento di un segnale.}
		\label{img-camp}
	\end{figure}
	
	\section{Sistemi a tempo discreto}\label{sec-sis_tmp_disc}
	
	Si definisce \textbf{sistema a tempo discreto} un sistema i cui elementi descrittivi, come funzioni di \emph{ingresso} e \emph{uscita}, sono successioni $(a_{k})_{k\in\mathbb{Z}}$, cioè funzioni a variabile discreta.
	
	\subsection{Proprietà sistemi a tempo discreto}
	
	Le proprietà dei sistemi a tempo discreto sono le stesse di quelle dei sistemi a tempo continuo.
	
	\begin{itemize}
		\item Linearità
		\item Tempo invarianza
		\item Casualità
		\item Stabilità asintotica
		\item \textsc{BIBO} stabilità
	\end{itemize}

	\newpage

	\section{Trasformata zeta}
	
	Sia $v(k)$ una successione a valori reali o complessi. Si definisce \textbf{trasformata zeta}:
	
	\begin{equation}
		\Z [v(k)] (z) := \sum_{k=-\infty}^{\infty} v(k) z^{-k} = V(z)
	\end{equation}

	La funzione $V:\mathbb{C}\rightarrow\mathbb{C}$, con $z\in\mathbb{C}$. Essa è definita per tutti i numeri complessi $z$ per cui la serie è convergente.
	
	\subsection{Proprietà della trasformata zeta}
	
	% Ho cambiato lo stile dell'elenco puntato solamente a scopo illustrativo.
	\begin{enumerate}
		\item Linearità
		\item Moltiplicazione per successione esponenziale
		\item Moltiplicazione per un monomio
		\item Ritardo temporale
		\item Anticipo temporale
	\end{enumerate}
	
	\newpage
	
	\section{Tabella riassuntiva trasformate notevoli}
	
	\begin{table}[!htbp]
		\centering
	
		\begin{tabular}{@{} c c @{}}
			\toprule
			Segnale & Trasformata $\Z$ \\
			\midrule
			$\mathit{A}$ 					& $A \cdot \dfrac{z}{z-1}$ 				\\
			$A\delta(k)$ 					& $A$									\\
			$A\delta(k-i)$					& $A z^{-i}$							\\
			$A\delta_{-1}(k)$				& $A \cdot \dfrac{z}{z-1}$				\\
			$\lambda^k \delta_{-1} (k)$		& $\dfrac{z}{z-\lambda}$				\\
			$k\lambda^k \delta_{-1} (k)$	& $\dfrac{\lambda z}{(z-\lambda)^2}$	\\
			\bottomrule
		\end{tabular}
	
		\caption{Trasformate notevoli}
		\label{tab-trasf_not}
	\end{table}
	
\end{document}