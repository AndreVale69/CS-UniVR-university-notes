\documentclass[a4paper]{article}
\usepackage[T1]{fontenc}			% pacchetto per \chapter
\usepackage[italian]{babel}
\usepackage[italian]{isodate}  		% formato delle date in italiano
\usepackage{graphicx}				% gestione delle immagini
\usepackage{amsfonts}
\usepackage{booktabs}				% tabelle di qualità superiore
\usepackage{amsmath}				% pacchetto matematica
\usepackage{enumitem}				% gestione delle liste
\usepackage{pifont}					% pacchetto con elenchi carini
\usepackage[x11names]{xcolor}		% pacchetto colori RGB
% Link ipertestuali per l'indice
\usepackage{xcolor}
\usepackage[linkcolor=black, citecolor=blue, urlcolor=cyan]{hyperref}
\hypersetup{
	colorlinks=true
}

%\usepackage{showframe}				% visualizzazione bordi
%\usepackage{showkeys}				% visualizzazione etichetta

\begin{document}
	% indice
	\tableofcontents
	
	\newpage
		
	\section{Introduzione}
	
	\subsection{Sistemi informativi, informazioni e dati}
	
	Ogni organizzazione è dotata di un \textbf{\emph{sistema informativo}}, che organizza e gestisce le informazioni necessarie per perseguire gli scopi dell'organizzazione stessa. Per indicare la \textbf{porzione automatizzata del sistema informativo} viene di solito utilizzato il termine \textbf{\emph{sistema informatico}}.\newline
	Nei sistemi informatici le informazioni vengono rappresentate per mezzo di \emph{dati}, che hanno bisogno di essere interpretati per fornire informazioni.\newline
	Esiste una differenza sottile tra dato e informazioni. Solitamente i primi, se presi da soli, non hanno significato, ma, una volta interpretati e correlati opportunamente, essi forniscono informazioni, che consentono di arricchire la conoscenza:

	\begin{description}
		\item[\textbf{Informazione}:] notizia, dato o elemento che consente di avere conoscenza più o meno esatta di fatti, situazioni, modi di essere;
		\item[\textbf{Dato}:] ciò che è immediatamente presente alla conoscenza, prima di ogni elaborazione. In informatica, sono elementi di informazione costituiti da simboli che devono essere elaborati.
	\end{description}
	
	\noindent
	\textcolor{Red3}{\textbf{[ESAME] Definizione base di dati}}: Una \textbf{\emph{base di dati}} è una collezione di dati, utilizzati per rappresentare con tecnologia informatica le informazioni di interesse per un sistema informativo.\newline
	
	\newpage
	
	
	
	\subsection{Basi di dati e sistemi di gestione di basi di dati}
	
	Inizialmente, venne adottato un ``\underline{approccio convenzionale}'' alla gestione dei dati. Esso \textbf{sfruttava} la presenza di archivi o \textbf{file per memorizzare} e \textbf{per ricercare dati}. Tuttavia, i metodi di accesso e condivisione erano semplici e banali.\newline
	Infatti, erano presenti numerosi \textbf{problemi}:
	
	\begin{itemize}
		\item[\ding{56}] \textbf{Accesso sequenziale}: la scarsa efficienza nell'accesso ai dati su file rendeva lento l'accesso a tali informazioni;
		\item[\ding{56}] \textbf{Ridondanza}: i dati di interesse per più programmi sono replicati tante volte quanti sono i programmi che li utilizzano, con evidente ridondanza e possibilità di incoerenza;
		\item[\ding{56}] \textbf{Inconsistenza}: una diretta conseguenza della ridondanza. Con la presenza di più copie di un determinato dato, l'eventuale cambiamento di uno solo potrebbe portare a questo effetto;
		\item[\ding{56}] \textbf{Progettazione duplicata}: per ogni programma viene replicata la progettazione.
	\end{itemize}
	
	La \textbf{soluzione} è arrivata negli anni '80 con l'avvento delle \textbf{basi di dati}. Quest'ultime gestiscono in modo integrato e flessibile le informazioni di interesse per diversi soggetti.\newline
	\newline \noindent	
	\textcolor{Red3}{\textbf{[ESAME] Definizione DBMS}}: Un \textbf{\emph{sistema di gestione di basi di dati}} (in inglese \emph{Data Base Management System}, \textbf{DBMS}) è un sistema software in grado di gestire collezioni di dati che siano:
	
	\begin{itemize}
		\item[\ding{52}] \textbf{Grandi};
		\item[\ding{52}] \textbf{Condivise};
		\item[\ding{52}] \textbf{Persistenti}.
	\end{itemize}
	
	\noindent
	\underline{assicurando} allo stesso tempo:
	
	\begin{itemize}
		\item[\ding{72}] \textbf{Affidabilità};
		\item[\ding{72}] \textbf{Privatezza};
		\item[\ding{72}] \textbf{Accesso efficiente}.
	\end{itemize}

	Il \textbf{vantaggio} di utilizzare un DBMS è stato evidenziato nella definizione. Quindi:
	
	\begin{itemize}
		\item[\ding{51}] \textbf{Maggiore astrazione} poiché le sue funzioni \underline{estendono il \emph{file system}}, fornendo la possibilità di accesso condiviso agli stessi dati da parte di più utenti e applicazioni;
		\item[\ding{51}] \textbf{Maggiore efficacia} poiché le operazioni di accesso ai dati si basano su un linguaggio di interrogazione.
	\end{itemize}
	
	
	
	
	\subsection{Linguaggi per basi di dati}
	Su un DBMS è possibile specificare operazioni di vario tipo, ma principalmente si distinguono in due categorie:

	\begin{itemize}
		\item \textbf{Linguaggi di definizione dei dati} (\emph{Data Definition Language}, abbreviato con \textbf{DDL}) utilizzati per \underline{definire} gli \underline{schemi logici}, \underline{esterni} e \underline{fisici} e le \underline{autorizzazioni per l'accesso};
		\item \textbf{Linguaggi di manipolazione dei dati} (\emph{Data Manipulation Language}, abbreviato con \textbf{DML}) \underline{utilizzati} per l'\underline{interrogazione} e l'\underline{aggiornamento} delle \underline{istanze} di basi di dati:
		\begin{itemize}
			\item \emph{Linguaggio di interrogazione}, estrae informazioni da una base di dati (SQL, algebra relazionale);
			\item \emph{Linguaggio di manipolazione}, popola la base di dati, modifica il suo contenuto con aggiunte, cancellazioni e variazioni sui dati (SQL).
		\end{itemize}
	\end{itemize}
	
	\newpage
	
	
	
	
	\subsection{Modelli dei dati}
	\textcolor{Red3}{\textbf{Definizione modello dei dati}}: Un \textbf{\emph{modello dei dati}} è un insieme di concetti utilizzati per organizzare i dati di interesse e descriverne la struttura in modo che essa risulti comprensibile ad un elaboratore.\newline
	Ogni modello dei dati fornisce \textbf{meccanismi di strutturazione}, analoghi ai \textbf{\emph{costruttori}} di tipo dei linguaggi di programmazione (es: Java), che permettono di definire nuovi tipi sulla base di tipi predefiniti (elementari) e costruttori di tipo.\newline
	Quindi, i \emph{costruttori} consentono di:
	
	\begin{itemize}
		\item[\ding{43}] \textbf{\emph{Definire}} le \textbf{strutture dati che conterranno le informazioni} della base di dati;
		\item[\ding{43}] \textbf{\emph{Specificare}} le \textbf{proprietà che dovranno soddisfare le istanze} di informazione che saranno contenuto nelle strutture dati.
	\end{itemize}

	\noindent \newline
	\textcolor{Red3}{\textbf{Definizione schemi e istanze}}: È molto importante distinguere gli \textbf{schemi} e le \textbf{istanze} dal concetto di modello dei dati:
	
	\begin{itemize}
		\item \textbf{\emph{Schema}}: parte \underline{invariante nel tempo}, è costituita dalle caratteristiche dei dati. In altre parole, è la descrizione della struttura e delle proprietà di una specifica base di dati fatta utilizzando i costrutti del modello dei dati;
		\item \textbf{\emph{Istanza} o \emph{stato}}: parte \underline{variabile nel tempo}, è costituita dai valori effettivi. Quest'ultimi, in un certo istante, popolano le strutture dati della base di dati.
	\end{itemize}
\end{document}